\documentclass[a4paper]{article}
\usepackage[utf8]{inputenc}
\usepackage{geometry}
\usepackage[pagebackref=true]{hyperref}
\usepackage{lmodern}
\usepackage{amsmath}
\usepackage{amssymb}
\usepackage{pifont}
\usepackage{bbm}
\usepackage{stmaryrd}
\usepackage{mathtools}

\newcommand{\followup}[1]{\textcolor{red}{ #1 }}
\newcommand{\Ubr}[2]{\underbrace{ #1 }_{\mathclap{\text{ #2 }}}}
\newcommand{\Br}[1]{\{ #1 \}}

\geometry{margin=2cm}

\title{Introduction to Algorithms - Study Notes}
\author{Isaac Riley}
\date{Last edited: \today}

\begin{document}
\maketitle
\tableofcontents
\newpage



%%%%%%%%%%%%%%%%%%%%%%%%%%%%%%%%%%%%%%%%%%%%%%%%%%%%%%%%%%%%%%%
%========================================================
%--------------------------------------------------

\section*{A Summations}
\subsection*{A.1 Summation formulas and properties}
\subsection*{A.2 Bounding summations}

\newpage
\section*{B Sets, Etc.}
\subsection*{B.1 Sets}
\subsection*{B.2 Relations}
\subsection*{B.3 Functions}
\subsection*{B.4 Graphs}
\subsection*{B.5 Trees}

\newpage
\section*{C Counting and Probability}
\subsection*{C.1 Counting}
\subsection*{C.2 Probability}
\subsection*{C.3 Discrete random variables}
\subsection*{C.4 The geometric and binomial distributions}
\subsection*{C.5 The tails of the binomial distribution}

\newpage
\section*{D Matrices}
\subsection*{D.1 Matrices and matrix operations}
\subsection*{D.2 Basic matrix properties}

\newpage
\section*{I Foundations}
\subsection*{Introduction 3}

\newpage
\section{The Role of Algorithms in Computing}
\subsection{Algorithms}
\subsection{Algorithms as a technology}

\newpage
\section{Getting Started}
\subsection{Insertion sort}
\subsection{Analyzing algorithms}
\subsection{Designing algorithms}

\newpage
\section{Growth of Functions}
\subsection{Asymptotic notation}
\subsection{Standard notations and common functions}

\newpage
\section{Divide-and-Conquer}
\subsection{The maximum-subarray problem}
\subsection{Strassen’s algorithm for matrix multiplication}
\subsection{The substitution method for solving recurrences}
\subsection{The recursion-tree method for solving recurrences}
\subsection{The master method for solving recurrences}
\subsection{Proof of the master theorem}

\newpage
\section{5 Probabilistic Analysis and Randomized Algorithms}
\subsection{5.1 The hiring problem}
\subsection{5.2 Indicator random variables}
\subsection{5.3 Randomized algorithms}
\subsection{5.4 Probabilistic analysis and further uses of indicator random variables}

\newpage
\section*{II Sorting and Order Statistics}
\subsection*{Introduction}

\newpage
\section{6 Heapsort}
\subsection{6.1 Heaps}
\subsection{6.2 Maintaining the heap property}
\subsection{6.3 Building a heap}
\subsection{6.4 The heapsort algorithm}
\subsection{6.5 Priority queues}

\newpage
\section{7 Quicksort}
\subsection{7.1 Description of quicksort}
\subsection{7.2 Performance of quicksort}
\subsection{7.3 A randomized version of quicksort}
\subsection{7.4 Analysis of quicksort}

\newpage
\section{8 Sorting in Linear Time}
\subsection{8.1 Lower bounds for sorting}
\subsection{8.2 Counting sort}
\subsection{8.3 Radix sort}
\subsection{8.4 Bucket sort}

\newpage
\section{9 Medians and Order Statistics}
\subsection{9.1 Minimum and maximum}
\subsection{9.2 Selection in expected linear time}
\subsection{9.3 Selection in worst-case linear time}

\newpage
\section*{III Data Structures}
\subsection*{Introduction}

\newpage
\section{10 Elementary Data Structures}
\subsection{10.1 Stacks and queues}
\subsection{10.2 Linked lists}
\subsection{10.3 Implementing pointers and objects}
\subsection{10.4 Representing rooted trees}

\newpage
\section{11 Hash Tables}
\subsection{11.1 Direct-address tables}
\subsection{11.2 Hash tables}
\subsection{11.3 Hash functions}
\subsection{11.4 Open addressing}
\subsection{11.5 Perfect hashing}

\newpage
\section{12 Binary Search Trees}
\subsection{12.1 What is a binary search tree?}
\subsection{12.2 Querying a binary search tree}
\subsection{12.3 Insertion and deletion}
\subsection{12.4 Randomly built binary search trees}

\newpage
\section{13 Red-Black Trees}
\subsection{13.1 Properties of red-black trees}
\subsection{13.2 Rotations}
\subsection{13.3 Insertion}
\subsection{13.4 Deletion}

\newpage
\section{14 Augmenting Data Structures}
\subsection{14.1 Dynamic order statistics}
\subsection{14.2 How to augment a data structure}
\subsection{14.3 Interval trees}

\newpage
\section{IV Advanced Design and Analysis Techniques}
\subsection{Introduction}

\newpage
\section{15 Dynamic Programming}
\subsection{15.1 Rod cutting}
\subsection{15.2 Matrix-chain multiplication}
\subsection{15.3 Elements of dynamic programming}
\subsection{15.4 Longest common subsequence}
\subsection{15.5 Optimal binary search trees}

\newpage
\section{16 Greedy Algorithms}
\subsection{16.1 An activity-selection problem}
\subsection{16.2 Elements of the greedy strategy}
\subsection{16.3 Huffman codes}
\subsection{16.4 Matroids and greedy methods}
\subsection{16.5 A task-scheduling problem as a matroid}

\newpage
\section{17 Amortized Analysis}
\subsection{17.1 Aggregate analysis}
\subsection{17.2 The accounting method}
\subsection{17.3 The potential method}
\subsection{17.4 Dynamic tables}

\newpage
\section*{V Advanced Data Structures}
\subsection*{Introduction}

\newpage
\section{18 B-Trees}
\subsection{18.1 Definition of B-trees}
\subsection{18.2 Basic operations on B-trees}
\subsection{18.3 Deleting a key from a B-tree}

\newpage
\section{19 Fibonacci Heaps}
\subsection{19.1 Structure of Fibonacci heaps}
\subsection{19.2 Mergeable-heap operations}
\subsection{19.3 Decreasing a key and deleting a node}
\subsection{19.4 Bounding the maximum degree}

\newpage
\section{20 van Emde Boas Trees}
\subsection{20.1 Preliminary approaches}
\subsection{20.2 A recursive structure}
\subsection{20.3 The van Emde Boas tree}

\newpage
\section{21 Data Structures for Disjoint Sets}
\subsection{21.1 Disjoint-set operations}
\subsection{21.2 Linked-list representation of disjoint sets}
\subsection{21.3 Disjoint-set forests}
\subsection{21.4 Analysis of union by rank with path compression}

\newpage
\section*{VI Graph Algorithms}
\subsection*{Introduction}

\newpage
\section{22 Elementary Graph Algorithms}
\subsection{22.1 Representations of graphs}
\subsection{22.2 Breadth-first search}
\subsection{22.3 Depth-first search}
\subsection{22.4 Topological sort}
\subsection{22.5 Strongly connected components}

\newpage
\section{23 Minimum Spanning Trees}
\subsection{23.1 Growing a minimum spanning tree}
\subsection{23.2 The algorithms of Kruskal and Prim}

\newpage
\section{24 Single-Source Shortest Paths}
\subsection{24.1 The Bellman-Ford algorithm}
\subsection{24.2 Single-source shortest paths in directed acyclic graphs}
\subsection{24.3 Dijkstra’s algorithm}
\subsection{24.4 Difference constraints and shortest paths}
\subsection{24.5 Proofs of shortest-paths properties}

\newpage
\section{25 All-Pairs Shortest Paths}
\subsection{25.1 Shortest paths and matrix multiplication}
\subsection{25.2 The Floyd-Warshall algorithm}
\subsection{25.3 Johnson’s algorithm for sparse graphs}

\newpage
\section{26 Maximum Flow}
\subsection{26.1 Flow networks}
\subsection{26.2 The Ford-Fulkerson method}
\subsection{26.3 Maximum bipartite matching}
\subsection{26.4 Push-relabel algorithms}
\subsection{26.5 The relabel-to-front algorithm}

\newpage
\section{VII Selected Topics}
\subsection{Introduction}

\newpage
\section{27 Multithreaded Algorithms}
\subsection{27.1 The basics of dynamic multithreading}
\subsection{27.2 Multithreaded matrix multiplication}
\subsection{27.3 Multithreaded merge sort}

\section{28 Matrix Operations}
\subsection{28.1 Solving systems of linear equations}
\subsection{28.2 Inverting matrices}
\subsection{28.3 Symmetric positive-definite matrices and least-squares approximation}

\section{29 Linear Programming}
\subsection{29.1 Standard and slack forms}
\subsection{29.2 Formulating problems as linear programs}
\subsection{29.3 The simplex algorithm}
\subsection{29.4 Duality}
\subsection{29.5 The initial basic feasible solution}

\section{30 Polynomials and the FFT}
\subsection{30.1 Representing polynomials}
\subsection{30.2 The DFT and FFT}
\subsection{30.3 Efficient FFT implementations}

\section{31 Number-Theoretic Algorithms}
\subsection{31.1 Elementary number-theoretic notions}
\subsection{31.2 Greatest common divisor}
\subsection{31.3 Modular arithmetic}
\subsection{31.4 Solving modular linear equations}
\subsection{31.5 The Chinese remainder theorem}
\subsection{31.6 Powers of an element}
\subsection{31.7 The RSA public-key cryptosystem}
\subsection{31.8 Primality testing}
\subsection{31.9 Integer factorization}

\section{32 String Matching}
\subsection{32.1 The naive string-matching algorithm}
\subsection{32.2 The Rabin-Karp algorithm}
\subsection{32.3 String matching with finite automata}
\subsection{32.4 The Knuth-Morris-Pratt algorithm}

\section{33 Computational Geometry}
\subsection{33.1 Line-segment properties}
\subsection{33.2 Determining whether any pair of segments intersects}
\subsection{33.3 Finding the convex hull}
\subsection{33.4 Finding the closest pair of points}

\section{34 NP-Completeness}
\subsection{34.1 Polynomial time}
\subsection{34.2 Polynomial-time verification}
\subsection{34.3 NP-completeness and reducibility}
\subsection{34.4 NP-completeness proofs}
\subsection{34.5 NP-complete problems}

\section{35 Approximation Algorithms}
\subsection{35.1 The vertex-cover problem}
\subsection{35.2 The traveling-salesman problem}
\subsection{35.3 The set-covering problem}
\subsection{35.4 Randomization and linear programming}
\subsection{35.5 The subset-sum problem}

\newpage
\section*{VIII Appendix: Mathematical Background}
\subsection*{Introduction}

\end{document}
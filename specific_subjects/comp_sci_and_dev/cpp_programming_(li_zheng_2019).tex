\documentclass[a4paper]{article}
\usepackage[utf8]{inputenc}
\usepackage{geometry}
\usepackage[pagebackref=true]{hyperref}
\usepackage{lmodern}
\usepackage{amsmath}
\usepackage{amssymb}
\usepackage{pifont}
\usepackage{bbm}
\usepackage{stmaryrd}
\usepackage{mathtools}
\usepackage[dvipsnames]{xcolor}

\newcommand{\followup}[1]{\textcolor{red}{#1}}
\newcommand{\obs}[1]{\textcolor{ForestGreen}{\textbf{#1}}}
\newcommand{\Ubr}[2]{\underbrace{ #1 }_{\mathclap{\text{ #2 }}}}
\newcommand{\Br}[1]{\{ #1 \}}

\geometry{margin=2cm}

\title{C++ Programming (Li Zheng et al. 2019) - Study Notes}
\author{Isaac Riley}
\date{Last edited: \today}

\begin{document}
\maketitle
\tableofcontents
\newpage



%%%%%%%%%%%%%%%%%%%%%%%%%%%%%%%%%%%%%%%%%%%%%%%%%%%%%%%%%%%%%%%

%========================================================

%--------------------------------------------------


\newpage
%%%%%%%%%%%%%%%%%%%%%%%%%%%%%%%%%%%%%%%%%%%%%%%%%%%%%%%%%%%%%%%
\section{Introduction}
%========================================================
\subsection{The Development of Computer Programming Languages}
\begin{itemize}
    \item \obs{Interesting phrasing: assembly language and high-level languages as ``closing the gap'' between computers and machines}
    \item Dichotomy: static vs dynamic features
\end{itemize}

%========================================================
\subsection{Object-Oriented Method}
Structured programming:
\begin{itemize}
    \item refinement of procedural programming
    \item two main ideas: 
    \begin{enumerate}  
        \item top-down structure
        \item stepwise refinement
    \end{enumerate}
    \item 3 basic structures:
    \begin{enumerate}
        \item sequential structure
        \item branch structure
        \item loop structure
    \end{enumerate}
    \item \obs{object: data and its operations as an interdependent and inseparable whole}
    \item \obs{class: abstraction of object; class : object $\sim$ mold : cast}
    \item \obs{encapsulation: hide certain aspects; ``show'' only what is necessary;  provide clean interface for interacting with the object}
    \item \obs{inheritance: make it possible to create a new class from an old class, which will inherit the characteristics of the old (base) class; facilitates reusability}
    \item \obs{polymorphism: allowing for variability in object characteristics across inheritance}
\end{itemize}


%========================================================
\subsection{Object-Oriented Software Development}
abbreviations:
\begin{itemize}
    \item OOA: object-oriented analysis
    \item OOD: 
    \item OOP: 
    \item OOT: 
    \item OOSM: 
    \item 
\end{itemize}

%========================================================
\subsection{Representation and Storage of Information}


%========================================================
\subsection{The Development and Process of Programs}


%========================================================
\subsection{Summary}



\newpage
%%%%%%%%%%%%%%%%%%%%%%%%%%%%%%%%%%%%%%%%%%%%%%%%%%%%%%%%%%%%%%%
\section{Elementary C++ Programming}
%========================================================
\subsection{An Overview of C++ Language}

%========================================================
\subsection{Basic Data Types and Expressions}

%========================================================
\subsection{Data Input and Output}

%========================================================
\subsection{The Fundamental Control Structures of Algorithms}

%========================================================
\subsection{User-Defined Data Type}

%========================================================
\subsection{Summary}



\newpage
%%%%%%%%%%%%%%%%%%%%%%%%%%%%%%%%%%%%%%%%%%%%%%%%%%%%%%%%%%%%%%%
\section{Functions}
%========================================================
\subsection{Definition and Use of Function}

%========================================================
\subsection{Inline Functions}

%========================================================
\subsection{Default Formal Parameters in Functions}

%========================================================
\subsection{Function Overloading}

%========================================================
\subsection{Using C++ System Functions}

%========================================================
\subsection{Summary}


\newpage
%%%%%%%%%%%%%%%%%%%%%%%%%%%%%%%%%%%%%%%%%%%%%%%%%%%%%%%%%%%%%%%
\section{Class and Object}
%========================================================
\subsection{Basic Features of Object-Oriented Design}

%========================================================
\subsection{Class and Object}

%========================================================
\subsection{Constructor and Destructor}

%========================================================
\subsection{Combination of Classes}

%========================================================
\subsection{UML}

%========================================================
\subsection{Program Instance – Personnel Information Management Program}

%========================================================
\subsection{Summary}


\newpage
%%%%%%%%%%%%%%%%%%%%%%%%%%%%%%%%%%%%%%%%%%%%%%%%%%%%%%%%%%%%%%%
\section{Data Sharing and Protecting}
%========================================================
\subsection{Scope and Visibility of Identifiers}

%========================================================
\subsection{Lifetime of Object}

%========================================================
\subsection{Static Members of Class}

%========================================================
\subsection{Friend of Class}

%========================================================
\subsection{Protection of Shared Data}

%========================================================
\subsection{Multifile Structure and Compilation Preprocessing Directives}

%========================================================
\subsection{Example – Personnel Information Management Program}

%========================================================
\subsection{Summary}


\newpage
%%%%%%%%%%%%%%%%%%%%%%%%%%%%%%%%%%%%%%%%%%%%%%%%%%%%%%%%%%%%%%%
\section{Arrays, Pointers, and Strings}
%========================================================
\subsection{Arrays}

%========================================================
\subsection{Pointers}

%========================================================
\subsection{Dynamic Memory Allocation}

%========================================================
\subsection{Deep Copy and Shallow Copy}

%========================================================
\subsection{Strings}

%========================================================
\subsection{Program Example – Personnel Information Management Program}

%========================================================
\subsection{Summary}

\newpage
%%%%%%%%%%%%%%%%%%%%%%%%%%%%%%%%%%%%%%%%%%%%%%%%%%%%%%%%%%%%%%%
\section{Inheritance and Derivation}
%========================================================
\subsection{Inheritance and Derivation of Class}

%========================================================
\subsection{Access Control}

%========================================================
\subsection{Type Compatible Rule}

%========================================================
\subsection{Constructor and Destructor of Derived Class}

%========================================================
\subsection{Identification and Access of Derived-Class Member}

%========================================================
\subsection{Program Example: Solving Linear Equations using Gaussian Elimination Method}

%========================================================
\subsection{Program Example: Personnel Information Management Program}

%========================================================
\subsection{Summary}

\newpage
%%%%%%%%%%%%%%%%%%%%%%%%%%%%%%%%%%%%%%%%%%%%%%%%%%%%%%%%%%%%%%%
\section{Polymorphism}
%========================================================
\subsection{An Overview of Polymorphism}

%========================================================
\subsection{Operator Overload}

%========================================================
\subsection{Virtual Function}

%========================================================
\subsection{Abstract Classes}

%========================================================
\subsection{Program Instance: Variable Stepwise Trapezoid Method to Calculate Functional Definite Integral}

%========================================================
\subsection{Program Instance: Improvement on Staff Information Management System for a Small Corporation}

%========================================================
\subsection{Summary}

\newpage
%%%%%%%%%%%%%%%%%%%%%%%%%%%%%%%%%%%%%%%%%%%%%%%%%%%%%%%%%%%%%%%
\section{Collections and Their Organization}
%========================================================
\subsection{Function Templates and Class Templates}

%========================================================
\subsection{Linear Collection}

%========================================================
\subsection{Organizing Data in Linear Collections}

%========================================================
\subsection{Application – Improving the HR Management Program of a Small Company}

%========================================================
\subsection{Summary}


\newpage
%%%%%%%%%%%%%%%%%%%%%%%%%%%%%%%%%%%%%%%%%%%%%%%%%%%%%%%%%%%%%%%
\section{Generic Programming and the Standard Template Library}
%========================================================
\subsection{Generic Programming}

%========================================================
\subsection{Containers in STL}

%========================================================
\subsection{Iterators}

%========================================================
\subsection{Algorithms in STL}

%========================================================
\subsection{Function Objects}

%========================================================
\subsection{Application – Improving the HR Management Program of a Small Company}

%========================================================
\subsection{Summary}

\newpage
%%%%%%%%%%%%%%%%%%%%%%%%%%%%%%%%%%%%%%%%%%%%%%%%%%%%%%%%%%%%%%%
\section{The I/O Stream Library and Input/Output}
%========================================================
\subsection{I/O Stream’s Concept and the Structure of a Stream Library}

%========================================================
\subsection{Output Stream}

%========================================================
\subsection{Input Stream}

%========================================================
\subsection{Input/output Stream}

%========================================================
\subsection{Example-improve Employee Information Management System}

%========================================================
\subsection{Summary}


\newpage
%%%%%%%%%%%%%%%%%%%%%%%%%%%%%%%%%%%%%%%%%%%%%%%%%%%%%%%%%%%%%%%
\section{Exception Handling}
%========================================================
\subsection{Basic Concepts of Exception Handling}

%========================================================
\subsection{The Implementation of Exception Handling in C++}

%========================================================
\subsection{Destruction and Construction in Exception Handling}

%========================================================
\subsection{Exception Handling of Standard Library}

%========================================================
\subsection{Program Example Improvement to Personal Information Administration Program in a Small Company}

%========================================================
\subsection{Summary}


\end{document}

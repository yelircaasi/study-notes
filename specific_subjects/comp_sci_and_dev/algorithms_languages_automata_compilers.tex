\documentclass[a4paper]{article}
\usepackage[utf8]{inputenc}
\usepackage{geometry}
\usepackage[pagebackref=true]{hyperref}
\usepackage{lmodern}
\usepackage{amsmath}
\usepackage{amssymb}
\usepackage{pifont}
\usepackage{bbm}
\usepackage{stmaryrd}
\usepackage{mathtools}

\newcommand{\Ubr}[2]{\underbrace{ #1 }_{\mathclap{\text{ #2 }}}}
\newcommand{\Br}[1]{\{ #1 \}}

\geometry{margin=2cm}

\title{Algorithms, Languages, Automata and Compilers - Study Notes}
\author{Isaac Riley}
\date{Last edited: \today}

\begin{document}
\maketitle
\tableofcontents
\newpage



%%%%%%%%%%%%%%%%%%%%%%%%%%%%%%%%%%%%%%%%%%%%%%%%%%%%%%%%%%%%%%%

%========================================================

%--------------------------------------------------


\section{Regular Languages and Regular Expressions}
\subsection{Basic Definitions}
\subsection{Regular Expressions in Theory}
\subsection{Regular Expressions in Practice}
\subsection{Regular Expressions in Software Products}
\subsection{Conclusions}

\newpage
\section{Finite Automata}
\subsection{Deterministic Finite Automata}
\subsection{Nondeterministic Finite Automata}
\subsection{The JFLAP Project and Finite Automata}
\subsection{Conclusions}

\newpage
\section{The Relationship Between Finite Automata and Regular Expressions}
\subsection{Conversion of a Regular Expression to a Finite Automaton}
\subsection{Conversion of a Finite Automaton to a Regular Expression}
\subsection{Searching Substrings Satisfying the Given Regular Expression}
\subsection{Conversion Functions in JFLAP 87 3.5 Conclusions}

\newpage
\section{Finite-State Machines in Practice}
\subsection{Simple Finite-State Models}
\subsection{About FSM-Based Programming}
\subsection{Conclusions}

\newpage
\section{Nonregular Languages and Context-Free Grammars}
\subsection{Nonregular Languages: The Pumping Lemma}
\subsection{Languages and Problems, Models of Computation}
\subsection{Context-FreeGrammars}
\subsection{RegularGrammars}
\subsection{Conclusions}

\newpage
\section{Pushdown Automata}
\subsection{Organization of a Pushdown Automaton}
\subsection{Conversion of a Context-Free Grammar to a Pushdown Automaton}
\subsection{Conversion of a Pushdown Automaton to a Context-Free Grammar}
\subsection{Deterministic and Nondeterministic Pushdown Automata: Two Big Differences}
\subsection{Pushdown Automata in JFLAP}
\subsection{Recognition of Deterministic Context-Free Languages}
\subsection{Conclusions}

\newpage
\section{Parsing}
\subsection{Unambiguous and Ambiguous Grammars}
\subsection{Leftmost Derivation, Rightmost Derivation}
\subsection{LL, LR, and Other Technical Details}
\subsection{A Parser for LR(1) Grammars}
\subsection{LR(1) Parser and Pushdown Automaton}
\subsection{Parser for LL(1) Grammars}
\subsection{Parser for Any Context-Free Grammar}
\subsection{Conclusions}

\newpage
\section{Compiler Generation}
\subsection{Translators,Compilers,Interpreters}
\subsection{TheCoco/RProject}
\subsection{IdeologyofCompilers}
\subsection{A Practical Example: A Translator for an Elementary Programming Language}
\subsection{Conclusions}

\newpage
\section{The Lindenmayer Systems (L-Systems)}
\subsection{Grammars as a Method of String Generation}
\subsection{GraphicalStringInterpretation}
\subsection{InnerL-SystemOrganization}
\subsection{L-SystemVisualizationInstruments}
\subsection{FractalPatterns}
\subsection{Varieties and Additional Capabilities of L-Systems}
\subsection{Conclusions}

\newpage
\section{Turing Machines}
\subsection{LookingBack}
\subsection{Beyond Context-Free Languages}
\subsection{Deterministic Turing Machine}
\subsection{The Turing Machine and Language Recognition}
\subsection{Formal Definition of the Turing Machine}
\subsection{Turing Machine Emulator}
\subsection{Programming a Turing Machine}
\subsection{Nondeterministic Turing Machine}
\subsection{Turing Machine Variations}
\subsection{Emulation of the Turing Machine with JFLAP}
\subsection{Encoding Machines and Universal Turing Machines}
\subsection{Conclusions}

\newpage
\section{Decidability and Complexity}
\subsection{Decidability and Undecidability of Languages}
\subsection{The Halting Problem}
\subsection{The Turing Machine and Decidability}
\subsection{What Is an Algorithm?}
\subsection{The Turing Machine and a Personal Computer}
\subsection{The Halting Problem and Programmers}
\subsection{Church Formalism and Functional Programming}
\subsection{Complexity of Problems and Systems}
\subsection{Conclusions}






\end{document}

\documentclass[a4paper]{article}
\usepackage[utf8]{inputenc}
\usepackage[ngerman]{babel}
\usepackage{geometry}
\usepackage[pagebackref=true]{hyperref}
\usepackage{lmodern}
\usepackage{amsmath}
\usepackage{amssymb}
\usepackage{pifont}
\usepackage{bbm}
\usepackage{stmaryrd}
\usepackage{mathtools}

\newcommand{\followup}[1]{\textcolor{red}{ #1 }}
\newcommand{\Ubr}[2]{\underbrace{ #1 }_{\mathclap{\text{ #2 }}}}
\newcommand{\Br}[1]{\{ #1 \}}

\geometry{margin=2cm}

\title{Mathematik (Arens u.a.) - Notizen}
\author{Isaac Riley}
%\date{August 2020}

\begin{document}
\maketitle
\tableofcontents
\newpage

\section*{Part I: }

%%%%%%%%%%%%%%%%%%%%%%%%%%%%%%%%%%%%%%%%%%%%%%%%%%%%%%%%%%%%%%%
\section{Mathematik - Wissenschaft und Werkzeug}

%--------------------------------------------------
\subsection{}

%--------------------------------------------------
\subsubsection{}


%%%%%%%%%%%%%%%%%%%%%%%%%%%%%%%%%%%%%%%%%%%%%%%%%%%%%%%%%%%%%%%
\section{Logik, Mengen, Abbildungen - die Sprache der Mathematik}

%%%%%%%%%%%%%%%%%%%%%%%%%%%%%%%%%%%%%%%%%%%%%%%%%%%%%%%%%%%%%%%
\section{Rechentechniken - die Werkzeuge der Mathematik}

%%%%%%%%%%%%%%%%%%%%%%%%%%%%%%%%%%%%%%%%%%%%%%%%%%%%%%%%%%%%%%%
\section{Elementare Funktionen - Bausteine der Analysis}

%%%%%%%%%%%%%%%%%%%%%%%%%%%%%%%%%%%%%%%%%%%%%%%%%%%%%%%%%%%%%%%
\section{Komplexe Zahlen - Rechnen mit imaginären Größen}

%%%%%%%%%%%%%%%%%%%%%%%%%%%%%%%%%%%%%%%%%%%%%%%%%%%%%%%%%%%%%%%
\section{Folgen - der Weg ins Unendliche}

%%%%%%%%%%%%%%%%%%%%%%%%%%%%%%%%%%%%%%%%%%%%%%%%%%%%%%%%%%%%%%%
\section{Stetige Funktionen - kleine Ursachen haben kleine Wirkungen}

%%%%%%%%%%%%%%%%%%%%%%%%%%%%%%%%%%%%%%%%%%%%%%%%%%%%%%%%%%%%%%%
\section{Reihen - Summieren bis zum Letzten}

%%%%%%%%%%%%%%%%%%%%%%%%%%%%%%%%%%%%%%%%%%%%%%%%%%%%%%%%%%%%%%%
\section{Potenzreihen - Alleskönner unter den Funktionen}

%%%%%%%%%%%%%%%%%%%%%%%%%%%%%%%%%%%%%%%%%%%%%%%%%%%%%%%%%%%%%%%
\section{Differenzialrechnung - Veränderungen kalkulieren}

%%%%%%%%%%%%%%%%%%%%%%%%%%%%%%%%%%%%%%%%%%%%%%%%%%%%%%%%%%%%%%%
\section{Integrale - vom Sammeln und Bilanzieren}

%%%%%%%%%%%%%%%%%%%%%%%%%%%%%%%%%%%%%%%%%%%%%%%%%%%%%%%%%%%%%%%
\section{Integrationstechniken -Tipps, Tricks und Näherungsverfahren}

%%%%%%%%%%%%%%%%%%%%%%%%%%%%%%%%%%%%%%%%%%%%%%%%%%%%%%%%%%%%%%%
\section{Differenzialgleichungen - Zusammenspiel von Funktionen und ihren Ableitungen}

%%%%%%%%%%%%%%%%%%%%%%%%%%%%%%%%%%%%%%%%%%%%%%%%%%%%%%%%%%%%%%%
\section{Lineare Gleichungssysteme - Grundlage der linearen Algebra}

%%%%%%%%%%%%%%%%%%%%%%%%%%%%%%%%%%%%%%%%%%%%%%%%%%%%%%%%%%%%%%%
\section{Vektorräume - Schauplätze der linearen Algebra}

%%%%%%%%%%%%%%%%%%%%%%%%%%%%%%%%%%%%%%%%%%%%%%%%%%%%%%%%%%%%%%%
\section{Matrizen und Determinanten - Zahlen in Reihen und Spalten}

%%%%%%%%%%%%%%%%%%%%%%%%%%%%%%%%%%%%%%%%%%%%%%%%%%%%%%%%%%%%%%%
\section{Lineare Abbildungen und Matrizen- abstrakte Sachverhalte in Zahlen augedrückt}

%%%%%%%%%%%%%%%%%%%%%%%%%%%%%%%%%%%%%%%%%%%%%%%%%%%%%%%%%%%%%%%
\section{Eignwerte und Eigenvektoren - oder wie man Matrizen diagonalisiert}

%%%%%%%%%%%%%%%%%%%%%%%%%%%%%%%%%%%%%%%%%%%%%%%%%%%%%%%%%%%%%%%
\section{Analytische Geometrie - Rechnen statt Zeichnen}

%%%%%%%%%%%%%%%%%%%%%%%%%%%%%%%%%%%%%%%%%%%%%%%%%%%%%%%%%%%%%%%
\section{Euklidische und unitäre Vektorräume}

%%%%%%%%%%%%%%%%%%%%%%%%%%%%%%%%%%%%%%%%%%%%%%%%%%%%%%%%%%%%%%%
\section{Quadriken - ebenso nützlich wie dekorativ}

%%%%%%%%%%%%%%%%%%%%%%%%%%%%%%%%%%%%%%%%%%%%%%%%%%%%%%%%%%%%%%%
\section{Tensoren - geschicktes Hantieren mit Indizes}

%%%%%%%%%%%%%%%%%%%%%%%%%%%%%%%%%%%%%%%%%%%%%%%%%%%%%%%%%%%%%%%
\section{Lineare Optimierung - ideale Ausnutzung von Kapazitäten}

%%%%%%%%%%%%%%%%%%%%%%%%%%%%%%%%%%%%%%%%%%%%%%%%%%%%%%%%%%%%%%%
\section{Funktionen mehrerer Variablen - Differenzieren im Raum}

%%%%%%%%%%%%%%%%%%%%%%%%%%%%%%%%%%%%%%%%%%%%%%%%%%%%%%%%%%%%%%%
\section{Gebietsintegrale - das Ausmessen von Körpern}

%%%%%%%%%%%%%%%%%%%%%%%%%%%%%%%%%%%%%%%%%%%%%%%%%%%%%%%%%%%%%%%
\section{Kurven und Flächen - von Krümmung, Torsion und Längnmessung}

%%%%%%%%%%%%%%%%%%%%%%%%%%%%%%%%%%%%%%%%%%%%%%%%%%%%%%%%%%%%%%%
\section{Vektoranalysis - von Quellen und Wirbeln}

%%%%%%%%%%%%%%%%%%%%%%%%%%%%%%%%%%%%%%%%%%%%%%%%%%%%%%%%%%%%%%%
\section{Differenzialgleichungssysteme - ein allgemeiner Zugang zu Differenzialgleichungen}

%%%%%%%%%%%%%%%%%%%%%%%%%%%%%%%%%%%%%%%%%%%%%%%%%%%%%%%%%%%%%%%
\section{Partielle Differenzialgleichungen - Modelle von Feldern und Wellen}

%%%%%%%%%%%%%%%%%%%%%%%%%%%%%%%%%%%%%%%%%%%%%%%%%%%%%%%%%%%%%%%
\newpage 
\section{Fouriertheorie - von schwingenden Saiten}
\subsection{Trigonometrische Polynome}


\subsection{Approximation im quadratischen Mittel}


\subsection{Fourierreihen}


\subsection{Die diskrete Fouriertransformation}



%%%%%%%%%%%%%%%%%%%%%%%%%%%%%%%%%%%%%%%%%%%%%%%%%%%%%%%%%%%%%%%
\newpage
\section{Funktionenanalysis - Operatoren wirken auf Funktionen}
\subsection{Normierte Räume, Banachräume, Hiberträume}


\subsection{Lineare, beschränkte Operatoren und Funktionale}


\subsection{Funktionale und Distributionen}


\subsection{Operatoren in Hilberträumen}


\subsection{Approximation von Operatoren}



%%%%%%%%%%%%%%%%%%%%%%%%%%%%%%%%%%%%%%%%%%%%%%%%%%%%%%%%%%%%%%%
\newpage
\section{Funktionentheorie - von komplexen Zusammenhängen}
\subsection{Komplexe Funktionen und Differenzierbarkeit}


\subsection{Komplexe Kurvenintegrale}


\subsection{Laurent-Reihen und Residuensatz}



%%%%%%%%%%%%%%%%%%%%%%%%%%%%%%%%%%%%%%%%%%%%%%%%%%%%%%%%%%%%%%%
\newpage
\section{Integraltransformationen - Multiplizieren statt Differenzieren}
\subsection{Transformation von Funktionen}


\subsection{Die Laplacetransformation}


\subsection{Die Fouriertransformation}
Fouriertransform:
$$\mathcal{F}(x)(s) = \int_{-\infty}^{\infty}e^{-ist}x(t) dt$$
für $s \in \mathbb{R}$.

Reminder:
$$c_k = \frac{1}{2\pi} \int_{-\pi}^{\pi} x(t) e^{-ikt} dt, k \in \mathbb{Z}.$$

Vergleich: Fourierreihen nur für periodische Funktionen geeignet

%%%%%%%%%%%%%%%%%%%%%%%%%%%%%%%%%%%%%%%%%%%%%%%%%%%%%%%%%%%%%%%
\newpage
\section{Spezielle Funktionen - nützliche Helfer}

%%%%%%%%%%%%%%%%%%%%%%%%%%%%%%%%%%%%%%%%%%%%%%%%%%%%%%%%%%%%%%%
\section{Optimierung und Variationsrechnung - Suche nach dem Besten}

%%%%%%%%%%%%%%%%%%%%%%%%%%%%%%%%%%%%%%%%%%%%%%%%%%%%%%%%%%%%%%%
\section{Deskriptive Statistik - wie man Daten beschreibt}

%%%%%%%%%%%%%%%%%%%%%%%%%%%%%%%%%%%%%%%%%%%%%%%%%%%%%%%%%%%%%%%
\section{Wahrscheinlichkeit - die Gesetze des Zufalls}

%%%%%%%%%%%%%%%%%%%%%%%%%%%%%%%%%%%%%%%%%%%%%%%%%%%%%%%%%%%%%%%
\section{Zufällige Variable - der Zufall betritt den $\mathbb{R}^1$}

%%%%%%%%%%%%%%%%%%%%%%%%%%%%%%%%%%%%%%%%%%%%%%%%%%%%%%%%%%%%%%%
\section{Spezielle Verteilungen - Modelle des Zufalls}

%%%%%%%%%%%%%%%%%%%%%%%%%%%%%%%%%%%%%%%%%%%%%%%%%%%%%%%%%%%%%%%
\section{Schätz- und Testtheorie - Bewerten und Entscheiden}

%%%%%%%%%%%%%%%%%%%%%%%%%%%%%%%%%%%%%%%%%%%%%%%%%%%%%%%%%%%%%%%
\section{Lineare Regression - die Suche nach Abhängigkeiten}



\end{document}

\documentclass[a4paper]{article}
\usepackage[utf8]{inputenc}
\usepackage{geometry}
\usepackage[pagebackref=true]{hyperref}
\usepackage{lmodern}
\usepackage{amsmath}
\usepackage{amssymb}
\usepackage{pifont}
\usepackage{bbm}
\usepackage{stmaryrd}
\usepackage{mathtools}

\newcommand{\followup}[1]{\textcolor{red}{ #1 }}
\newcommand{\Ubr}[2]{\underbrace{ #1 }_{\mathclap{\text{ #2 }}}}
\newcommand{\Br}[1]{\{ #1 \}}

\geometry{margin=2cm}

\title{Transition to Advanced Mathematics - Study Notes}
\author{Isaac Riley}
\date{Last edited: \today}

\begin{document}
\maketitle
\tableofcontents
\newpage



%%%%%%%%%%%%%%%%%%%%%%%%%%%%%%%%%%%%%%%%%%%%%%%%%%%%%%%%%%%%%%%
%========================================================
%--------------------------------------------------


\section{Logic and Proofs}   %p 
\subsection{Propositions and Connectives}   %p %%%%%%%%%%%%%%%%%%%%%%
\subsubsection*{proposition}
\subsubsection*{paradox}
\subsubsection*{simple/atomic propositions}
\subsubsection*{conjunction}
\subsubsection*{disjunction}
\subsubsection*{well-formed formulas}
\subsubsection*{propositional form}
\subsubsection*{equivalent proopositions}
\subsubsection*{denial}
\subsubsection*{tautology}
\subsubsection*{Law of Excluded Middle}
\subsubsection*{contradiction}
\subsubsection*{Theorem 1.1}

\newpage
\subsection{Conditionals and Biconditionals}   %p %%%%%%%%%%%%%%%%%%%%%%
\subsubsection*{conditional sentence}
\subsubsection*{antecedent}
\subsubsection*{consequent}
\subsubsection*{modus ponens}
\subsubsection*{converse}
\subsubsection*{contrapositive}
\subsubsection*{biconditional sentence}
\subsubsection*{De Morgan's Laws}
\subsubsection*{Theorem 1.2}

\newpage
\subsection{Quantifiers}   %p %%%%%%%%%%%%%%%%%%%%%%%%%%%%%%%%
\subsubsection*{inverse/opposite}
\subsubsection*{open sentence}
\subsubsection*{predicate}
\subsubsection*{equivalence of open sentences}
\subsubsection*{universal quantifier}
\subsubsection*{existential quantifier}
\subsubsection*{unique existence quantifier}

\newpage
\subsection{Mathematical Proofs}   %p %%%%%%%%%%%%%%%%%%%%%%%%%%%
\subsubsection*{theorem}
\subsubsection*{proof}
\subsubsection*{axioms}
\subsubsection*{postulates}
\subsubsection*{undefined terms}
\subsubsection*{modus ponens rule}
\subsubsection*{direct proof of p => q}
\subsubsection*{proof by contraposition of P => Q}
\subsubsection*{proof of P by contradiction}
\subsubsection*{two-part proof of P <=> Q}
\subsubsection*{Law of Cosines}
\subsubsection*{consistent axiom systems}
\subsubsection*{undecidable}
\subsubsection*{Theorem 1.3}
\subsubsection*{odd/even example}
\subsubsection*{distance proof example}
\subsubsection*{divisibility example}
\subsubsection*{proof that sqrt(2) is irrational}
\subsubsection*{proof that the number of primes is infinite}
\subsubsection*{proof example using Law of Cosines}
\subsubsection*{absolute value example proof}
\subsubsection*{parity proof 3 examples}

\newpage
\subsection{Proofs Involving Quantifiers}   %p %%%%%%%%%%%%%%%%%%%%%%%
\subsubsection*{existence theorems}
\subsubsection*{constructive proof}
\subsubsection*{proof of existence by contradiction}
\subsubsection*{proof of existence conditional}
\subsubsection*{direct proof of universal statement}
\subsubsection*{proof of universal statement by contradiction}
\subsubsection*{proof of unique existential quantifier}
\subsubsection*{polynomial proof example}
\subsubsection*{Fundamental Theorem of Algebra}
\subsubsection*{Complex Root Theorem}
\subsubsection*{parity example with squares}
\subsubsection*{polynomial example proof}
\subsubsection*{multiplicative inverse proof}
\subsubsection*{ratio inequality example proof}
\subsection{Additional Examples of Proofs}   %p %%%%%%%%%%%%%%%%%%%%%%
\subsubsection*{6 qualifier manipulations}
\subsubsection*{4 common quantifier errors}
\subsubsection*{evenness of $m^2-m$}
\subsubsection*{proof that perp. slopes have -1 product}
\subsubsection*{example}
\subsubsection*{example}
\subsubsection*{Rolle's Theorem}

%%%%%%%%%%%%%%%%%%%%%%%%%%%%%%%%%%%%%%%%%%%%%%%%%%%%%%%%%
\newpage
\section{Set Theory}   %p 
\subsection{Basic Notions of Set Theory}   %p %%%%%%%%%%%%%%%%%%%%%%%
\subsubsection*{set}
\subsubsection*{elements}
\subsubsection*{closed interval}
\subsubsection*{open interval}
\subsubsection*{half-open / half-closed}
\subsubsection*{open ray}
\subsubsection*{closed ray}
\subsubsection*{empty set}
\subsubsection*{subset}
\subsubsection*{direct proof of subset relationship}
\subsubsection*{improper subset}
\subsubsection*{proper subset}
\subsubsection*{power set}
\subsubsection*{set equality}
\subsubsection*{vacuous}
\subsubsection*{trivial}
\subsubsection*{Theorem 2.1: proof that the empty set is a subset of every set}
\subsubsection*{Theorem 2.2}
\subsubsection*{Theorem 2.3}
\subsubsection*{Theorem 2.4: power set cardinality}
\subsubsection*{Theorem 2.5}
\subsubsection*{Theorem 2.6}

\newpage
\subsection{Set Operations}   %p %%%%%%%%%%%%%%%%%%%%%%%%%%%%%%
\subsubsection*{Theorem 2.7}
\subsubsection*{Theorem 2.8}
\subsubsection*{union}
\subsubsection*{intersection}
\subsubsection*{difference}
\subsubsection*{set complement}

\newpage
\subsection{Extended Set Operations and Indexed Families of Sets}   %p %%%%%%%%%%
\subsubsection*{union over A}
\subsubsection*{intersection over A}
\subsubsection*{indexed family of sets}
\subsubsection*{index (set theory)}
\subsubsection*{indexed set}
\subsubsection*{indexed set example}
\subsubsection*{union and intersection over index family of sets}
\subsubsection*{pairwise disjoint}
\subsubsection*{Theorem 2.9}
\subsubsection*{Theorem 2.10}

\newpage
\subsection{Induction}   %p %%%%%%%%%%%%%%%%%%%%%%%%%%%%%%%%%
\subsubsection*{7 properties of N}
\subsubsection*{PMI}
\subsubsection*{inductive set}
\subsubsection*{inductive definition of factorial}
\subsubsection*{complete (strong) induction}
\subsubsection*{example of inductive set definition}
\subsubsection*{example with proof}
\subsubsection*{example with proof}
\subsubsection*{example with proof}
\subsubsection*{example with proof}
\subsubsection*{example with proof}
\subsubsection*{Theorem 2.11}
\subsubsection*{Theorem 2.12: PCI}
\subsubsection*{example with proof}
\subsubsection*{Theorem 2.13: Well-Ordering Principle}
\subsubsection*{Theorem 2.14: Division Algorithm for N}

%\subsection{Equivalent Forms of Induction}   %p 
\newpage
\subsection{Principles of Counting}   %p %%%%%%%%%%%%%%%%%%%%%%%%%%
\subsubsection*{finite set}
\subsubsection*{cardinality of A (incl. notation)}
\subsubsection*{principle of inclusion and exclusion}
\subsubsection*{permutation of a set}
\subsubsection*{combination of elements taken r at a time}
\subsubsection*{binomial coefficient}
\subsubsection*{Construction of Pascal's Triangle}
\subsubsection*{Theorem 2.15: The Sum Rule}
\subsubsection*{Theorem 2.16}
\subsubsection*{Theorem 2.17}
\subsubsection*{Theorem 2.18: The Product Rule}
\subsubsection*{Theorem 2.19}
\subsubsection*{Theorem 2.20}
\subsubsection*{Corollary 2.21}
\subsubsection*{Theorem 2.22}
\subsubsection*{Theorem 2.23}

%%%%%%%%%%%%%%%%%%%%%%%%%%%%%%%%%%%%%%%%%%%%%%%%%%%%%%%%%
\newpage
\section{Relations}   %p 
\subsection{Cartesian Products and Relations}   %p %%%%%%%%%%%%%%%%%%%%
\subsubsection*{ordered pair}
\subsubsection*{coordinate}
\subsubsection*{property of ordered pairs}
\subsubsection*{orderen n-tuple}
\subsubsection*{ordered triple}
\subsubsection*{Cartesian product}
\subsubsection*{relation}
\subsubsection*{relation on A}
\subsubsection*{domain of relation}
\subsubsection*{range of relation}
\subsubsection*{graph of relation}
\subsubsection*{identity relation}
\subsubsection*{directed graph (digraph)}
\subsubsection*{inverse of R}
\subsubsection*{composite of R and S}
\subsubsection*{Theorem 3.1}
\subsubsection*{Theorem 3.2}
\subsubsection*{Theorem 3.3}

\newpage
\subsection{Equivalence Relations}   %p  %%%%%%%%%%%%%%%%%%%%%%%%%%
\subsubsection*{reflexive relation}
\subsubsection*{symmetric relation}
\subsubsection*{transitive relation}
\subsubsection*{loop (in graph)}
\subsubsection*{equivalence relation on A}
\subsubsection*{equivalence class of A}
\subsubsection*{A modulo R}
\subsubsection*{parity}
\subsubsection*{subdigraph}
\subsubsection*{complete digraph}
\subsubsection*{mod}
\subsubsection*{Theorem 3.4}

\newpage
\subsection{Partitions}   %p %%%%%%%%%%%%%%%%%%%%%%%%%%%%%%%%%
\subsubsection*{partition of set}
\subsubsection*{Theorem 3.5}
\subsubsection*{Theorem 3.6}
%\subsection{Ordering Relations}   %p 

\newpage
\subsection{Graphs of Relations}   %p %%%%%%%%%%%%%%%%%%%%%%%%%%%
\subsubsection*{graph}
\subsubsection*{vertices}
\subsubsection*{edges}
\subsubsection*{isomorphic graphs}
\subsubsection*{multigraph}
\subsubsection*{simple graph}
\subsubsection*{adjacent vertices}
\subsubsection*{degree of vertex}
\subsubsection*{size of graph}
\subsubsection*{order or graph}
\subsubsection*{null graph}
\subsubsection*{isolated vertex}
\subsubsection*{complete graph}
\subsubsection*{subgraph}
\subsubsection*{walk}
\subsubsection*{traverse}
\subsubsection*{initial vertex}
\subsubsection*{terminal vertex}
\subsubsection*{length of walk}
\subsubsection*{path}
\subsubsection*{closed walk}
\subsubsection*{closed path}
\subsubsection*{cycle}
\subsubsection*{reachable/accessible vertex}
\subsubsection*{distance between vertices}
\subsubsection*{connected graph}
\subsubsection*{disconnected graph}
\subsubsection*{component of graph}
\subsubsection*{Theorem 3.7}
\subsubsection*{Theorem 3.8}
\subsubsection*{Theorem 3.9}



%%%%%%%%%%%%%%%%%%%%%%%%%%%%%%%%%%%%%%%%%%%%%%%%%%%%%%%%%
\newpage
\section{Functions}   %p 
\subsection{Functions as Relations}   %p %%%%%%%%%%%%%%%%%%%%%%%%%%
\subsubsection*{function}
\subsubsection*{mapping}
\subsubsection*{range}
\subsubsection*{codomain}
\subsubsection*{value of f at x}
\subsubsection*{image of x under f}
\subsubsection*{pre-image of y under f}
\subsubsection*{arguments of function}
\subsubsection*{characteristic function of A}
\subsubsection*{step function}
\subsubsection*{canonical map}
\subsubsection*{identity function on A}
\subsubsection*{greatest integer function}
\subsubsection*{inclusion map}
\subsubsection*{projection functions}
\subsubsection*{Theorem 4.1}
\subsubsection*{Theorem 4.2}
\subsubsection*{Theorem 4.3}
\subsubsection*{Theorem 4.4}
\subsubsection*{Theorem 4.5}

\newpage
\subsection{Constructions of Functions}   %p %%%%%%%%%%%%%%%%%%%%%%%%
\subsubsection*{inverse of function}
\subsubsection*{restriction of f to D}
\subsubsection*{extension of function}
\subsubsection*{Theorem 4.6}

\newpage
\subsection{Functions That Are Onto; One-to-One Functions}   %p %%%%%%%%%%%%%
\subsubsection*{onto}
\subsubsection*{surjection}
\subsubsection*{one-to-one}
\subsubsection*{injection}
\subsubsection*{ono-to-one correspondence}
\subsubsection*{bijection}
\subsubsection*{Theorem 4.7}
\subsubsection*{Theorem 4.8}
\subsubsection*{Theorem 4.9}
\subsubsection*{Corollary 4.10}
\subsubsection*{Theorem 4.11}
\subsubsection*{Theorem 4.12}
\subsubsection*{Theorem 4.13}
\subsubsection*{Theorem 4.14}
\subsubsection*{Theorem 4.15}

\newpage
\subsection{Induced Set Functions}   %p %%%%%%%%%%%%%%%%%%%%%%%%%%
\subsubsection*{image of X}
\subsubsection*{image set of X}
\subsubsection*{inverse image of Y}
\subsubsection*{Theorem 4.16}
\subsubsection*{Theorem 4.17}
\subsubsection*{Theorem 4.18}

%%%%%%%%%%%%%%%%%%%%%%%%%%%%%%%%%%%%%%%%%%%%%%%%%%%%%%%%%
\newpage
\section{Cardinality}   %p 
\subsection{Equivalent Sets; Finite Sets}   %p %%%%%%%%%%%%%%%%%%%%%%%
\subsubsection*{equivalent}
\subsubsection*{one-to-one correspondence}
\subsubsection*{finite set}
\subsubsection*{cardinal number 0}
\subsubsection*{cardinal number k}
\subsubsection*{infinite}
\subsubsection*{pigeonhole principle}
\subsubsection*{example and proof}
\subsubsection*{Theorem 5.1}
\subsubsection*{Theorem 5.2}
\subsubsection*{Lemma 5.3}
\subsubsection*{Lemma 5.4}
\subsubsection*{Theorem 5.5}
\subsubsection*{Theorem 5.6}
\subsubsection*{Theorem 5.7}
\subsubsection*{Lemma 5.8}
\subsubsection*{Theorem 5.9}
\subsubsection*{Corollary 5.10}

\newpage
\subsection{Infinite Sets}   %p %%%%%%%%%%%%%%%%%%%%%%%%%%%%%%%
\subsubsection*{denumerable set}
\subsubsection*{cardinal number aleph 0}
\subsubsection*{countable}
\subsubsection*{uncountable}
\subsubsection*{normalized form}
\subsubsection*{Infinite Hotel puzzle}
\subsubsection*{Theorem 5.11}
\subsubsection*{Theorem 5.12}
\subsubsection*{Theorem 5.13}
\subsubsection*{Theorem 5.14}
\subsubsection*{Theorem 5.15}
\subsubsection*{Theorem 5.16}
\subsubsection*{Theorem 5.17}
\subsubsection*{Theorem 5.18}
\subsubsection*{Theorem 5.19}
\subsubsection*{Theorem 5.20}

\newpage
\subsection{The Ordering of Cardinal Numbers}   %p %%%%%%%%%%%%%%%%%%%%
\subsubsection*{cardinal number A double-bar}
\subsubsection*{3 definitions}
\subsubsection*{Theorem 5.21}
\subsubsection*{Theorem 5.22}
\subsubsection*{Theorem 5.23: Cantor-Schröder-Bernstein Theorem}
\subsubsection*{Corollary 5.24}

\newpage
\subsection{Comparability of Cardinal Numbers and the Axiom of Choice}   %p %%%%%%%
\subsubsection*{}trichotomy property
\subsubsection*{choice function}
\subsubsection*{Axiom of Choice}
\subsubsection*{continuum hypothesis}
\subsubsection*{Theorem 5.25}
\subsubsection*{Theorem 5.26}
\subsubsection*{Theorem 5.27}

\newpage
\subsection{Countable Sets}   %p %%%%%%%%%%%%%%%%%%%%%%%%%%%%%%
\subsubsection*{Theorem 5.28}
\subsubsection*{Corollary 5.29}
\subsubsection*{Theorem 5.30}
\subsubsection*{Theorem 5.31}
\subsubsection*{Theorem 5.32}
\subsubsection*{Theorem 5.33}
%%%%%%%%%%%%%%%%%%%%%%%%%%%%%%%%%%%%%%%%%%%%%%%%%%%%%%%%%
\newpage
\section{Concepts of Algebra: Groups}   %p 
\subsection{Algebraic Structures}   %p %%%%%%%%%%%%%%%%%%%%%%%%%%%
\subsubsection*{algebraic structure/system}
\subsubsection*{closedness of a set under a binary operation}
\subsubsection*{order of algebraic system}
\subsubsection*{operation table Cayley table}
\subsubsection*{commutativity}
\subsubsection*{associativity}
\subsubsection*{identity element}
\subsubsection*{inverse}
\subsubsection*{main diagonal}
\subsubsection*{operaton preserving mapping}
\subsubsection*{Theorem 6.1}
\subsubsection*{Theorem 6.2}
\subsubsection*{Theorem 6.3}

\newpage
\subsection{Groups}   %p %%%%%%%%%%%%%%%%%%%%%%%%%%%%%%%%%%
\subsubsection*{axiomatic approach}
\subsubsection*{group}
\subsubsection*{abelian group}
\subsubsection*{independent axiom}
\subsubsection*{negative}
\subsubsection*{multiples vs powers}
\subsubsection*{homomorphism}
\subsubsection*{homomorphic image}
\subsubsection*{Theorem 6.4}
\subsubsection*{Theorem 6.5}
\subsubsection*{Theorem 6.6}

\newpage
\subsection{Examples of Groups}   %p %%%%%%%%%%%%%%%%%%%%%%%%%%%
\subsubsection*{permutation as function}
\subsubsection*{symmetric group on n symbols}
\subsubsection*{permutation groups}
\subsubsection*{symmetry}
\subsubsection*{octic group}
\subsubsection*{Lemma 6.7}
\subsubsection*{Theorem 6.8}
\subsubsection*{Theorem 6.9}

\newpage
\subsection{Subgroups}   %p %%%%%%%%%%%%%%%%%%%%%%%%%%%%%%%%
\subsubsection*{subgroup}
\subsubsection*{identity subgroup}
\subsubsection*{trivial subgroup}
\subsubsection*{proper subgroups}
\subsubsection*{kernel of a homomorphism}
\subsubsection*{cyclic subgroup generated an element of a group}
\subsubsection*{cyclic group}
\subsubsection*{generator for a group}
\subsubsection*{order of an element in a group}
\subsubsection*{Theorem 6.10}
\subsubsection*{Theorem 6.11}
\subsubsection*{Theorem 6.12}
\subsubsection*{Theorem 6.13}

\newpage
\subsection{Cosets and Lagrange's Theorem}   %p %%%%%%%%%%%%%%%%%%%%%
\subsubsection*{coset}
\subsubsection*{left coset}
\subsubsection*{right coset}
\subsubsection*{Theorem 6.14}
\subsubsection*{Theorem 6.15}
\subsubsection*{Theorem 6.16}
\subsubsection*{Corollary 6.17}
\subsubsection*{Corollary 6.18}
\subsubsection*{Corollary 6.19}

\newpage
\subsection{Quotient Groups}   %p %%%%%%%%%%%%%%%%%%%%%%%%%%%%%
\subsubsection*{normal (of subgroup in group)}
\subsubsection*{quotient group}
\subsubsection*{G modulo H}
\subsubsection*{Theorem 6.20}
\subsubsection*{Theorem 6.21}
\subsection{Isomorphism; The Fundamental Theorem of Group Homomorphisms}   %p %%%
\subsubsection*{isomorphism}
\subsubsection*{isomorphic}
\subsubsection*{Theorem 6.22}
\subsubsection*{Theorem 6.23a}
\subsubsection*{Theorem 6.23b}
\subsubsection*{Theorem 6.24: Fundamental Theorem of Group Homomorphisms}

%%%%%%%%%%%%%%%%%%%%%%%%%%%%%%%%%%%%%%%%%%%%%%%%%%%%%%%%% 7
\newpage
\section{Concepts of Analysis: Completeness of the Real Numbers}   %p 
\subsection{Ordered Field Properties of the Real Numbers}   %p %%%%%%%%%%%%%%
\subsubsection*{field}
\subsubsection*{ordered field}
\subsubsection*{upper bound}
\subsubsection*{bounded above}
\subsubsection*{lower bound}
\subsubsection*{bounded below}
\subsubsection*{bounded}
\subsubsection*{least upper bound}
\subsubsection*{supremum}
\subsubsection*{greatest lower bound}
\subsubsection*{infimum}
\subsubsection*{complete ordered field}
\subsubsection*{Theorem 7.1}

\newpage
\subsection{The Heine-Borel Theorem}   %p %%%%%%%%%%%%%%%%%%%%%%%%
\subsubsection*{delta-neighborhood of a}
\subsubsection*{interior point of A}
\subsubsection*{open set in R}
\subsubsection*{closed set in R}
\subsubsection*{Theorem 7.2}
\subsubsection*{Theorem 7.3}
\subsubsection*{Theorem 7.4}
\subsubsection*{Theorem 7.5}
\subsubsection*{cover}
\subsubsection*{subcover}
\subsubsection*{compact subset of R}
\subsubsection*{boundary point}
\subsubsection*{Heine-Borel Theorem}

\newpage
\subsection{The Bolzano-Weierstrass Theorem}   %p %%%%%%%%%%%%%%%%%%%%
\subsubsection*{accumulation point for a set}
\subsubsection*{derived set of A}
\subsubsection*{sequence}
\subsubsection*{real sequence}
\subsubsection*{nth term of a sequence}
\subsubsection*{Theorem 7.7}
\subsubsection*{Theorem 7.8: Bolzano-Weierstrass Theorem}

\newpage
\subsection{The Bounded Monotone Sequence Theorem}   %p %%%%%%%%%%%%%%%
\subsubsection*{bounded sequence}
\subsubsection*{limit}
\subsubsection*{convergence of a sequence}
\subsubsection*{divergence of a sequence}
\subsubsection*{increasing sequence}
\subsubsection*{decreasing sequence}
\subsubsection*{monotone sequence}
\subsubsection*{e}
\subsubsection*{Cauchy sequence}

\newpage
\subsection{Equivalents of Completeness}   %p %%%%%%%%%%%%%%%%%%%%%%%
\subsubsection*{Lemma 7.14}
\subsubsection*{Lemma 7.15}
\subsubsection*{Lemma 7.16}



\end{document}


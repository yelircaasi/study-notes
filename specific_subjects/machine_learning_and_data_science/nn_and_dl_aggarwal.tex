\documentclass[a4paper]{article}
\usepackage[utf8]{inputenc}
\usepackage{geometry}
\usepackage[pagebackref=true]{hyperref}
\usepackage{lmodern}
\usepackage{amsmath}
\usepackage{amssymb}
\usepackage{pifont}
\usepackage{bbm}
\usepackage{stmaryrd}
\usepackage{mathtools}
\usepackage{xcolor}

\newcommand{\followup}[1]{\textcolor{red}{ #1 }}
\newcommand{\Ubr}[2]{\underbrace{ #1 }_{\mathclap{\text{ #2 }}}}
\newcommand{\Br}[1]{\{ #1 \}}

\geometry{margin=2cm}

\title{Neural Networks and Deep Learning (Aggarwal) - Study Notes}
\author{Isaac Riley}
\date{Last edited: \today}

\begin{document}
\maketitle
\tableofcontents
\newpage


\section*{\followup{Pre-Reading General Questions}}
\followup{
\begin{itemize}
    \item What is a Hopfield Network?
    \item What is a Restricted Boltzmann Machine?
    \item What is the general mathematical formulation of convolution?
    \item What is pretraining?
    \item How does backpropogation through time work?
    \item What are the details of the attention mechanism?
\end{itemize}
}

\newpage
%%%%%%%%%%%%%%%%%%%%%%%%%%%%%%%%%%%%%%%%%%%%%%%%%%%%%%%%%%%%%%% 1
\section{An Introduction to Neural Networks}
%======================================================== 
\subsection{Introduction}

%======================================================== 
\subsection{The Basic Architecture of Neural Networks}

%======================================================== 
\subsection{Training A Neural Network with Backpropagation}

%======================================================== 
\subsection{Practical Issues in Neural Network Training}

%======================================================== 
\subsection{The Secrets to the Power of Function Composition}

%======================================================== 
\subsection{Common Neural Architectures}

%======================================================== 
\subsection{Advanced Topics}

%======================================================== 
\subsection{Two Notable Benchmarks}

%======================================================== 
\subsection{Summary}

%======================================================== 
\subsection{Bibliographic Notes}

%======================================================== 
\subsection{Exercises}

\newpage
%%%%%%%%%%%%%%%%%%%%%%%%%%%%%%%%%%%%%%%%%%%%%%%%%%%%%%%%%%%%%%% 2
\section{Machine Learning with Shallow Neural Networks}
%======================================================== 
\subsection{Introduction}

%======================================================== 
\subsection{Neural Architectures for Binary Classification Models}

%======================================================== 
\subsection{Neural Architectures for Multiclass Models}

%======================================================== 
\subsection{Backpropagated Saliency for Feature Selection}

%======================================================== 
\subsection{Matrix Factorization with Autoencoders}

%======================================================== 
\subsection{Other Applications}

%======================================================== 
\subsection{Word2vec: An Application of Simple Neural Architectures}

%======================================================== 
\subsection{Simple Neural Architectures for Graph Embeddings}

%======================================================== 
\subsection{Summary}

%======================================================== 
\subsection{Bibliographic Notes}

%======================================================== 
\subsection{Exercises}

\newpage
%%%%%%%%%%%%%%%%%%%%%%%%%%%%%%%%%%%%%%%%%%%%%%%%%%%%%%%%%%%%%%% 3
\section{Training Deep Neural Networks}
%======================================================== 
\subsection{Introduction}

%======================================================== 
\subsection{Backpropagation: The Gory Details}

%======================================================== 
\subsection{Setup and Initialization Issues}

%======================================================== 
\subsection{The Vanishing and Exploding Gradient Problems}

%======================================================== 
\subsection{Gradient-Descent Strategies}

%======================================================== 
\subsection{Batch Normalization}

%======================================================== 
\subsection{Practical Tricks for Acceleration and Compression}

%======================================================== 
\subsection{Summary}

%======================================================== 
\subsection{Bibliographic Notes}

%======================================================== 
\subsection{Exercises}

\newpage
%%%%%%%%%%%%%%%%%%%%%%%%%%%%%%%%%%%%%%%%%%%%%%%%%%%%%%%%%%%%%%% 4
\section{Teaching Deep Learners to Generalize}
%======================================================== 
\subsection{Introduction}

%======================================================== 
\subsection{The Bias-Variance Trade-Off}

%======================================================== 
\subsection{Generalization Issues in Model Tuning and Evaluation}

%======================================================== 
\subsection{Penalty-Based Regularization}

%======================================================== 
\subsection{Ensemble Methods}

%======================================================== 
\subsection{Early Stopping}

%======================================================== 
\subsection{Unsupervised Pretraining}

%======================================================== 
\subsection{Continuation and Curriculum Learning}

%======================================================== 
\subsection{Parameter Sharing}

%======================================================== 
\subsection{Regularization in Unsupervised Applications}

%======================================================== 
\subsection{Summary}

%======================================================== 
\subsection{Bibliographic Notes}

%======================================================== 
\subsection{Exercises}

\newpage
%%%%%%%%%%%%%%%%%%%%%%%%%%%%%%%%%%%%%%%%%%%%%%%%%%%%%%%%%%%%%%% 5
\section{Radial Basis Function Networks}
%======================================================== 
\subsection{Introduction}

%======================================================== 
\subsection{Training an RBF Network}

%======================================================== 
\subsection{Variations and Special Cases of RBF Networks}

%======================================================== 
\subsection{Relationship with Kernel Methods}

%======================================================== 
\subsection{Summary}

%======================================================== 
\subsection{Bibliographic Notes}

%======================================================== 
\subsection{Exercises}

\newpage
%%%%%%%%%%%%%%%%%%%%%%%%%%%%%%%%%%%%%%%%%%%%%%%%%%%%%%%%%%%%%%% 6
\section{Restricted Boltzmann Machines}
%======================================================== 
\subsection{Introduction}

%======================================================== 
\subsection{Hopfield Networks}

%======================================================== 
\subsection{The Boltzmann Machine}

%======================================================== 
\subsection{Restricted Boltzmann Machines}

%======================================================== 
\subsection{Applications of Restricted Boltzmann Machines}

%======================================================== 
\subsection{Using RBMs beyond Binary Data Types}

%======================================================== 
\subsection{Stacking Restricted Boltzmann Machines}

%======================================================== 
\subsection{Summary}


%======================================================== 
\subsection{Bibliographic Notes}

%======================================================== 
\subsection{Exercises}

\newpage
%%%%%%%%%%%%%%%%%%%%%%%%%%%%%%%%%%%%%%%%%%%%%%%%%%%%%%%%%%%%%%% 7
\section{Recurrent Neural Networks}
%======================================================== 
\subsection{Introduction}

%======================================================== 
\subsection{The Architecture of Recurrent Neural Networks}

%======================================================== 
\subsection{The Challenges of Training Recurrent Networks}

%======================================================== 
\subsection{Echo-State Networks}

%======================================================== 
\subsection{Long Short-Term Memory (LSTM)}

%======================================================== 
\subsection{Gated Recurrent Units (GRUs)}

%======================================================== 
\subsection{Applications of Recurrent Neural Networks}

%======================================================== 
\subsection{Summary}

%======================================================== 
\subsection{Bibliographic Notes}

%======================================================== 
\subsection{Exercises}

\newpage
%%%%%%%%%%%%%%%%%%%%%%%%%%%%%%%%%%%%%%%%%%%%%%%%%%%%%%%%%%%%%%% 8
\section{Convolutional Neural Networks}
%======================================================== 
\subsection{Introduction}

%======================================================== 
\subsection{The Basic Structure of a Convolutional Network}

%======================================================== 
\subsection{Training and Convolutional Network}

%======================================================== 
\subsection{Case Studies of Convolutional Architectures}

%======================================================== 
\subsection{Visualization and Unsupervised Learning}

%======================================================== 
\subsection{Applications of Convolutional Networks}

%======================================================== 
\subsection{Summary}


%======================================================== 
\subsection{Bibliographic Notes}

%======================================================== 
\subsection{Exercises}

\newpage
%%%%%%%%%%%%%%%%%%%%%%%%%%%%%%%%%%%%%%%%%%%%%%%%%%%%%%%%%%%%%%% 9
\section{Deep Reinforcement Learning}
%======================================================== 
\subsection{Introduction}

%======================================================== 
\subsection{Stateless Algorithms: Multi-Armed Bandits}

%======================================================== 
\subsection{The Basic Framework of Reinforcement Learning}

%======================================================== 
\subsection{Bootstrapping for Value Function Learning}

%======================================================== 
\subsection{Policy Gradient Methods}

%======================================================== 
\subsection{Monte Carlo Tree Search}

%======================================================== 
\subsection{Case Studies}

%======================================================== 
\subsection{Practical Challenges Associated with Safety}

%======================================================== 
\subsection{Summary}


%======================================================== 
\subsection{Bibliographic Notes}

%======================================================== 
\subsection{Exercises}

\newpage
%%%%%%%%%%%%%%%%%%%%%%%%%%%%%%%%%%%%%%%%%%%%%%%%%%%%%%%%%%%%%%% 10
\section{Advanced Topics in Deep Learning}
%======================================================== 
\subsection{Introduction}

%======================================================== 
\subsection{Attention Mechanisms}

%======================================================== 
\subsection{Neural Networks with External Memory}

%======================================================== 
\subsection{Generative Adversarial Networks (GANs)}

%======================================================== 
\subsection{Competitive Learning}

%======================================================== 
\subsection{Limitations of Neural Networks}

%======================================================== 
\subsection{Summary}

%======================================================== 
\subsection{Bibliographic Notes}

%======================================================== 
\subsection{Exercises}





%--------------------------------------------------
%\subsubsection{}









\end{document}

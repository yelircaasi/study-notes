\documentclass{article}

\usepackage[margin=0.7in]{geometry}
\usepackage[utf8]{inputenc}
\usepackage{amsmath}
\usepackage{amssymb}
\usepackage[dvipsnames]{xcolor}

\newcommand{\followup}[1]{\textcolor{red}{ #1 }}

\title{Notizen - Maschinelles Lernen (Alpaydin)}
\author{Isaac Riley}
\date{März 2020}

\begin{document}
\maketitle

\section{Einführung} %%%%%%%%%%%%%%%%%%%%%%%%%%%%%%%%%%%%%%%%%%%%%%%%%%%%%%%% 1
  \subsection{Was ist maschinelles Lernen?} %-----------------------------------------------------------------------------------------------------
      \textcolor{blue}{\#\#\#\#\#\#\#\#\#}
            \begin{itemize}
            \color{red}
              \item Was unterscheidet ML von KI? \textcolor{blue}{t}
            \color{ForestGreen}
              \item Definiere maschinelles Lernen. \textcolor{blue}{t}
            \end{itemize}
  \subsection{Beispiele für Anwendungen des ML} %----------------------------------------------------------------------------------------------
      \textcolor{blue}{\#\#\#\#\#\#\#\#\#}
    \subsubsection{Erlernen von Assoziationen} %~~~~~~~~~~~~~~~~~~~~~~~~~~~~~~~~~~~~~~~~~~
      \textcolor{blue}{\#\#\#\#\#\#\#\#\#}

      \begin{itemize}
      \color{red}
        \item  \textcolor{blue}{\_\_\_\_}
        \item  \textcolor{blue}{\_\_\_\_}
      \color{ForestGreen}
        \item  \textcolor{blue}{\_\_\_\_}
        \item  \textcolor{blue}{\_\_\_\_}
        \item  \textcolor{blue}{\_\_\_\_}
      \end{itemize}

    \subsubsection{Klassifikation} %~~~~~~~~~~~~~~~~~~~~~~~~~~~~~~~~~~~~~~~~~~~~~~~~~~~~~
      \textcolor{blue}{\#\#\#\#\#\#\#\#\#}
    \subsubsection{Regression} %~~~~~~~~~~~~~~~~~~~~~~~~~~~~~~~~~~~~~~~~~~~~~~~~~~~~~~
      \textcolor{blue}{\#\#\#\#\#\#\#\#\#}
    \subsubsection{Unüberwachtes Lernen} %~~~~~~~~~~~~~~~~~~~~~~~~~~~~~~~~~~~~~~~~~~~~~
      \textcolor{blue}{\#\#\#\#\#\#\#\#\#}
    \subsubsection{Bestärkendes Lernen} %~~~~~~~~~~~~~~~~~~~~~~~~~~~~~~~~~~~~~~~~~~~~~~~
      \textcolor{blue}{\#\#\#\#\#\#\#\#\#}
  \subsection{Anmerkungen} %--------------------------------------------------------------------------------------------------------------------------
      \textcolor{blue}{\#\#\#\#\#\#\#\#\#}
  \subsection{Relevante Ressourcen} %--------------------------------------------------------------------------------------------------------------
      \textcolor{blue}{\#\#\#\#\#\#\#\#\#}
  \subsection{Übungen} %--------------------------------------------------------------------------------------------------------------------------------
      \textcolor{blue}{\#\#\#\#\#\#\#\#\#}
  \subsection{Literaturangaben} %----------------------------------------------------------------------------------------------------------------------
      \textcolor{blue}{\#\#\#\#\#\#\#\#\#}

      \begin{itemize}
      \color{red}
        \item  \textcolor{blue}{\_\_\_\_}
        \item  \textcolor{blue}{\_\_\_\_}
      \color{ForestGreen}
        \item  \textcolor{blue}{\_\_\_\_}
        \item  \textcolor{blue}{\_\_\_\_}
      \end{itemize}



\newpage
\section{Überwachtes Lernen} %%%%%%%%%%%%%%%%%%%%%%%%%%%%%%%%%%%%%%%%%%%%%%%%%%% 2
  \subsection{Erlernen einer Klasse anhand von Beispielen} %---------------------------------------------------------------------------------
      \textcolor{blue}{\#\#\#\#\#\#\#\#\#}
  \subsection{Vapnik-Chervonenkis(VC)-Dimension} %-------------------------------------------------------------------------------------------
      \textcolor{blue}{\#\#\#\#\#\#\#\#\#}
  \subsection{PAC-Lernen} %----------------------------------------------------------------------------------------------------------------------------
      \textcolor{blue}{\#\#\#\#\#\#\#\#\#}
  \subsection{Rauschen} %-------------------------------------------------------------------------------------------------------------------------------
      \textcolor{blue}{\#\#\#\#\#\#\#\#\#}
  \subsection{Erlernen multipler Klassen} %---------------------------------------------------------------------------------------------------------
      \textcolor{blue}{\#\#\#\#\#\#\#\#\#}
  \subsection{Modellauswahl und Generalisierung} %--------------------------------------------------------------------------------------------
      \textcolor{blue}{\#\#\#\#\#\#\#\#\#}
  \subsection{Dimensionen eine Algorithmus für überwachtes Lernen} %-------------------------------------------------------------------
      \textcolor{blue}{\#\#\#\#\#\#\#\#\#}
  \subsection{Anmerkungen} %-------------------------------------------------------------------------------------------------------------------------
      \textcolor{blue}{\#\#\#\#\#\#\#\#\#}
  \subsection{Literaturangaben} %---------------------------------------------------------------------------------------------------------------------
      \textcolor{blue}{\#\#\#\#\#\#\#\#\#}

      \begin{itemize}
      \color{red}
        \item  \textcolor{blue}{\_\_\_\_}
        \item  \textcolor{blue}{\_\_\_\_}
      \color{ForestGreen}
        \item  \textcolor{blue}{\_\_\_\_}
        \item  \textcolor{blue}{\_\_\_\_}
      \end{itemize}



\newpage
\section{Bayessche Entscheidungstheorie} %%%%%%%%%%%%%%%%%%%%%%%%%%%%%%%%%%%%%%%%%%%%% 3
  \subsection{Einführung} %-----------------------------------------------------------------------------------------------------------------------------
      \textcolor{blue}{\#\#\#\#\#\#\#\#\#}
  \subsection{Klassifikation} %--------------------------------------------------------------------------------------------------------------------------
      \textcolor{blue}{\#\#\#\#\#\#\#\#\#}
  \subsection{Verluste und Risiken} %----------------------------------------------------------------------------------------------------------------
      \textcolor{blue}{\#\#\#\#\#\#\#\#\#}
  \subsection{Diskriminanzfunktionen} %------------------------------------------------------------------------------------------------------------
      \textcolor{blue}{\#\#\#\#\#\#\#\#\#}
  \subsection{Assoziationsregeln} %------------------------------------------------------------------------------------------------------------------
      \textcolor{blue}{\#\#\#\#\#\#\#\#\#}
  \subsection{Anmerkungen} %-------------------------------------------------------------------------------------------------------------------------
      \textcolor{blue}{\#\#\#\#\#\#\#\#\#}
  \subsection{Übungen} %-------------------------------------------------------------------------------------------------------------------------------
      \textcolor{blue}{\#\#\#\#\#\#\#\#\#}
  \subsection{Literaturangeben} %---------------------------------------------------------------------------------------------------------------------
      \textcolor{blue}{\#\#\#\#\#\#\#\#\#}

      \begin{itemize}
      \color{red}
        \item  \textcolor{blue}{\_\_\_\_}
        \item. \textcolor{blue}{\_\_\_\_}
      \color{ForestGreen}
        \item  \textcolor{blue}{\_\_\_\_}
        \item. \textcolor{blue}{\_\_\_\_}
      \end{itemize}



\newpage
\section{Parametrische Methoden} %%%%%%%%%%%%%%%%%%%%%%%%%%%%%%%%%%%%%%%%%%%%%%%%% 4
  \subsection{Einführung} %-----------------------------------------------------------------------------------------------------------------------------
      \textcolor{blue}{\#\#\#\#\#\#\#\#\#}
  \subsection{Maximum-Likelihood-Schätzung} %-------------------------------------------------------------------------------------------------
    \subsubsection{Bernoulli-Verteilung} %~~~~~~~~~~~~~~~~~~~~~~~~~~~~~~~~~~~~~~~~~~~~~~~
      \begin{itemize}
      \color{red}
        \item Was ist die Bernoulli-Verteilung? \textcolor{blue}{\_\_\_\_}
        \item  \textcolor{blue}{\_\_\_\_}
        \item  \textcolor{blue}{\_\_\_\_}
      \end{itemize}
      \textcolor{blue}{\#\#\#\#\#\#\#\#\#}
    \subsubsection{Multinomiale Dichte} %~~~~~~~~~~~~~~~~~~~~~~~~~~~~~~~~~~~~~~~~~~~~~~~
      \textcolor{blue}{\#\#\#\#\#\#\#\#\#}
    \subsubsection{Gauß-Verteilung (Normalverteilung)} %~~~~~~~~~~~~~~~~~~~~~~~~~~~~~~~~~~~
      \textcolor{blue}{\#\#\#\#\#\#\#\#\#}
  \subsection{Bewertung eines Schätzers: Verzerrung und Varianz} %----------------------------------------------------------------------
      \textcolor{blue}{\#\#\#\#\#\#\#\#\#}
  \subsection{Der Bayessche Schätzer} %-----------------------------------------------------------------------------------------------------------
      \textcolor{blue}{\#\#\#\#\#\#\#\#\#}
  \subsection{Parametrische Klassifikation} %------------------------------------------------------------------------------------------------------
      \begin{itemize}
      \color{red}
        \item  \textcolor{blue}{\_\_\_\_}
        \item. \textcolor{blue}{\_\_\_\_}
      \color{ForestGreen}
        \item  \textcolor{blue}{\_\_\_\_}
        \item. \textcolor{blue}{\_\_\_\_}
      \end{itemize}
      \textcolor{blue}{\#\#\#\#\#\#\#\#\#}
  \subsection{Regression} %-----------------------------------------------------------------------------------------------------------------------------
      \textcolor{blue}{\#\#\#\#\#\#\#\#\#}
  \subsection{Das Verzerrung/Varianz-Dilemma} %------------------------------------------------------------------------------------------------

      \textcolor{blue}{\#\#\#\#\#\#\#\#\#}
  \subsection{Modellauswahl} %-------------------------------------------------------------------------------------------------------------------------

      \textcolor{blue}{\#\#\#\#\#\#\#\#\#}
  \subsection{Anmerkungen} %--------------------------------------------------------------------------------------------------------------------------

      \textcolor{blue}{\#\#\#\#\#\#\#\#\#}
  \subsection{Übungen} %---------------------------------------------------------------------------------------------------------------------------------

      \textcolor{blue}{\#\#\#\#\#\#\#\#\#}
  \subsection{Literaturangaben} %----------------------------------------------------------------------------------------------------------------------

      \textcolor{blue}{\#\#\#\#\#\#\#\#\#}

      \begin{itemize}
      \color{red}
        \item  \textcolor{blue}{\_\_\_\_}
        \item  \textcolor{blue}{\_\_\_\_}
      \color{ForestGreen}
        \item  \textcolor{blue}{\_\_\_\_}
        \item  \textcolor{blue}{\_\_\_\_}
      \end{itemize}



\newpage
\section{Multivariate Methoden} %%%%%%%%%%%%%%%%%%%%%%%%%%%%%%%%%%%%%%%%%%%%%%%%%%% 5
      \textcolor{blue}{\#\#\#\#\#\#\#\#\#}
  \subsection{Multivariate Methoden} %--------------------------------------------------------------------------------------------------------------
      \textcolor{blue}{\#\#\#\#\#\#\#\#\#}
  \subsection{Parameterschätzung} %----------------------------------------------------------------------------------------------------------------
      \textcolor{blue}{\#\#\#\#\#\#\#\#\#}
  \subsection{Schätzung von Fehlenden Werten} %-----------------------------------------------------------------------------------------------
      \textcolor{blue}{\#\#\#\#\#\#\#\#\#}
  \subsection{Multivariate Normalverteilung} %-----------------------------------------------------------------------------------------------------
      \textcolor{blue}{\#\#\#\#\#\#\#\#\#}
  \subsection{Multivariate Klassifikation} %----------------------------------------------------------------------------------------------------------
      \textcolor{blue}{\#\#\#\#\#\#\#\#\#}
  \subsection{Anpassen der Komplexität} %---------------------------------------------------------------------------------------------------------
      \textcolor{blue}{\#\#\#\#\#\#\#\#\#}
  \subsection{Diskrete Merkmale} %-------------------------------------------------------------------------------------------------------------------
      \textcolor{blue}{\#\#\#\#\#\#\#\#\#}
  \subsection{Multivariate Regression} %-------------------------------------------------------------------------------------------------------------
      \textcolor{blue}{\#\#\#\#\#\#\#\#\#}
  \subsection{Anmerkungen} %--------------------------------------------------------------------------------------------------------------------------
      \textcolor{blue}{\#\#\#\#\#\#\#\#\#}
  \subsection{Übungen} %--------------------------------------------------------------------------------------------------------------------------------
      \textcolor{blue}{\#\#\#\#\#\#\#\#\#}
  \subsection{Literaturangaben} %---------------------------------------------------------------------------------------------------------------------
      \textcolor{blue}{\#\#\#\#\#\#\#\#\#}

      \begin{itemize}
      \color{red}
        \item
        \item
        \item
      \end{itemize}


      \begin{itemize}
      \color{ForestGreen}
        \item
        \item
        \item
      \end{itemize}



\newpage
\section{Dimensionalitätsreduktion} %%%%%%%%%%%%%%%%%%%%%%%%%%%%%%%%%%%%%%%%%%%%%%%%% 6
  \subsection{Einführung} %-----------------------------------------------------------------------------------------------------------------------------
      \textcolor{blue}{\#\#\#\#\#\#\#\#\#}
  \subsection{Teilmengenselektion} %----------------------------------------------------------------------------------------------------------------
      \textcolor{blue}{\#\#\#\#\#\#\#\#\#}
  \subsection{Hauptkomponentenanalyse} %-------------------------------------------------------------------------------------------------------
      \textcolor{blue}{\#\#\#\#\#\#\#\#\#}
  \subsection{Merkmalseinbettung} %-----------------------------------------------------------------------------------------------------------------
      \textcolor{blue}{\#\#\#\#\#\#\#\#\#}
  \subsection{Faktorenanalyse} %----------------------------------------------------------------------------------------------------------------------
      \textcolor{blue}{\#\#\#\#\#\#\#\#\#}
  \subsection{Singulärwertzerlegung und Zerlegung von Matrizen} %------------------------------------------------------------------------
      \textcolor{blue}{\#\#\#\#\#\#\#\#\#}
  \subsection{Multidimensionale Skalierung} %-----------------------------------------------------------------------------------------------------
      \textcolor{blue}{\#\#\#\#\#\#\#\#\#}
  \subsection{Lineare Diskriminanzanalyse} %------------------------------------------------------------------------------------------------------
      \textcolor{blue}{\#\#\#\#\#\#\#\#\#}
  \subsection{Kanonische Korrelationsanalyse} %-------------------------------------------------------------------------------------------------
      \textcolor{blue}{\#\#\#\#\#\#\#\#\#}
  \subsection{Isomap} %----------------------------------------------------------------------------------------------------------------------------------
      \textcolor{blue}{\#\#\#\#\#\#\#\#\#}
  \subsection{Lokal lineare Einbettung} %------------------------------------------------------------------------------------------------------------
      \textcolor{blue}{\#\#\#\#\#\#\#\#\#}
  \subsection{Laplacesche Eigenmaps} %------------------------------------------------------------------------------------------------------------
      \textcolor{blue}{\#\#\#\#\#\#\#\#\#}
  \subsection{Anmerkungen} %--------------------------------------------------------------------------------------------------------------------------
      \textcolor{blue}{\#\#\#\#\#\#\#\#\#}
  \subsection{Übungen} %--------------------------------------------------------------------------------------------------------------------------------
      \textcolor{blue}{\#\#\#\#\#\#\#\#\#}
  \subsection{Literaturangaben} %----------------------------------------------------------------------------------------------------------------------
      \textcolor{blue}{\#\#\#\#\#\#\#\#\#}

      \begin{itemize}
      \color{red}
        \item
        \item
        \item
      \end{itemize}


      \begin{itemize}
      \color{ForestGreen}
        \item
        \item
        \item
      \end{itemize}



\newpage
\section{Clusteranalyse} %%%%%%%%%%%%%%%%%%%%%%%%%%%%%%%%%%%%%%%%%%%%%%%%%%%%%%% 7
  \subsection{Einführung} %------------------------------------------------------------------------------------------------------------------------------
      \textcolor{blue}{\#\#\#\#\#\#\#\#\#}
  \subsection{Mischungsdichten} %--------------------------------------------------------------------------------------------------------------------
      \textcolor{blue}{\#\#\#\#\#\#\#\#\#}
  \subsection{$k$-Means-Clusteranalyse} %--------------------------------------------------------------------------------------------------------
      \textcolor{blue}{\#\#\#\#\#\#\#\#\#}
  \subsection{Expectation-Maximization-Algorithmus} %-----------------------------------------------------------------------------------------
      \textcolor{blue}{\#\#\#\#\#\#\#\#\#}
  \subsection{Mischungsmodelle mit verborgenen Variablen} %-------------------------------------------------------------------------------
      \textcolor{blue}{\#\#\#\#\#\#\#\#\#}
  \subsection{Überwachtes Lernen nach einer Clusteranalyse} %-----------------------------------------------------------------------------
      \textcolor{blue}{\#\#\#\#\#\#\#\#\#}
  \subsection{Spektrale Clusteranalyse} %----------------------------------------------------------------------------------------------------------
      \textcolor{blue}{\#\#\#\#\#\#\#\#\#}
  \subsection{Hierarchische Clusteranalyse} %-----------------------------------------------------------------------------------------------------
      \textcolor{blue}{\#\#\#\#\#\#\#\#\#}
  \subsection{Auswahl der Anzahl an Clustern} %-------------------------------------------------------------------------------------------------
      \textcolor{blue}{\#\#\#\#\#\#\#\#\#}
  \subsection{Anmerkungen} %-------------------------------------------------------------------------------------------------------------------------
      \textcolor{blue}{\#\#\#\#\#\#\#\#\#}
  \subsection{Übungen} %--------------------------------------------------------------------------------------------------------------------------------
      \textcolor{blue}{\#\#\#\#\#\#\#\#\#}
  \subsection{Literaturangaben} %---------------------------------------------------------------------------------------------------------------------
      \textcolor{blue}{\#\#\#\#\#\#\#\#\#}

      \begin{itemize}
      \color{red}
        \item
        \item
        \item
      \end{itemize}


      \begin{itemize}
      \color{ForestGreen}
        \item
        \item
        \item
      \end{itemize}



\newpage
\section{Nichtparametrische Methoden} %%%%%%%%%%%%%%%%%%%%%%%%%%%%%%%%%%%%%%%%%%%%%%% 8
       \textcolor{blue}{\#\#\#\#\#\#\#\#\#}
  \subsection{Einführung} %-----------------------------------------------------------------------------------------------------------------------------
       \textcolor{blue}{\#\#\#\#\#\#\#\#\#}
  \subsection{Nichtparametrische Dichteschätzung} %-------------------------------------------------------------------------------------------
       \textcolor{blue}{\#\#\#\#\#\#\#\#\#}
    \subsubsection{Histogrammschätzer} %~~~~~~~~~~~~~~~~~~~~~~~~~~~~~~~~~~~~~~~~~~~~~~
       \textcolor{blue}{\#\#\#\#\#\#\#\#\#}
    \subsubsection{Kernel-Schätzer} %~~~~~~~~~~~~~~~~~~~~~~~~~~~~~~~~~~~~~~~~~~~~~~~~~~
       \textcolor{blue}{\#\#\#\#\#\#\#\#\#}
    \subsubsection{$k$-Nächste-Nachbarn-Schätzer} %~~~~~~~~~~~~~~~~~~~~~~~~~~~~~~~~~~~~~
       \textcolor{blue}{\#\#\#\#\#\#\#\#\#}
  \subsection{Verallgemeinerung auf multivariate Daten} %-------------------------------------------------------------------------------------
       \textcolor{blue}{\#\#\#\#\#\#\#\#\#}
  \subsection{Nichtparametrische Klassifikation} %------------------------------------------------------------------------------------------------
       \textcolor{blue}{\#\#\#\#\#\#\#\#\#}
  \subsection{Verdichtete Nächste-Nachbarn-Methode} %---------------------------------------------------------------------------------------
       \textcolor{blue}{\#\#\#\#\#\#\#\#\#}
  \subsection{Abstandsbasierte Klassifikation} %---------------------------------------------------------------------------------------------------
       \textcolor{blue}{\#\#\#\#\#\#\#\#\#}
  \subsection{Ausreißererkennung} %-----------------------------------------------------------------------------------------------------------------
       \textcolor{blue}{\#\#\#\#\#\#\#\#\#}
  \subsection{Nichtparametrische Regression: Glättungsmodelle} %-------------------------------------------------------------------------
       \textcolor{blue}{\#\#\#\#\#\#\#\#\#}
      \subsubsection{Gleitende Mittelwertglättung} %~~~~~~~~~~~~~~~~~~~~~~~~~~~~~~~~~~~~~~~~
       \textcolor{blue}{\#\#\#\#\#\#\#\#\#}
    \subsubsection{Glättung durch Kernel-Funktion} %~~~~~~~~~~~~~~~~~~~~~~~~~~~~~~~~~~~~~~
       \textcolor{blue}{\#\#\#\#\#\#\#\#\#}
    \subsubsection{Gleitende Linienglättung} %~~~~~~~~~~~~~~~~~~~~~~~~~~~~~~~~~~~~~~~~~~~
       \textcolor{blue}{\#\#\#\#\#\#\#\#\#}
  \subsection{Wahl des Glättungsparameters} %---------------------------------------------------------------------------------------------------
       \textcolor{blue}{\#\#\#\#\#\#\#\#\#}
  \subsection{Anmerkungen} %--------------------------------------------------------------------------------------------------------------------------
       \textcolor{blue}{\#\#\#\#\#\#\#\#\#}
  \subsection{Übungen} %---------------------------------------------------------------------------------------------------------------------------------
       \textcolor{blue}{\#\#\#\#\#\#\#\#\#}
  \subsection{Literaturangaben} %----------------------------------------------------------------------------------------------------------------------
      \textcolor{blue}{\#\#\#\#\#\#\#\#\#}

      \begin{itemize}
      \color{red}
        \item
        \item
        \item
      \end{itemize}


      \begin{itemize}
      \color{ForestGreen}
        \item
        \item
        \item
      \end{itemize}




\newpage
\section{Entscheidungsbäume} %%%%%%%%%%%%%%%%%%%%%%%%%%%%%%%%%%%%%%%%%%%%%%%%%%% 9
  \subsection{Einführung} %------------------------------------------------------------------------------------------------------------------------------ 9.1
      \textcolor{blue}{\#\#\#\#\#\#\#\#\#}
  \subsection{Univariate Bäume} %--------------------------------------------------------------------------------------------------------------------- 9.2
      \textcolor{blue}{\#\#\#\#\#\#\#\#\#}
    \subsubsection{Klassifikationsbäume} %~~~~~~~~~~~~~~~~~~~~~~~~~~~~~~~~~~~~~~~~~~~~~ 9.2.1
      \textcolor{blue}{\#\#\#\#\#\#\#\#\#}
    \subsubsection{Regressionsbäume} %~~~~~~~~~~~~~~~~~~~~~~~~~~~~~~~~~~~~~~~~~~~~~~ 9.2.2
      \textcolor{blue}{\#\#\#\#\#\#\#\#\#}
  \subsection{Pruning} %---------------------------------------------------------------------------------------------------------------------------------- 9.3
      \textcolor{blue}{\#\#\#\#\#\#\#\#\#}
  \subsection{Regelextraktion aus Bäumen} %------------------------------------------------------------------------------------------------------ 9.4
      \textcolor{blue}{\#\#\#\#\#\#\#\#\#}
  \subsection{Lernen von Regeln anhand von Daten} %------------------------------------------------------------------------------------------ 9.5
      \textcolor{blue}{\#\#\#\#\#\#\#\#\#}
  \subsection{Multivariate Bäume} %------------------------------------------------------------------------------------------------------------------- 9.6
      \textcolor{blue}{\#\#\#\#\#\#\#\#\#}
  \subsection{Anmerkungen} %-------------------------------------------------------------------------------------------------------------------------- 9.7
      \textcolor{blue}{\#\#\#\#\#\#\#\#\#}
  \subsection{Übungen} %-------------------------------------------------------------------------------------------------------------------------------- 9.8
      \textcolor{blue}{\#\#\#\#\#\#\#\#\#}
  \subsection{Literaturangaben} %---------------------------------------------------------------------------------------------------------------------- 9.9
      \textcolor{blue}{\#\#\#\#\#\#\#\#\#}

      \begin{itemize}
      \color{red}
        \item
        \item
        \item
      \end{itemize}


      \begin{itemize}
      \color{ForestGreen}
        \item
        \item
        \item
      \end{itemize}




\newpage
\section{Lineare Diskriminanz} %%%%%%%%%%%%%%%%%%%%%%%%%%%%%%%%%%%%%%%%%%%%%%%%%%%% 10
  \subsection{Einführung} %------------------------------------------------------------------------------------------------------------------------------
      \textcolor{blue}{\#\#\#\#\#\#\#\#\#}
  \subsection{Generalisierung des linearen Modells} %-------------------------------------------------------------------------------------------
      \textcolor{blue}{\#\#\#\#\#\#\#\#\#}
  \subsection{Geometrie der linearen Diskriminanz} %--------------------------------------------------------------------------------------------
      \textcolor{blue}{\#\#\#\#\#\#\#\#\#}
    \subsubsection{Zwei Klassen} %~~~~~~~~~~~~~~~~~~~~~~~~~~~~~~~~~~~~~~
      \textcolor{blue}{\#\#\#\#\#\#\#\#\#}
    \subsubsection{Multiple Klassen} %~~~~~~~~~~~~~~~~~~~~~~~~~~~~~~~~~~~~
      \textcolor{blue}{\#\#\#\#\#\#\#\#\#}
  \subsection{Paarweise Trennung} %----------------------------------------------------------------------------------------------------------------
      \textcolor{blue}{\#\#\#\#\#\#\#\#\#}
  \subsection{Neubetrachtung der parametrischen Distanz} %---------------------------------------------------------------------------------
      \textcolor{blue}{\#\#\#\#\#\#\#\#\#}
  \subsection{Gradientenabstieg} %-------------------------------------------------------------------------------------------------------------------
      \textcolor{blue}{\#\#\#\#\#\#\#\#\#}
  \subsection{Logistische Diskriminanz} %-----------------------------------------------------------------------------------------------------------
      \textcolor{blue}{\#\#\#\#\#\#\#\#\#}
    \subsubsection{Zwei Klassen} %~~~~~~~~~~~~~~~~~~~~~~~~~~~~~~~~~~~~~~
      \textcolor{blue}{\#\#\#\#\#\#\#\#\#}
    \subsubsection{Multiple Klassen} %~~~~~~~~~~~~~~~~~~~~~~~~~~~~~~~~~~~~
      \textcolor{blue}{\#\#\#\#\#\#\#\#\#}
  \subsection{Diskriminanz durch Regression} %--------------------------------------------------------------------------------------------------
      \textcolor{blue}{\#\#\#\#\#\#\#\#\#}
  \subsection{Lernen von Rangordnen} %-----------------------------------------------------------------------------------------------------------
      \textcolor{blue}{\#\#\#\#\#\#\#\#\#}
  \subsection{Anmerkungen} %-------------------------------------------------------------------------------------------------------------------------
      \textcolor{blue}{\#\#\#\#\#\#\#\#\#}
  \subsection{Übungen} %--------------------------------------------------------------------------------------------------------------------------------
      \textcolor{blue}{\#\#\#\#\#\#\#\#\#}
  \subsection{Literaturangaben} %----------------------------------------------------------------------------------------------------------------------
      \textcolor{blue}{\#\#\#\#\#\#\#\#\#}

      \begin{itemize}
      \color{red}
        \item
        \item
        \item
      \end{itemize}


      \begin{itemize}
      \color{ForestGreen}
        \item
        \item
        \item
      \end{itemize}




\newpage
\section{Mehrlagige Perzeptronen} %%%%%%%%%%%%%%%%%%%%%%%%%%%%%%%%%%%%%%%%%%%%%%%%%% 11
  \subsection{Einführung} %-------------------------------------------------------------------------------------------------------------------------------
      \textcolor{blue}{\#\#\#\#\#\#\#\#\#}
    \subsubsection{Das Gehirn verstehen} %~~~~~~~~~~~~~~~~~~~~~~~~~~~~~~~~
      \textcolor{blue}{\#\#\#\#\#\#\#\#\#}
    \subsubsection{Neuronale Netze und Parallelverarbeitung} %~~~~~~~~~~~~~~~~~
      \textcolor{blue}{\#\#\#\#\#\#\#\#\#}
  \subsection{Das Perzeptron} %-------------------------------------------------------------------------------------------------------------------------
      \textcolor{blue}{\#\#\#\#\#\#\#\#\#}
  \subsection{Training eines Perzeptrons} %----------------------------------------------------------------------------------------------------------
      \textcolor{blue}{\#\#\#\#\#\#\#\#\#}
  \subsection{Lernen von Booleschen Funktions} %------------------------------------------------------------------------------------------------
      \textcolor{blue}{\#\#\#\#\#\#\#\#\#}
  \subsection{Mehrlagige Perzeptronen} %------------------------------------------------------------------------------------------------------------
      \textcolor{blue}{\#\#\#\#\#\#\#\#\#}
  \subsection{Das MLP als universelle Näherungsfunktion} %------------------------------------------------------------------------------------
      \textcolor{blue}{\#\#\#\#\#\#\#\#\#}
  \subsection{Backpropagation-Algorithmus} %-------------------------------------------------------------------------------------------------------
      \textcolor{blue}{\#\#\#\#\#\#\#\#\#}
  \subsection{Nichtlineare Regression} %---------------------------------------------------------------------------------------------------------------
      \textcolor{blue}{\#\#\#\#\#\#\#\#\#}
    \subsubsection{Nichtlineare Regression} %~~~~~~~~~~~~~~~~~~~~~~~~~~~~~~~
      \textcolor{blue}{\#\#\#\#\#\#\#\#\#}
    \subsubsection{Zweiklassendiskriminanz} %~~~~~~~~~~~~~~~~~~~~~~~~~~~~~~
      \textcolor{blue}{\#\#\#\#\#\#\#\#\#}
    \subsubsection{Diskriminanz bei multiplen Klassen} %~~~~~~~~~~~~~~~~~~~~~~
      \textcolor{blue}{\#\#\#\#\#\#\#\#\#}
    \subsubsection{Multiple verborgene Schichten} %~~~~~~~~~~~~~~~~~~~~~~~~~~
      \textcolor{blue}{\#\#\#\#\#\#\#\#\#}
  \subsection{Trainingsprozeduren} %-------------------------------------------------------------------------------------------------------------------
      \textcolor{blue}{\#\#\#\#\#\#\#\#\#}
    \subsubsection{Verbesserung der Konvergenz} %~~~~~~~~~~~~~~~~~~~~~~~~~~
      \textcolor{blue}{\#\#\#\#\#\#\#\#\#}
    \subsubsection{Übertraining} %~~~~~~~~~~~~~~~~~~~~~~~~~~~~~~~~~~~~~~~
      \textcolor{blue}{\#\#\#\#\#\#\#\#\#}
    \subsubsection{Strukturieren des Netzes} %~~~~~~~~~~~~~~~~~~~~~~~~~~~~~~
      \textcolor{blue}{\#\#\#\#\#\#\#\#\#}
    \subsubsection{Hinweise} %~~~~~~~~~~~~~~~~~~~~~~~~~~~~~~~~~~~~~~~~~~
      \textcolor{blue}{\#\#\#\#\#\#\#\#\#}
  \subsection{Anpassung der Netzgröße} %-----------------------------------------------------------------------------------------------------------
      \textcolor{blue}{\#\#\#\#\#\#\#\#\#}
  \subsection{Bayessche Betrachtungsweise des Lernens} %------------------------------------------------------------------------------------
      \textcolor{blue}{\#\#\#\#\#\#\#\#\#}
  \subsection{Dimensionalitätsreduktion} %------------------------------------------------------------------------------------------------------------
      \textcolor{blue}{\#\#\#\#\#\#\#\#\#}
  \subsection{Lernen mit Zeitreihen} %------------------------------------------------------------------------------------------------------------------
      \textcolor{blue}{\#\#\#\#\#\#\#\#\#}
    \subsubsection{Time Delay Neural Networks} %~~~~~~~~~~~~~~~~~~~~~~~~~~~
      \textcolor{blue}{\#\#\#\#\#\#\#\#\#}
    \subsubsection{Rekurrente Netze} %~~~~~~~~~~~~~~~~~~~~~~~~~~~~~~~~~~~
      \textcolor{blue}{\#\#\#\#\#\#\#\#\#}
  \subsection{Tiefes Lernen} %----------------------------------------------------------------------------------------------------------------------------
      \textcolor{blue}{\#\#\#\#\#\#\#\#\#}
  \subsection{Anmerkungen} %----------------------------------------------------------------------------------------------------------------------------
      \textcolor{blue}{\#\#\#\#\#\#\#\#\#}
  \subsection{Übungen} %----------------------------------------------------------------------------------------------------------------------------------
      \textcolor{blue}{\#\#\#\#\#\#\#\#\#}
  \subsection{Literaturangaben} %------------------------------------------------------------------------------------------------------------------------
      \textcolor{blue}{\#\#\#\#\#\#\#\#\#}
    \subsubsection{} %~~~~~~~~~~~~~~~~~~~~~~~~~~~~~~~~~~~~~~~~~~~~~~~~
      \textcolor{blue}{\#\#\#\#\#\#\#\#\#}

      \begin{itemize}
      \color{red}
        \item
        \item
        \item
      \end{itemize}


      \begin{itemize}
      \color{ForestGreen}
        \item
        \item
        \item
      \end{itemize}




\newpage
\section{Lokale Modelle} %%%%%%%%%%%%%%%%%%%%%%%%%%%%%%%%%%%%%%%%%%%%%%%%%%%%%%% 12
  \subsection{Einführung} %--------------------------------------------------------------------------------------------------------------------------------
      \textcolor{blue}{\#\#\#\#\#\#\#\#\#}
  \subsection{Kompetitives Lernen} %-------------------------------------------------------------------------------------------------------------------
      \textcolor{blue}{\#\#\#\#\#\#\#\#\#}
    \subsubsection{Online k-Means-Algorithmus} %~~~~~~~~~~~~~~~~~~~~~~~~~~~
      \textcolor{blue}{\#\#\#\#\#\#\#\#\#}
    \subsubsection{Adaptive Resonanztheorie} %~~~~~~~~~~~~~~~~~~~~~~~~~~~~~
      \textcolor{blue}{\#\#\#\#\#\#\#\#\#}
    \subsubsection{Selbstorganisierende Merkmalskarten} %~~~~~~~~~~~~~~~~~~~~
      \textcolor{blue}{\#\#\#\#\#\#\#\#\#}
  \subsection{Radiale Basisfunktionen} %--------------------------------------------------------------------------------------------------------------
      \textcolor{blue}{\#\#\#\#\#\#\#\#\#}
  \subsection{Einbinden von regelbasiertem Wissen} %--------------------------------------------------------------------------------------------
      \textcolor{blue}{\#\#\#\#\#\#\#\#\#}
  \subsection{Normalisierte Basisfunktionen} %-------------------------------------------------------------------------------------------------------
      \textcolor{blue}{\#\#\#\#\#\#\#\#\#}
  \subsection{Kompetitive Basisfunktionen} %---------------------------------------------------------------------------------------------------------
      \textcolor{blue}{\#\#\#\#\#\#\#\#\#}
  \subsection{Lernen mit Vektorquantisierung} %-----------------------------------------------------------------------------------------------------
      \textcolor{blue}{\#\#\#\#\#\#\#\#\#}
  \subsection{Gemischte Expertensysteme} %--------------------------------------------------------------------------------------------------------
      \textcolor{blue}{\#\#\#\#\#\#\#\#\#}
    \subsubsection{Kooperative Expertensysteme} %~~~~~~~~~~~~~~~~~~~~~~~~~~
      \textcolor{blue}{\#\#\#\#\#\#\#\#\#}
    \subsubsection{Kompetitive Expertensysteme} %~~~~~~~~~~~~~~~~~~~~~~~~~~
      \textcolor{blue}{\#\#\#\#\#\#\#\#\#}
  \subsection{Hierarchisch gemischte Expertensysteme} %---------------------------------------------------------------------------------------
      \textcolor{blue}{\#\#\#\#\#\#\#\#\#}
  \subsection{Anmerkungen} %----------------------------------------------------------------------------------------------------------------------------
      \textcolor{blue}{\#\#\#\#\#\#\#\#\#}
  \subsection{Übungen} %----------------------------------------------------------------------------------------------------------------------------------
      \textcolor{blue}{\#\#\#\#\#\#\#\#\#}
  \subsection{Literaturangaben} %------------------------------------------------------------------------------------------------------------------------
      \textcolor{blue}{\#\#\#\#\#\#\#\#\#}

      \begin{itemize}
      \color{red}
        \item
        \item
      \color{ForestGreen}
        \item
        \item
      \end{itemize}




\newpage
\section{Kernel-Maschinen} %%%%%%%%%%%%%%%%%%%%%%%%%%%%%%%%%%%%%%%%%%%%%%%%%%%%% 13
  \subsection{Einführung} %--------------------------------------------------------------------------------------------------------------------------------
      \textcolor{blue}{\#\#\#\#\#\#\#\#\#}
  \subsection{Die optimal trennende Hyperebene} %------------------------------------------------------------------------------------------------
      \textcolor{blue}{\#\#\#\#\#\#\#\#\#}
  \subsection{Der nicht trennbare Fall: Soft-Margin-Trennebenen} %---------------------------------------------------------------------------
      \textcolor{blue}{\#\#\#\#\#\#\#\#\#}
  \subsection{$v$-SVM} %----------------------------------------------------------------------------------------------------------------------------------
      \textcolor{blue}{\#\#\#\#\#\#\#\#\#}
  \subsection{Kernel-Trick} %------------------------------------------------------------------------------------------------------------------------------
      \textcolor{blue}{\#\#\#\#\#\#\#\#\#}
  \subsection{Vektorielle Kernel} %-----------------------------------------------------------------------------------------------------------------------
      \textcolor{blue}{\#\#\#\#\#\#\#\#\#}
  \subsection{Definition von Kerneln} %-----------------------------------------------------------------------------------------------------------------
      \textcolor{blue}{\#\#\#\#\#\#\#\#\#}
  \subsection{Multiple-Kernel-Lernen} %----------------------------------------------------------------------------------------------------------------
      \textcolor{blue}{\#\#\#\#\#\#\#\#\#}
  \subsection{Mehrklassen-Kernel-Maschinen} %----------------------------------------------------------------------------------------------------
      \textcolor{blue}{\#\#\#\#\#\#\#\#\#}
  \subsection{Kernel-Maschinen und Regression} %------------------------------------------------------------------------------------------------
      \textcolor{blue}{\#\#\#\#\#\#\#\#\#}
  \subsection{Kernel-Maschinen und Ranking} %----------------------------------------------------------------------------------------------------
      \textcolor{blue}{\#\#\#\#\#\#\#\#\#}
  \subsection{Einklassen-Kernel-Maschinen} %------------------------------------------------------------------------------------------------------
      \textcolor{blue}{\#\#\#\#\#\#\#\#\#}
  \subsection{Breiter-Margin-Nächster-Nachbar-Klassifikator} %--------------------------------------------------------------------------------
      \textcolor{blue}{\#\#\#\#\#\#\#\#\#}
  \subsection{Dimensionalitätsreduktion mit Kernel} %---------------------------------------------------------------------------------------------
      \textcolor{blue}{\#\#\#\#\#\#\#\#\#}
  \subsection{Anmerkungen} %---------------------------------------------------------------------------------------------------------------------------
      \textcolor{blue}{\#\#\#\#\#\#\#\#\#}
  \subsection{Übungen} %----------------------------------------------------------------------------------------------------------------------------------
      \textcolor{blue}{\#\#\#\#\#\#\#\#\#}
  \subsection{Literaturangaben} %------------------------------------------------------------------------------------------------------------------------
      \textcolor{blue}{\#\#\#\#\#\#\#\#\#}

      \begin{itemize}
      \color{red}
        \item
        \item
      \color{ForestGreen}
        \item
        \item
      \end{itemize}




\newpage
\section{Graphenmodelle} %%%%%%%%%%%%%%%%%%%%%%%%%%%%%%%%%%%%%%%%%%%%%%%%%%%%%% 14
  \subsection{Einführung} %--------------------------------------------------------------------------------------------------------------------------------
      \textcolor{blue}{\#\#\#\#\#\#\#\#\#}
  \subsection{Kanonische Fälle für bedingte Unabhängigkeit} %---------------------------------------------------------------------------------
      \textcolor{blue}{\#\#\#\#\#\#\#\#\#}
  \subsection{Generative Modelle} %--------------------------------------------------------------------------------------------------------------------
      \textcolor{blue}{\#\#\#\#\#\#\#\#\#}
  \subsection{d-Separation} %-----------------------------------------------------------------------------------------------------------------------------
      \textcolor{blue}{\#\#\#\#\#\#\#\#\#}
  \subsection{Belief-Propagation} %---------------------------------------------------------------------------------------------------------------------
      \textcolor{blue}{\#\#\#\#\#\#\#\#\#}
    \subsubsection{Ketten} %~~~~~~~~~~~~~~~~~~~~~~~~~~~~~~~~~~~~~~~~~~~~
      \textcolor{blue}{\#\#\#\#\#\#\#\#\#}
    \subsubsection{Bäume} %~~~~~~~~~~~~~~~~~~~~~~~~~~~~~~~~~~~~~~~~~~~~
      \textcolor{blue}{\#\#\#\#\#\#\#\#\#}
    \subsubsection{Mehrfachbäume} %~~~~~~~~~~~~~~~~~~~~~~~~~~~~~~~~~~~~~
      \textcolor{blue}{\#\#\#\#\#\#\#\#\#}
    \subsubsection{Verbindungsbäume} %~~~~~~~~~~~~~~~~~~~~~~~~~~~~~~~~~~~
      \textcolor{blue}{\#\#\#\#\#\#\#\#\#}
  \subsection{Ungerichtete Graphen: Markovsche Zufallsfelder} %-----------------------------------------------------------------------------
      \textcolor{blue}{\#\#\#\#\#\#\#\#\#}
  \subsection{Lernen der Struktur eines Graphenmodells} %-------------------------------------------------------------------------------------
      \textcolor{blue}{\#\#\#\#\#\#\#\#\#}
  \subsection{Einflussdiagramme} %--------------------------------------------------------------------------------------------------------------------
      \textcolor{blue}{\#\#\#\#\#\#\#\#\#}
  \subsection{Anmerkungen} %---------------------------------------------------------------------------------------------------------------------------
      \textcolor{blue}{\#\#\#\#\#\#\#\#\#}
  \subsection{Übungen} %----------------------------------------------------------------------------------------------------------------------------------
      \textcolor{blue}{\#\#\#\#\#\#\#\#\#}
  \subsection{Literaturangaben} %-----------------------------------------------------------------------------------------------------------------------
      \textcolor{blue}{\#\#\#\#\#\#\#\#\#}

      \begin{itemize}
      \color{red}
        \item
        \item
      \color{ForestGreen}
        \item
        \item
      \end{itemize}




\newpage
\section{Hidden-Markov-Modelle} %%%%%%%%%%%%%%%%%%%%%%%%%%%%%%%%%%%%%%%%%%%%%%%%%% 15
  \subsection{Einführung} %--------------------------------------------------------------------------------------------------------------------------------
      \textcolor{blue}{\#\#\#\#\#\#\#\#\#}
  \subsection{Diskrete Markov-Prozesse} %-----------------------------------------------------------------------------------------------------------
      \textcolor{blue}{\#\#\#\#\#\#\#\#\#}
  \subsection{Hidden-Markov-Modelle} %--------------------------------------------------------------------------------------------------------------
      \textcolor{blue}{\#\#\#\#\#\#\#\#\#}
  \subsection{Drei grundsätzliche Probleme eines HMM} %--------------------------------------------------------------------------------------
      \textcolor{blue}{\#\#\#\#\#\#\#\#\#}
  \subsection{Evalueierungsproblem} %----------------------------------------------------------------------------------------------------------------
      \textcolor{blue}{\#\#\#\#\#\#\#\#\#}
  \subsection{Herausfinden der Zustandssequenz} %----------------------------------------------------------------------------------------------
      \textcolor{blue}{\#\#\#\#\#\#\#\#\#}
  \subsection{Lernen von Modellparametern} %------------------------------------------------------------------------------------------------------
      \textcolor{blue}{\#\#\#\#\#\#\#\#\#}
  \subsection{Kontinuierliche Beobachtungen} %----------------------------------------------------------------------------------------------------
      \textcolor{blue}{\#\#\#\#\#\#\#\#\#}
  \subsection{Das HMM als Graphenmodell} %------------------------------------------------------------------------------------------------------
      \textcolor{blue}{\#\#\#\#\#\#\#\#\#}
  \subsection{Modellauswahl im HMM} %--------------------------------------------------------------------------------------------------------------
      \textcolor{blue}{\#\#\#\#\#\#\#\#\#}
  \subsection{Anmerkungen} %---------------------------------------------------------------------------------------------------------------------------
      \textcolor{blue}{\#\#\#\#\#\#\#\#\#}
  \subsection{Übungen} %----------------------------------------------------------------------------------------------------------------------------------
      \textcolor{blue}{\#\#\#\#\#\#\#\#\#}
  \subsection{Literaturangaben} %-----------------------------------------------------------------------------------------------------------------------
      \textcolor{blue}{\#\#\#\#\#\#\#\#\#}

      \begin{itemize}
      \color{red}
        \item
        \item
      \color{ForestGreen}
        \item
        \item
      \end{itemize}




\newpage
\section{Bayessche Schätzung} %%%%%%%%%%%%%%%%%%%%%%%%%%%%%%%%%%%%%%%%%%%%%%%%%%% 16
  \subsection{Einführung} %--------------------------------------------------------------------------------------------------------------------------------
      \textcolor{blue}{\#\#\#\#\#\#\#\#\#}
  \subsection{Bayessche Schätzung der Parameter diskreter Verteilungen} %---------------------------------------------------------------
      \textcolor{blue}{\#\#\#\#\#\#\#\#\#}
    \subsubsection{$K>2$-Zustände: Dirichlet-Verteilung} %~~~~~~~~~~~~~~~~~~~~
      \textcolor{blue}{\#\#\#\#\#\#\#\#\#}
    \subsubsection{$K>2$-Zustände: Betaverteilung} %~~~~~~~~~~~~~~~~~~~~~~~~
      \textcolor{blue}{\#\#\#\#\#\#\#\#\#}
  \subsection{Bayessche Schätzung der Parameter einer Gau-Verteilung} %-----------------------------------------------------------------
      \textcolor{blue}{\#\#\#\#\#\#\#\#\#}
    \subsubsection{Univariater Fall: Unbekannter Mittelwert, bekannte Varianz} %~~~~~
      \textcolor{blue}{\#\#\#\#\#\#\#\#\#}
    \subsubsection{Univariater Fall: Unbekannter Mittelwert, unbekannte Varianz} %~~~
      \textcolor{blue}{\#\#\#\#\#\#\#\#\#}
    \subsubsection{Univariater Fall: Unbekannter Mittelwert, unbekannte Kovarianz} %~
      \textcolor{blue}{\#\#\#\#\#\#\#\#\#}
  \subsection{Bayessche Schätzung der Parameter einer Funktion} %--------------------------------------------------------------------------
      \textcolor{blue}{\#\#\#\#\#\#\#\#\#}
    \subsubsection{Regression} %~~~~~~~~~~~~~~~~~~~~~~~~~~~~~~~~~~~~~~~~
      \textcolor{blue}{\#\#\#\#\#\#\#\#\#}
    \subsubsection{Regression mit Prior für die Präzision des Rauschens} %~~~~~~~~
      \textcolor{blue}{\#\#\#\#\#\#\#\#\#}
    \subsubsection{Der Gebrauch von Basis/Kernel-Funktionen} %~~~~~~~~~~~~~~~
      \textcolor{blue}{\#\#\#\#\#\#\#\#\#}
    \subsubsection{Bayessche Klassifikation} %~~~~~~~~~~~~~~~~~~~~~~~~~~~~~
      \textcolor{blue}{\#\#\#\#\#\#\#\#\#}
  \subsection{Wahl eines Priors} %------------------------------------------------------------------------------------------------------------------------
      \textcolor{blue}{\#\#\#\#\#\#\#\#\#}
  \subsection{Bayesscher Modellvergleich} %----------------------------------------------------------------------------------------------------------
      \textcolor{blue}{\#\#\#\#\#\#\#\#\#}
  \subsection{Bayessche Schätzung für ein Mischungsmodell} %--------------------------------------------------------------------------------
      \textcolor{blue}{\#\#\#\#\#\#\#\#\#}
  \subsection{Nichtparametrische Bayessche Modelle} %------------------------------------------------------------------------------------------
      \textcolor{blue}{\#\#\#\#\#\#\#\#\#}
  \subsection{Gausche Prozesse} %----------------------------------------------------------------------------------------------------------------------
      \textcolor{blue}{\#\#\#\#\#\#\#\#\#}
  \subsection{Dirichlet-Prozesse und Chinaestaurants} %------------------------------------------------------------------------------------------
      \textcolor{blue}{\#\#\#\#\#\#\#\#\#}
  \subsection{Latente Dirichlet-Allokation} %-----------------------------------------------------------------------------------------------------------
      \textcolor{blue}{\#\#\#\#\#\#\#\#\#}
  \subsection{Betaprozesse und indische Büffets} %-------------------------------------------------------------------------------------------------
      \textcolor{blue}{\#\#\#\#\#\#\#\#\#}
  \subsection{Anmerkungen} %----------------------------------------------------------------------------------------------------------------------------
      \textcolor{blue}{\#\#\#\#\#\#\#\#\#}
  \subsection{Übungen} %----------------------------------------------------------------------------------------------------------------------------------
      \textcolor{blue}{\#\#\#\#\#\#\#\#\#}
  \subsection{Literaturangaben} %------------------------------------------------------------------------------------------------------------------------
      \textcolor{blue}{\#\#\#\#\#\#\#\#\#}

      \begin{itemize}
      \color{red}
        \item
        \item
       \color{ForestGreen}
        \item
        \item
      \end{itemize}




\newpage
\section{Kombination mehrer Lerner} %%%%%%%%%%%%%%%%%%%%%%%%%%%%%%%%%%%%%%%%%%%%%%%% 17
  \subsection{Einführung} %--------------------------------------------------------------------------------------------------------------------------------
      \textcolor{blue}{\#\#\#\#\#\#\#\#\#}
  \subsection{Generierung diverser Lerner} %---------------------------------------------------------------------------------------------------------
      \textcolor{blue}{\#\#\#\#\#\#\#\#\#}
  \subsection{Methoden der Modellkombination} %--------------------------------------------------------------------------------------------------
      \textcolor{blue}{\#\#\#\#\#\#\#\#\#}
  \subsection{Voting} %--------------------------------------------------------------------------------------------------------------------------------------
      \textcolor{blue}{\#\#\#\#\#\#\#\#\#}
  \subsection{Fehlerkorrekturcodes} %------------------------------------------------------------------------------------------------------------------
      \textcolor{blue}{\#\#\#\#\#\#\#\#\#}
  \subsection{Bagging} %-----------------------------------------------------------------------------------------------------------------------------------
      \textcolor{blue}{\#\#\#\#\#\#\#\#\#}
  \subsection{Boosting} %----------------------------------------------------------------------------------------------------------------------------------
      \textcolor{blue}{\#\#\#\#\#\#\#\#\#}
  \subsection{Neubetrachtung der gemischten Expertensysteme} %---------------------------------------------------------------------------
      \textcolor{blue}{\#\#\#\#\#\#\#\#\#}
  \subsection{Geschachelte Generalisierung} %------------------------------------------------------------------------------------------------------
      \textcolor{blue}{\#\#\#\#\#\#\#\#\#}
  \subsection{Feinabstimmung eines Ensembles} %------------------------------------------------------------------------------------------------
      \textcolor{blue}{\#\#\#\#\#\#\#\#\#}
    \subsubsection{Wahl einer Teilmenge des Ensembles} %~~~~~~~~~~~~~~~~~~~~~~~~~~~~~
      \textcolor{blue}{\#\#\#\#\#\#\#\#\#}
    \subsubsection{Konstruktion von Metalernern} %~~~~~~~~~~~~~~~~~~~~~~~~~~~~~~~~~~~
      \textcolor{blue}{\#\#\#\#\#\#\#\#\#}
  \subsection{Kaskadierung} %----------------------------------------------------------------------------------------------------------------------------
      \textcolor{blue}{\#\#\#\#\#\#\#\#\#}
  \subsection{Anmerkungen} %----------------------------------------------------------------------------------------------------------------------------
      \textcolor{blue}{\#\#\#\#\#\#\#\#\#}
  \subsection{Übungen} %----------------------------------------------------------------------------------------------------------------------------------
      \textcolor{blue}{\#\#\#\#\#\#\#\#\#}
  \subsection{Literaturangaben} %-----------------------------------------------------------------------------------------------------------------------
      \textcolor{blue}{\#\#\#\#\#\#\#\#\#}

      \begin{itemize}
      \color{red}
        \item
        \item
      \color{ForestGreen}
        \item
        \item
      \end{itemize}




\newpage
\section{Bestärkendes Lernen} %%%%%%%%%%%%%%%%%%%%%%%%%%%%%%%%%%%%%%%%%%%%%%%%%%% 18
  \subsection{Einführung} %--------------------------------------------------------------------------------------------------------------------------------
      \textcolor{blue}{\#\#\#\#\#\#\#\#\#}
  \subsection{Fälle mit einem Zustand: $K$-armiger Bandit} %----------------------------------------------------------------------------------
      \textcolor{blue}{\#\#\#\#\#\#\#\#\#}
  \subsection{Elemente des bestärkenden Lernens} %---------------------------------------------------------------------------------------------
      \textcolor{blue}{\#\#\#\#\#\#\#\#\#}
  \subsection{Modellbasiertes Lernen} %---------------------------------------------------------------------------------------------------------------
      \textcolor{blue}{\#\#\#\#\#\#\#\#\#}
    \subsubsection{Wertiteration} %~~~~~~~~~~~~~~~~~~~~~~~~~~~~~~~~~~~~~~~~~~~~~~~~
      \textcolor{blue}{\#\#\#\#\#\#\#\#\#}
    \subsubsection{Taktiteration} %~~~~~~~~~~~~~~~~~~~~~~~~~~~~~~~~~~~~~~~~~~~~~~~~~
      \textcolor{blue}{\#\#\#\#\#\#\#\#\#}
  \subsection{Lernen mit temporaler Differenz} %----------------------------------------------------------------------------------------------------
      \textcolor{blue}{\#\#\#\#\#\#\#\#\#}
    \subsubsection{Explorationsstrategien} %~~~~~~~~~~~~~~~~~~~~~~~~~~~~~~~~~~~~~~~~~
      \textcolor{blue}{\#\#\#\#\#\#\#\#\#}
    \subsubsection{Deterministische Belohnungen} %~~~~~~~~~~~~~~~~~~~~~~~~~~~~~~~~~~~
      \textcolor{blue}{\#\#\#\#\#\#\#\#\#}
    \subsubsection{Nichtdeterministische Belohnungen und Aktionen} %~~~~~~~~~~~~~~~~~~~~~
      \textcolor{blue}{\#\#\#\#\#\#\#\#\#}
    \subsubsection{Eignungsprotokolle} %~~~~~~~~~~~~~~~~~~~~~~~~~~~~~~~~~~~~~~~~~~~~
      \textcolor{blue}{\#\#\#\#\#\#\#\#\#}
  \subsection{Generalisierung} %-------------------------------------------------------------------------------------------------------------------------
      \textcolor{blue}{\#\#\#\#\#\#\#\#\#}
  \subsection{Teilweise beobachtbare Zustände} %-------------------------------------------------------------------------------------------------
      \textcolor{blue}{\#\#\#\#\#\#\#\#\#}
  \subsection{Beispiel: Das Tigerproblem} %----------------------------------------------------------------------------------------------------------
      \textcolor{blue}{\#\#\#\#\#\#\#\#\#}
  \subsection{Anmerkungen} %---------------------------------------------------------------------------------------------------------------------------
      \textcolor{blue}{\#\#\#\#\#\#\#\#\#}
  \subsection{Übungen} %----------------------------------------------------------------------------------------------------------------------------------
      \textcolor{blue}{\#\#\#\#\#\#\#\#\#}
  \subsection{Literaturangaben} %-----------------------------------------------------------------------------------------------------------------------
      \textcolor{blue}{\#\#\#\#\#\#\#\#\#}

      \begin{itemize}
      \color{red}
        \item
        \item
      \color{ForestGreen}
        \item
        \item
      \end{itemize}




\newpage
\section{Experimente mit maschinellem Lernen} %%%%%%%%%%%%%%%%%%%%%%%%%%%%%%%%%%%%%%%%%%% 19
  \subsection{Einführung} %--------------------------------------------------------------------------------------------------------------------------------
       \textcolor{blue}{\#\#\#\#\#\#\#\#\#}
  \subsection{Faktoren, Antwort und Strategie beim Experimentieren} %----------------------------------------------------------------------
       \textcolor{blue}{\#\#\#\#\#\#\#\#\#}
  \subsection{Antwortflächenmethode} %---------------------------------------------------------------------------------------------------------------
       \textcolor{blue}{\#\#\#\#\#\#\#\#\#}
  \subsection{Randomisieren, Wiederholen und Blocken} %--------------------------------------------------------------------------------------
       \textcolor{blue}{\#\#\#\#\#\#\#\#\#}
  \subsection{Richtlinien für Experimente mit maschinellem Lernen} %------------------------------------------------------------------------
       \textcolor{blue}{\#\#\#\#\#\#\#\#\#}
  \subsection{Kreuzvalidierung und Resampling-Methoden} %-----------------------------------------------------------------------------------
       \textcolor{blue}{\#\#\#\#\#\#\#\#\#}
    \subsubsection{$K$-fache Kreuzvalidierung} %~~~~~~~~~~~~~~~~~~~~~~~~~~~~~~~~~~~~~~
       \textcolor{blue}{\#\#\#\#\#\#\#\#\#}
    \subsubsection{$5 \times 2$-Kreuzvalidierung} %~~~~~~~~~~~~~~~~~~~~~~~~~~~~~~~~~~~~
       \textcolor{blue}{\#\#\#\#\#\#\#\#\#}
    \subsubsection{Bootstrapping} %~~~~~~~~~~~~~~~~~~~~~~~~~~~~~~~~~~~~~~~~~~~~~~~~
       \textcolor{blue}{\#\#\#\#\#\#\#\#\#}
  \subsection{Leistungsmessung für Klassifikatoren} %---------------------------------------------------------------------------------------------
       \textcolor{blue}{\#\#\#\#\#\#\#\#\#}
  \subsection{Intervallschätzung} %-----------------------------------------------------------------------------------------------------------------------
       \textcolor{blue}{\#\#\#\#\#\#\#\#\#}
  \subsection{Hypothesenprüfung} %---------------------------------------------------------------------------------------------------------------------
       \textcolor{blue}{\#\#\#\#\#\#\#\#\#}
  \subsection{Bewertung der Leistungsfähigkeit von Klassifikationsalgorithmen} %----------------------------------------------------------
       \textcolor{blue}{\#\#\#\#\#\#\#\#\#}
    \subsubsection{Binomialtest} %~~~~~~~~~~~~~~~~~~~~~~~~~~~~~~~~~~~~~~~~~~~~~~~~~~
       \textcolor{blue}{\#\#\#\#\#\#\#\#\#}
    \subsubsection{Test der Approzimierten Normalverteilung} %~~~~~~~~~~~~~~~~~~~~~~~~~~~~~~~~~~~~~~~~~~~~~~~~
       \textcolor{blue}{\#\#\#\#\#\#\#\#\#}
    \subsubsection{$t$-Test} %~~~~~~~~~~~~~~~~~~~~~~~~~~~~~~~~~~~~~~~~~~~~~~~~~~~~~
       \textcolor{blue}{\#\#\#\#\#\#\#\#\#}
  \subsection{Vergleich von zwei Klassifikationsalgorithmen} %-----------------------------------------------------------------------------------
       \textcolor{blue}{\#\#\#\#\#\#\#\#\#}
    \subsubsection{Der McNemarsche Test} %~~~~~~~~~~~~~~~~~~~~~~~~~~~~~~~~~~~~~~~~~
       \textcolor{blue}{\#\#\#\#\#\#\#\#\#}
    \subsubsection{Gepaarter $t$-Test mit $K$-facher Kreuzvalidierung} %~~~~~~~~~~~~~~~~~~~~
       \textcolor{blue}{\#\#\#\#\#\#\#\#\#}
    \subsubsection{Gepaarter $t$-Test mit $5 \times 2$-facher Kreuzvalidierung} %~~~~~~~~~~~~~~
       \textcolor{blue}{\#\#\#\#\#\#\#\#\#}
    \subsubsection{Gepaarter $F$-Test mit $5 \times 2$-facher Kreuzvalidierung} %~~~~~~~~~~~~~
       \textcolor{blue}{\#\#\#\#\#\#\#\#\#}
  \subsection{Vergleich mehrerer Algorithmen: Varianzanalyse} %-------------------------------------------------------------------------------
       \textcolor{blue}{\#\#\#\#\#\#\#\#\#}
  \subsection{Vergleich über mehrere Datenmengen} %--------------------------------------------------------------------------------------------
       \textcolor{blue}{\#\#\#\#\#\#\#\#\#}
    \subsubsection{Vergleich zweier Algorithmen} %~~~~~~~~~~~~~~~~~~~~~~~~~~~~~~~~~~~~~
       \textcolor{blue}{\#\#\#\#\#\#\#\#\#}
    \subsubsection{Vergleich mehrerer Algorithmen} %~~~~~~~~~~~~~~~~~~~~~~~~~~~~~~~~~~~
       \textcolor{blue}{\#\#\#\#\#\#\#\#\#}
  \subsection{Multivariate Tests} %------------------------------------------------------------------------------------------------------------------------
       \textcolor{blue}{\#\#\#\#\#\#\#\#\#}
    \subsubsection{Vergleich zweier Algorithmen} %~~~~~~~~~~~~~~~~~~~~~~~~~~~~~~~~~~~~~
       \textcolor{blue}{\#\#\#\#\#\#\#\#\#}
    \subsubsection{Vergleich mehrerer Algorithmen} %~~~~~~~~~~~~~~~~~~~~~~~~~~~~~~~~~~~
       \textcolor{blue}{\#\#\#\#\#\#\#\#\#}
  \subsection{Anmerkungen} %----------------------------------------------------------------------------------------------------------------------------
       \textcolor{blue}{\#\#\#\#\#\#\#\#\#}
  \subsection{Übungen} %----------------------------------------------------------------------------------------------------------------------------------
       \textcolor{blue}{\#\#\#\#\#\#\#\#\#}
  \subsection{Literaturangaben} %------------------------------------------------------------------------------------------------------------------------
       \textcolor{blue}{\#\#\#\#\#\#\#\#\#}

      \begin{itemize}
      \color{red}
        \item
        \item
      \color{ForestGreen}
        \item
        \item
      \end{itemize}




\newpage

\end{document}

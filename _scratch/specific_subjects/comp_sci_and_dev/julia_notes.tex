\documentclass[a4paper]{article}
\usepackage[utf8]{inputenc}
\usepackage{geometry}
\usepackage[pagebackref=true]{hyperref}
\usepackage{lmodern}
\usepackage{amsmath}
\usepackage{amssymb}
\usepackage{pifont}
\usepackage{bbm}
\usepackage{stmaryrd}
\usepackage{mathtools}
\usepackage{verbatim}

\newcommand{\followup}[1]{\textcolor{red}{ #1 }}
\newcommand{\Ubr}[2]{\underbrace{ #1 }_{\mathclap{\text{ #2 }}}}
\newcommand{\Br}[1]{\{ #1 \}}

\geometry{margin=2cm}

\title{Julia (Programming Language) - Study Notes}
\author{Isaac Riley}
\date{Last edited: \today}

\begin{document}
\maketitle
\tableofcontents
\newpage

%%%%%%%%%%%%%%%%%%%%%%%%%%%%%%%%%%%%%%%%%%%%%%%%%%%%%%%%%%%%%%%
%========================================================
%--------------------------------------------------

\section{Articles and Blogs}
Julia's mutually-reinforcing triad (\href{https://redrapids.medium.com/why-is-julias-flux-catching-fire-for-ml-8763848df90c}{link}):
\begin{itemize}
    \item transparent
    \item performant
    \item productive
\end{itemize}

\noindent\texttt{using Foo} imports the entire namespace, unlike in Python. However, the very Pythonic \texttt{import Foo as Bar} is now possible in Julia. This is mostly useful when the package name conflicts with an identifier; abbreviations were already possible thanks to the keyword \texttt{const}.

\end{document}

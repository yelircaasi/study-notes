\documentclass[a4paper,12pt]{article}
\usepackage[utf8]{inputenc}
\usepackage{geometry}
\usepackage[pagebackref=true]{hyperref}
\usepackage{lmodern}
\usepackage{amsmath}
\usepackage{amssymb}
\usepackage{pifont}
\usepackage{bbm}
\usepackage{xcolor}
\usepackage{stmaryrd}
\usepackage{mathtools}

\newcommand{\Ubr}[2]{\underbrace{ #1 }_{\mathclap{\text{ #2 }}}}
\newcommand{\Br}[1]{\{\, #1 \,\}}
\newcommand{\R}{\mathbbm{R}}
\newcommand{\Z}{\mathbbm{Z}}
\newcommand{\N}{\mathbbm{N}}
\newcommand{\Q}{\mathbbm{Q}}
\newcommand{\M}[1]{ \begin{bmatrix} #1 \end{bmatrix} }
\newcommand{\vsp}[1]{\vspace{#1mm}\\}
\newcommand{\vecx}{\textbf{x}}
\newcommand{\vecy}{\textbf{y}}
\newcommand{\vecz}{\textbf{z}}
\newcommand{\veco}{\textbf{0}}
\newcommand{\veca}{\textbf{a}}
\newcommand{\vecb}{\textbf{b}}
\newcommand{\vecc}{\textbf{c}}
\newcommand{\vece}{\textbf{e}}
\newcommand{\matA}{\textbf{A}}
\newcommand{\matB}{\textbf{B}}
%\newcommand{\tab}[2]{\begin{tabular}{#1} #2 \end{tabular}}
\newcommand{\inv}{^{-1}}
\newcommand{\pr}{^\prime}
\newcommand{\image}{\textrm{Im}}
\newcommand{\stsum}{\sum_{i=1}^n}
\newcommand{\id}{\textrm{id}}
\newcommand{\rr}[1]{\tiny \stackrel{\tiny #1}{\xrightarrow{\hspace{6mm}}}}
%\newcommand{\arr}[1]{\stackrel{#1}{\longrightarrow}}

\geometry{margin=2cm}

\title{Math for Machine Learning - Solutions to Exercises}
\author{Isaac Riley}
%\date{August 2020}

\begin{document}
\maketitle
\tableofcontents
\newpage



%%%%%%%%%%%%%%%%%%%%%%%%%%%%%%%%%%%%%%%%%%%%%%%%%%%%%%%%%%%%%%%
\setcounter{section}{1}
\newpage%%%%%%%%%%%%%%%%%%%%%%%%%%%%%%%%%%%%%%%%%%%%%%%%%%%%%%%%%%%%%%%
\section{Linear Algebra}
%>>>>>>>>>>>>>>   2.1  >>>>>>>>>>>>>>>>>>>>>>>>>>>>>>>>>>>>>>>>>>>>>>>>>>>>>>>>>>>>>>>>>>>>>>>>>>>>>>>>>>>>>>>>>>>>>>>>>>>>>>>>>>>>>>>>>>>>>>>>>>>>>>>>>>>>>>>>>>>>>>>>>>>
\subsection{}
We consider $(\R \setminus \Br{-1}, \star)$, where
\begin{equation} \tag{2.134}
a \star b := ab+a+b, \hspace{1cm} a,b \in \R \setminus \Br{>1}
\end{equation}
\begin{itemize}
 \item [a.] Show that $(\R \setminus \Br{-1}, \star)$ is an Abelian group.
 \item [b.] Solve
\begin{equation*}
3 \star x \star x = 15
\end{equation*}
in the Abelian group $(\R \setminus \Br{-1}, \star)$, where $\star$ is defined in (2.134).
\end{itemize}
%------------------------------------------------------------------------------------------------------------------------------------------------------------------------
\textcolor{blue}{
\begin{itemize}
 \item [a.]
 \item [b.]
\end{itemize}
}
%>>>>>>>>>>>>>>   2.2  >>>>>>>>>>>>>>>>>>>>>>>>>>>>>>>>>>>>>>>>>>>>>>>>>>>>>>>>>>>>>>>>>>>>>>>>>>>>>>>>>>>>>>>>>>>>>>>>>>>>>>>>>>>>>>>>>>>>>>>>>>>>>>>>>>>>>>>>>>>>>>>>>>>
\subsection{}
Let $n$ be in $\N \setminus \Br{0}$. Let $k,x$ be in $\Z$. We define the congruence class $\overline{k}$ of the integer $k$ as the set
\begin{align*}
\overline{k} &= \Br{ x \in \Z \mid x-k=0 (\mod n) } \vspace{1mm}\\
&= \Br{ x \in \Z \mid \exists a \in \Z: (x-k = n \cdot a) }.
\end{align*}
We now define $\Z / n\Z$ (sometimes written $\Z_n$) as the set of all congruence classes modulo $n$. Euclidean division implies that this set is a finite set containing $n$ elements:
$$\Z_n = \Br{\overline{0}, \overline{1},...,\overline{n-1}}$$
For all $\overline{a}, \overline{b} \in \Z_n$, we define
$$\overline{a} \oplus \overline{b} := \overline{a+b}$$
\begin{itemize}
 \item [a.] Show that $(Z_n,\oplus)$ is a group. Is it Abelian?
 \item [b.] We now define another operation $\otimes$ for all $\overline{a}$ and $\overline{b}$ in $\Z_n$ as
 $$\overline{a} \otimes \overline{b} = \overline{a \times b},$$
 where $a \times b$ represents the usual multiplication in $\Z$.\\
 Let $n=5$. Draw the times table of the elements of $\Z_5 \setminus \Br{\overline{0}}$ under $\otimes$, i.e., calculate the products $\overline{a} \otimes \overline{b}$ for all $\overline{a}$ and $\overline{b}$ in $\Z_5 \setminus \Br{\overline{0}}$.\\
 Hence, show that $\Z_5 \setminus \Br{\overline{0}}$ is closed under $\otimes$ and possesses a neutral element for $\otimes$.
 Display the inverse of all elements in $\Z_5 \setminus \Br{\overline{0}}$ under $\otimes$.\\
 Conclude that $(\Z_5 \setminus \Br{\overline{0}}, \otimes)$ is an Abelian group.
 \item [c.] Show that $\Z_8 \setminus \Br{\overline{0}}, \otimes$ is not a group.
 \item [d.] We recall that the Bézout theorem states that two integers $a$ and $b$ are relatively prime (i.e., $gcd(a,b)=1$) if an only if there exist two integers $u$ and $v$ such that $au+bv=1$. Show that $(\Z_n \setminus \Br{\overline{0}, \otimes})$ is a group if and only if $n \in \N \setminus \Br{0}$ is prime.
\end{itemize}
%------------------------------------------------------------------------------------------------------------------------------------------------------------------------
\textcolor{blue}{
\begin{itemize}
 \item [a.]
 \item [b.]
 \item [c.]
 \item [d.]
\end{itemize}
}
%>>>>>>>>>>>>>>   2.3  >>>>>>>>>>>>>>>>>>>>>>>>>>>>>>>>>>>>>>>>>>>>>>>>>>>>>>>>>>>>>>>>>>>>>>>>>>>>>>>>>>>>>>>>>>>>>>>>>>>>>>>>>>>>>>>>>>>>>>>>>>>>>>>>>>>>>>>>>>>>>>>>>>>
\subsection{}
Consider the set $\mathcal{G}$ of $3 \times 3$ matrices defined as follows:
$$\mathcal{G} = \left\{ \, \M{1&x&z\\0&1&y\\0&0&1} \in \R^{3 \times 3} \, \middle| \, x,y,z \in \R \, \right\}$$
We define $\cdot$ as the standard matrix multiplication.\\
Is $(\mathcal{G},\cdot)$ a group? If yes, is it Abelian? Justify your answer.
%------------------------------------------------------------------------------------------------------------------------------------------------------------------------
\vsp{3}
\textcolor{blue}{
Answer to come...
}
%>>>>>>>>>>>>>>   2.4  >>>>>>>>>>>>>>>>>>>>>>>>>>>>>>>>>>>>>>>>>>>>>>>>>>>>>>>>>>>>>>>>>>>>>>>>>>>>>>>>>>>>>>>>>>>>>>>>>>>>>>>>>>>>>>>>>>>>>>>>>>>>>>>>>>>>>>>>>>>>>>>>>>>
\subsection{}
Compute the following matrix products, if possible:
\begin{itemize}
 \item [a.] $\M{1&2\\4&5\\7&8} \M{1&1&0\\0&1&1\\1&0&1}$
 \item [b.] $\M{1&2&3\\4&5&6\\7&8&9} \M{1&1&0\\0&1&1\\1&0&1}$
 \item [c.] $\M{1&1&0\\0&1&1\\1&0&1} \M{1&2&3\\4&5&6\\7&8&9}$
 \item [d.] $\M{1&2&1&2\\4&1&-1&-4} \M{0&3\\1&-1\\2&1\\5&2}$
 \item [e.] $\M{0&3\\1&-1\\2&1\\5&2} \M{1&2&1&2\\4&1&-1&-4}$
\end{itemize}
%------------------------------------------------------------------------------------------------------------------------------------------------------------------------
\textcolor{blue}{
\begin{itemize}
 \item [a.]
 \item [b.]
 \item [c.]
 \item [d.]
 \item [e.]
\end{itemize}
}
%>>>>>>>>>>>>>>   2.5  >>>>>>>>>>>>>>>>>>>>>>>>>>>>>>>>>>>>>>>>>>>>>>>>>>>>>>>>>>>>>>>>>>>>>>>>>>>>>>>>>>>>>>>>>>>>>>>>>>>>>>>>>>>>>>>>>>>>>>>>>>>>>>>>>>>>>>>>>>>>>>>>>>>
\subsection{}
Find the set $\mathcal{S}$ of all solutions in $x$ of the following inhomogeneous linear systems $\matA \veca = \vecb$, where $\matA$ and $\vecb$ are defined as follows:
\begin{itemize}
 \item [a.] $$\matA = \M{1&1&-1&-1\\2&5&-7&-5\\2&-1&1&3\\5&2&-4&2}, \vecb = \M{1\\-2\\4\\6}$$
 \item [b.] $$\matA = \M{1&-1&0&0&1\\1&1&0&-3&0\\2&-1&0&1&-1\\-1&2&0&-2&-1}, \vecb = \M{3\\6\\5\\-1}$$
\end{itemize}
%------------------------------------------------------------------------------------------------------------------------------------------------------------------------
\textcolor{blue}{
\begin{itemize}
 \item [a.]
 \item [b.]
\end{itemize}
}
%>>>>>>>>>>>>>>   2.6  >>>>>>>>>>>>>>>>>>>>>>>>>>>>>>>>>>>>>>>>>>>>>>>>>>>>>>>>>>>>>>>>>>>>>>>>>>>>>>>>>>>>>>>>>>>>>>>>>>>>>>>>>>>>>>>>>>>>>>>>>>>>>>>>>>>>>>>>>>>>>>>>>>>
\subsection{}
Using Gaussian elimination, find all solutions of the inhomogeneous equation system $\matA \vecx=\vecb$ with
$$\matA = \M{0&1&0&0&1&0\\0&0&0&1&1&0\\0&1&0&0&0&1}, \vecb = \M{2\\-1\\1}$$
%------------------------------------------------------------------------------------------------------------------------------------------------------------------------
\textcolor{blue}{
Answer to come...
}
%>>>>>>>>>>>>>>   2.7  >>>>>>>>>>>>>>>>>>>>>>>>>>>>>>>>>>>>>>>>>>>>>>>>>>>>>>>>>>>>>>>>>>>>>>>>>>>>>>>>>>>>>>>>>>>>>>>>>>>>>>>>>>>>>>>>>>>>>>>>>>>>>>>>>>>>>>>>>>>>>>>>>>>
\subsection{}
Find all solutions in $\vecx = \M{x_1\\x_2\\x_3} \in \R^3$ of the equation system $\matA \vecx = 12 \vecx$, where
$$\matA = \M{6&4&3\\6&0&9\\0&8&0}$$
and $\sum_{i=1}^3 x_i = 1$.
%------------------------------------------------------------------------------------------------------------------------------------------------------------------------
\textcolor{blue}{
Answer to come...
}
%>>>>>>>>>>>>>>   2.8  >>>>>>>>>>>>>>>>>>>>>>>>>>>>>>>>>>>>>>>>>>>>>>>>>>>>>>>>>>>>>>>>>>>>>>>>>>>>>>>>>>>>>>>>>>>>>>>>>>>>>>>>>>>>>>>>>>>>>>>>>>>>>>>>>>>>>>>>>>>>>>>>>>>
\subsection{}
Determine the inverses of the following matrices if possible:
\begin{itemize}
 \item [a.] $$\matA = \M{2&3&4\\3&4&5\\4&5&6}$$
 \item [b.] $$\matA = \M{1&0&1&0\\0&1&1&0\\1&1&0&1\\1&1&1&0}$$
\end{itemize}
%------------------------------------------------------------------------------------------------------------------------------------------------------------------------
\textcolor{blue}{
\begin{itemize}
 \item [a.]
 \item [b.]
\end{itemize}
}
%>>>>>>>>>>>>>>   2.9  >>>>>>>>>>>>>>>>>>>>>>>>>>>>>>>>>>>>>>>>>>>>>>>>>>>>>>>>>>>>>>>>>>>>>>>>>>>>>>>>>>>>>>>>>>>>>>>>>>>>>>>>>>>>>>>>>>>>>>>>>>>>>>>>>>>>>>>>>>>>>>>>>>>
\subsection{}
Which if the following sets are subspaces of $\R$?
\begin{itemize}
 \item [a.] $A = \Br{(\lambda,\lambda+\mu^3,\lambda-\mu^3) \mid \lambda, \mu \in \R}$
 \item [b.] $B = \Br{(\lambda^2, -\lambda^2,0) \mid \lambda \in \R}$
 \item [c.] Let $\gamma$ be in $\R$. \\ $C = \Br{(\xi_1,\xi_2, \xi_3) \in \R^3 \mid \xi_1-2\xi_2+3\xi_3 = \gamma}$
 \item [d.] $D = \Br{(\xi_1,\xi_2, \xi_3) \in \R^3 \mid \xi_2 \in \Z}$
\end{itemize}
%------------------------------------------------------------------------------------------------------------------------------------------------------------------------
\textcolor{blue}{
\begin{itemize}
 \item [a.]
 \item [b.]
 \item [c.]
 \item [d.]
\end{itemize}
}
%>>>>>>>>>>>>>>  2.10  >>>>>>>>>>>>>>>>>>>>>>>>>>>>>>>>>>>>>>>>>>>>>>>>>>>>>>>>>>>>>>>>>>>>>>>>>>>>>>>>>>>>>>>>>>>>>>>>>>>>>>>>>>>>>>>>>>>>>>>>>>>>>>>>>>>>>>>>>>>>>>>>>>>
\subsection{}
Are the following sets of vectors linearly independent?
\begin{itemize}
 \item [a.] $$\vecx_1 = \M{2\\-1\\3}, \;\; \vecx_2 = \M{1\\1\\-2}, \vecx_3 = \M{3\\-3\\8}$$
 \item [b.] $$\vecx_1 = \M{1\\2\\1\\0\\0}, \;\; \vecx_2 = \M{1\\1\\0\\1\\1}, \vecx_3 = \M{1\\0\\0\\1\\1}$$
\end{itemize}
%------------------------------------------------------------------------------------------------------------------------------------------------------------------------
\textcolor{blue}{
\begin{itemize}
 \item [a.]
 \item [b.]
\end{itemize}
}
%>>>>>>>>>>>>>>  2.11  >>>>>>>>>>>>>>>>>>>>>>>>>>>>>>>>>>>>>>>>>>>>>>>>>>>>>>>>>>>>>>>>>>>>>>>>>>>>>>>>>>>>>>>>>>>>>>>>>>>>>>>>>>>>>>>>>>>>>>>>>>>>>>>>>>>>>>>>>>>>>>>>>>>
\subsection{}
Write $$\vecy=\M{1\\-2\\5}$$ as a linear combination of $$\vecx_1 = \M{1\\1\\1}, \;\; \vecx_2 = \M{1\\2\\3}, \vecx_3 = \M{2\\-1\\1}$$
%------------------------------------------------------------------------------------------------------------------------------------------------------------------------
\textcolor{blue}{
Answer to come...
}
%>>>>>>>>>>>>>>  2.12  >>>>>>>>>>>>>>>>>>>>>>>>>>>>>>>>>>>>>>>>>>>>>>>>>>>>>>>>>>>>>>>>>>>>>>>>>>>>>>>>>>>>>>>>>>>>>>>>>>>>>>>>>>>>>>>>>>>>>>>>>>>>>>>>>>>>>>>>>>>>>>>>>>>
\subsection{}
Consider two subspaces of $\R^4$:
$$
U_1 = \textrm{span}\left( \M{1\\1\\-3\\1}, \M{2\\-1\\0\\-1}, \M{-1\\1\\-1\\1} \right), \;\;\;\;
U_2 = \textrm{span}\left( \M{-1\\-2\\2\\1}, \M{2\\-2\\0\\0}, \M{-3\\6\\-2\\-1} \right)
$$
Determine a basis of $U_1 \cap U_2$.
%------------------------------------------------------------------------------------------------------------------------------------------------------------------------
\textcolor{blue}{
Answer to come...
}
%>>>>>>>>>>>>>>  2.13  >>>>>>>>>>>>>>>>>>>>>>>>>>>>>>>>>>>>>>>>>>>>>>>>>>>>>>>>>>>>>>>>>>>>>>>>>>>>>>>>>>>>>>>>>>>>>>>>>>>>>>>>>>>>>>>>>>>>>>>>>>>>>>>>>>>>>>>>>>>>>>>>>>>
\subsection{}
Consider two subspaces $U_1$ and $U_2$, where $U_1$ is the solution space of the homogeneous equation system $\matA_1 \vecx = \veco$ and $U_2$ is the solution space of the homogeneous equation system $\matA_2 \vecx = \veco$ with
$$\matA_1 = \M{1&0&1\\1&-2&-1\\2&1&3\\1&0&1}, \;\;\;\; \matA_2 = \M{3&-3&0\\1&2&3\\7&-5&2\\3&-1&2}$$
\begin{itemize}
 \item [a.] Determine the dimension of $U_1,U_2$.
 \item [b.] Determine bases of $U_1$ and $U_2$.
 \item [c.] Determine a basis of $U_1 \cap U_2$.
\end{itemize}
%------------------------------------------------------------------------------------------------------------------------------------------------------------------------
\textcolor{blue}{
\begin{itemize}
 \item [a.]
 \item [b.]
 \item [c.]
\end{itemize}
}
%>>>>>>>>>>>>>>  2.14  >>>>>>>>>>>>>>>>>>>>>>>>>>>>>>>>>>>>>>>>>>>>>>>>>>>>>>>>>>>>>>>>>>>>>>>>>>>>>>>>>>>>>>>>>>>>>>>>>>>>>>>>>>>>>>>>>>>>>>>>>>>>>>>>>>>>>>>>>>>>>>>>>>>
\subsection{}
Consider two subspaces $U_1$ and $U_2$, where $U_1$ is spanned by the columns of $\matA_1$ and $U_2$ is spanned by the columns of $\matA_2$ with
$$\matA_1 = \M{1&0&1\\1&-2&-1\\2&1&3\\1&0&1}, \;\;\;\; \matA_2 = \M{3&-3&0\\1&2&3\\7&-5&2\\3&-1&2}$$

\begin{itemize}
 \item [a.] Determine the dimension of $U_1,U_2$.
 \item [b.] Determine bases of $U_1$ and $U_2$.
 \item [c.] Determine a basis of $U_1 \cap U_2$.
\end{itemize}
%------------------------------------------------------------------------------------------------------------------------------------------------------------------------
\textcolor{blue}{
\begin{itemize}
 \item [a.] Row reduction of $\matA_1$: \vsp{2}
$\M{1&0&1\\1&-2&-1\\2&1&3\\1&0&1}
\rightarrow
\M{R_1\\R_2-R_1\\R_3-2R_1\\R_4-R_1}
=
\M{1&0&1\\0&-2&-2\\0&1&1\\0&0&0}
\rightarrow
\M{R_1\\-\frac{1}{2}R_2\\R_3+\frac{1}{2}R_2\\R_4}
=
\M{1&0&1\\0&1&1\\0&0&0\\0&0&0}
$
\vsp{2}                $\Rightarrow \dim(U_1) = 2$ because $\matA_1$ is row equivalent to a matrix with two pivots.
\vsp{3}
Row reduction of $\matA_2$: \vsp{2}
$\M{3&-3&0\\1&2&3\\7&-5&2\\3&-1&2}
\rightarrow
\M{\frac{1}{3}R_1\\R_2-\frac{1}{3}R_1\\R_3-\frac{7}{3}R_1\\R_4-R_1}
=
\M{1&-1&0\\0&3&3\\0&2&2\\0&-2&2}
\rightarrow
\M{R_1\\\frac{1}{3}R_2\\R_4+R_3\\R_3-\frac{2}{3}R_2}
=
\M{1&0&1\\0&1&1\\0&0&4\\0&0&0}
\vsp{3}
\rightarrow
\M{R_1-\frac{1}{4}R_3\\R_2-\frac{1}{4}R_3\\\frac{1}{4}R_3\\R_4}
=
\M{1&0&0\\0&1&0\\0&0&1\\0&0&0}
$
\vsp{2}                $\Rightarrow \dim(U_2) = 3$ because $\matA_2$ is row equivalent to a matrix with three pivots.
 \item [b.] We can use the pivot columns of the RREF matrices calculated in part (a) as bases for $U_1$ and $U_2$, respectively.
 \item [c.] Since $U_2$ is a 3-dimensional subspace of $\R^4$ and $U_1$ is only a 2-dimensional subspace of $\R^4$, $U_1$ is also a subspace of $U_2$.
 This means that $U_1 \cap U_2 = U_1$, so we can use the basis for $U_1$ that we found in part (b).
\end{itemize}
}
%>>>>>>>>>>>>>>  2.15  >>>>>>>>>>>>>>>>>>>>>>>>>>>>>>>>>>>>>>>>>>>>>>>>>>>>>>>>>>>>>>>>>>>>>>>>>>>>>>>>>>>>>>>>>>>>>>>>>>>>>>>>>>>>>>>>>>>>>>>>>>>>>>>>>>>>>>>>>>>>>>>>>>>
\subsection{}
Let $F = \Br{(xy,z) \in R^3 \mid x+y-z=0}$ and $G = \Br{(a-b,a+b,a-3b) \mid a,b \in \R}$
\begin{itemize}
 \item [a.] Show that $F$ and $G$ are subspaces of $\R^3$.
 \item [b.] Calculate $F \cap G$ without resorting to any basis vector.
 \item [c.] Find one basis for $F$ and $G$, calculate $F \cap G$ using the basis vectors previously found and check your result with the previous question.
\end{itemize}
%------------------------------------------------------------------------------------------------------------------------------------------------------------------------
\textcolor{blue}{
\begin{itemize}
 \item [a.]
 \item [b.]
 \item [c.]
\end{itemize}
}
%>>>>>>>>>>>>>>  2.16  >>>>>>>>>>>>>>>>>>>>>>>>>>>>>>>>>>>>>>>>>>>>>>>>>>>>>>>>>>>>>>>>>>>>>>>>>>>>>>>>>>>>>>>>>>>>>>>>>>>>>>>>>>>>>>>>>>>>>>>>>>>>>>>>>>>>>>>>>>>>>>>>>>>
\subsection{}
Are the following mappings linear?
\begin{itemize}
 \item [a.] Let $a,b, \in \R$.
                \begin{align*}
                \Phi: L^1([a,b]) &\rightarrow \R \\
                f &\mapsto \Phi(f) = \int_a^b f(x)dx,
                \end{align*}
                where $L^1([a,b])$ denotes the set of integrable functions on $[a,b]$.
 \item [b.] \begin{align*}
                \Phi: L^1: C^1 &\rightarrow C^0\\
                f &\mapsto \Phi(f) = f\pr
                \end{align*}
                where for $k. \geq 1$, $C^k$ denotes the set of $k$ times continuously differentiable functions, and $C^0$ denotes the set of continuous functions.
 \item [c.] \begin{align*}
                \Phi: \R &\rightarrow \R \\
                x &\mapsto \Phi(x) = \cos(x)
                \end{align*}
 \item [d.] \begin{align*}
                \Phi: \R^3 &\rightarrow \R^2 \\
                x &\mapsto \M{1&2&3\\1&4&3}\vecx
                \end{align*}
 \item [e.] Let $\theta$ be in $[0,2\pi[$ and
                \begin{align*}
                \Phi: \R^2 &\rightarrow \R^2 \\
                \vecx &\mapsto \M{\cos(\theta)&\sin(\theta)\\-\sin(\theta)&\cos(\theta)} \vecx
                \end{align*}
\end{itemize}
%------------------------------------------------------------------------------------------------------------------------------------------------------------------------
\textcolor{blue}{
\begin{itemize}
 \item [a.]
 \item [b.]
 \item [c.]
 \item [d.]
 \item [e.]
\end{itemize}
}
%>>>>>>>>>>>>>>  2.17  >>>>>>>>>>>>>>>>>>>>>>>>>>>>>>>>>>>>>>>>>>>>>>>>>>>>>>>>>>>>>>>>>>>>>>>>>>>>>>>>>>>>>>>>>>>>>>>>>>>>>>>>>>>>>>>>>>>>>>>>>>>>>>>>>>>>>>>>>>>>>>>>>>>
\subsection{}
Consider the linear mapping
\begin{align*}
\Phi: \R^3 \rightarrow \R^4 \\
\Phi \left(\M{x_1\\x_2\\x_3}\right) = \M{3x_1+2x_2+x_3\\x_1+x_2+x_3\\x_1-3x_2\\2x_1+3x_2+x_3}
\end{align*}
\begin{itemize}
 \item Find the transformation matrix $\matA_\Phi$.
 \item Determine rk$(\matA_\Phi)$.
 \item Compute the kernel and image of $\Phi$. What are $\dim(\ker(\Phi))$ and $\dim(\image(\Phi))$?
\end{itemize}
%------------------------------------------------------------------------------------------------------------------------------------------------------------------------
\textcolor{blue}{
\begin{itemize}
 \item
 \item
 \item
\end{itemize}
}
%>>>>>>>>>>>>>>  2.18  >>>>>>>>>>>>>>>>>>>>>>>>>>>>>>>>>>>>>>>>>>>>>>>>>>>>>>>>>>>>>>>>>>>>>>>>>>>>>>>>>>>>>>>>>>>>>>>>>>>>>>>>>>>>>>>>>>>>>>>>>>>>>>>>>>>>>>>>>>>>>>>>>>>
\subsection{}
Let $E$ be a vector space. Let $f$ and $g$ be two automorphisms on $E$ such that $f \circ g = \textrm{id}_E$ (i.e. $f \circ g$ is the identity mapping $\textrm{id}_E$). Show that $\ker(f) = ker(g \circ f), \, \image(g) = \image(g \circ f)$ and that $\ker(f) \cap \image(g) = \Br{\veco_E}$.
%------------------------------------------------------------------------------------------------------------------------------------------------------------------------
\vsp{3}
\textcolor{blue}{
Answer to come...
}
%>>>>>>>>>>>>>>  2.19  >>>>>>>>>>>>>>>>>>>>>>>>>>>>>>>>>>>>>>>>>>>>>>>>>>>>>>>>>>>>>>>>>>>>>>>>>>>>>>>>>>>>>>>>>>>>>>>>>>>>>>>>>>>>>>>>>>>>>>>>>>>>>>>>>>>>>>>>>>>>>>>>>>>
\subsection{}
Consider and endomorphism $\Phi: \R^3 \rightarrow \R^3$ whose transformation matrix (with respect to the standard basis in $\R ^3$) is
$$\matA_\Phi = \M{1&1&0\\1&-1&0\\1&1&1}$$
\begin{itemize}
 \item [a.] Determine $\ker(\Phi)$ and $\image(\Phi)$.
 \item [b.] Determine the transformation matrix $\tilde{\matA}_\Phi$ with respect to the basis
 $$B = \left( \M{1\\1\\1}, \M{1\\2\\1}, \M{1\\0\\0} \right),$$
 i.e., perform a basis change toward the new basis $B$.
\end{itemize}
%------------------------------------------------------------------------------------------------------------------------------------------------------------------------
\textcolor{blue}{
\begin{itemize}
 \item [a.]
 \item [b.]
\end{itemize}
}
%>>>>>>>>>>>>>>  2.20  >>>>>>>>>>>>>>>>>>>>>>>>>>>>>>>>>>>>>>>>>>>>>>>>>>>>>>>>>>>>>>>>>>>>>>>>>>>>>>>>>>>>>>>>>>>>>>>>>>>>>>>>>>>>>>>>>>>>>>>>>>>>>>>>>>>>>>>>>>>>>>>>>>>
\subsection{}
Let us consider $\vecb_1,\vecb_2, \vecb_1\pr, \vecb_2\pr$, 4 vectors of $\R^2$ expressed in the standard basis of $\R^2$ as
$$\vecb_1 = \M{2\\1}, \;\; \vecb_2 = \M{-1\\1}, \;\; \vecb_1\pr = \M{2\\-2}, \;\; \vecb_2\pr = \M{1\\1}$$
and let us define two ordered bases $B = (\vecb_1,\vecb_2)$ and $B\pr = (\vecb_1\pr,\vecb_2\pr)$ of $\R^2$.
\begin{itemize}
 \item [a.] Show that $B$ and $B\pr$ are two bases of $\R^2$ and draw those basis vectors.
 \item [b.] Compute the matrix $\textbf{P}_1$ that performs a basis change from $B\pr$ to $B$.
 \item [c.] We consider $\vecc_1, \vecc_2, \vecc_3$, three vectors of $\R^3$ defined in the standard basis of $\R^3$ as
 $$\vecc_1 = \M{1\\2\\-1}, \;\; \vecc_2 = \M{0\\-1\\2}, \;\; \vecc_3 = \M{1\\0\\-1}$$
 and we define $C = (\vecc_1, \vecc_2, \vecc_3)$.
 \begin{enumerate}
  \item [(i)] Show that $C$ is a basis of $\R^3$, e.g. by using determinants (see section 4.1).
  \item [(ii)] Let us call $C\pr = (\vecc_1\pr,\vecc_2\pr,\vecc_3\pr)$ the standard basis of $\R^3$. Determine the matrix $\textbf{P}_2$ that performs the basis change from $C$ to $C\pr$.
 \end{enumerate}
 \item [d.] We consider a homomorphism $\Phi: \R^2 \rightarrow \R^3$, such that
 \begin{align*}
  \Phi(\vecb_1 + \vecb_2) &= \vecc_2+\vecc_3 \\
  \Phi(\vecb_1 - \vecb_2) &= \vecc_1-\vecc_2+\vecc_3
 \end{align*}
 where $B = (\vecb_1,\vecb_2)$ and $C = (\vecc_1,\vecc_2,\vecc_3)$ are ordered bases of $\R^2$ and $\R^3$, respectively. \\
 Determine the trnsormation matrix $\matA_\Phi$ of $\Phi$ with respect to the ordered bases $B$ and $C$.
 \item [e.]Determine $\matA\pr$, the transformation matrix of $\phi$ with respect to the bases $\textbf{B}\pr$ and $\textbf{C}\pr$.
 \item [f.] Let us consider the vector $x \in \R^2$ whose coordinates in $B\pr$ are $[2,3]^\top$. In other words, $\vecx = 2\vecb_1\pr +3\vecb_2\pr$.
 \begin{enumerate}
  \item [(i)] Calculate the coordinates of $\vecx$ in $B$.
  \item [(ii)] Based on that, compute the coordinates of $\Phi(\vecx)$ expressed in $C$.
  \item [(iii)] Then, write $\Phi(\vecx)$ in terms of $\vecc_1\pr, \vecc_2\pr, \vecc_3\pr$.
  \item [(iv)] Use the representation of $\vecx$ in $B\pr$ and the matrix $\matA\pr$ to find this result directly.
 \end{enumerate}
\end{itemize}
%------------------------------------------------------------------------------------------------------------------------------------------------------------------------
\textcolor{blue}{
\begin{itemize}
 \item [a.]
 \item [b.]
 \item [c.]
 \begin{enumerate}
  \item [(i)]
  \item [(ii)]
 \end{enumerate}
 \item [d.]
 \item [e.]
 \item[f.]
 \begin{enumerate}
  \item [(i)]
  \item [(ii)]
  \item [(iii)]
  \item [(iv)]
 \end{enumerate}
\end{itemize}
}

\newpage%%%%%%%%%%%%%%%%%%%%%%%%%%%%%%%%%%%%%%%%%%%%%%%%%%%%%%%%%%%%%%%
\section{Analytic Geometry}
%>>>>>>>>>>>>>>   3.1  >>>>>>>>>>>>>>>>>>>>>>>>>>>>>>>>>>>>>>>>>>>>>>>>>>>>>>>>>>>>>>>>>>>>>>>>>>>>>>>>>>>>>>>>>>>>>>>>>>>>>>>>>>>>>>>>>>>>>>>>>>>>>>>>>>>>>>>>>>>>>>>>>>>
\subsection{}
Show that $\left< \cdot,\cdot \right>$ defined for all $\vecx = [x_1, x_2]^\top \in R^2$ and $\vecy = [y_1,y_2]^top \in \R^2$ by
$$\left< \vecx, \vecy \right> := x_1 y_1 - (x_1y_2 + x_2y_1) + 2(x_2y_2)$$
is an inner product.
%------------------------------------------------------------------------------------------------------------------------------------------------------------------------
\vsp{3}
\textcolor{blue}{
Answer to come...
}
%>>>>>>>>>>>>>>   3.2  >>>>>>>>>>>>>>>>>>>>>>>>>>>>>>>>>>>>>>>>>>>>>>>>>>>>>>>>>>>>>>>>>>>>>>>>>>>>>>>>>>>>>>>>>>>>>>>>>>>>>>>>>>>>>>>>>>>>>>>>>>>>>>>>>>>>>>>>>>>>>>>>>>>
\subsection{}
Consider $\R^2$ with $\left< \vecx, \vecy \right>$ defined for all $\vecx$ and $\vecy$ in $\R^2$ as
$$\left< \vecx, \vecy \right> := \vecx^\top \Ubr{\M{2&0\\1&2}}{=: \matA} \vecy$$
Is $\left< \cdot, \cdot \right>$ an inner product?
%------------------------------------------------------------------------------------------------------------------------------------------------------------------------
\vsp{3}
\textcolor{blue}{
Answer to come...
}
%>>>>>>>>>>>>>>   3.3  >>>>>>>>>>>>>>>>>>>>>>>>>>>>>>>>>>>>>>>>>>>>>>>>>>>>>>>>>>>>>>>>>>>>>>>>>>>>>>>>>>>>>>>>>>>>>>>>>>>>>>>>>>>>>>>>>>>>>>>>>>>>>>>>>>>>>>>>>>>>>>>>>>>
\subsection{}
Compute the distance between
$$\vecx=\M{1\\2\\3}, \;\; \vecy = \M{-1\\-1\\0}$$
using
\begin{itemize}
 \item [a.] $\left< \vecx, \vecy \right> := \vecx^\top \vecy$
 \item [b.] $\left< \vecx, \vecy \right> := \vecx^\top \matA \vecy, \matA := \M{2&1&0\\1&3&-1\\0&-1&2}$
\end{itemize}
%------------------------------------------------------------------------------------------------------------------------------------------------------------------------
\textcolor{blue}{
\begin{itemize}
 \item [a.]
 \item [b.]
\end{itemize}
}
%>>>>>>>>>>>>>>   3.4  >>>>>>>>>>>>>>>>>>>>>>>>>>>>>>>>>>>>>>>>>>>>>>>>>>>>>>>>>>>>>>>>>>>>>>>>>>>>>>>>>>>>>>>>>>>>>>>>>>>>>>>>>>>>>>>>>>>>>>>>>>>>>>>>>>>>>>>>>>>>>>>>>>>
\subsection{}
Compute the angle between
$$\vecx=\M{1\\2}, \;\; \vecy=\M{-1\\-1}$$
using
\begin{itemize}
 \item [a.] $\left< \vecx, \vecy \right> := \vecx^\top \vecy$
 \item [b.] $\left< \vecx, \vecy \right> := \vecx^\top \matB \vecy, \; \matB := \M{2&1\\1&3}$
\end{itemize}
%------------------------------------------------------------------------------------------------------------------------------------------------------------------------
\textcolor{blue}{
\begin{itemize}
 \item [a.]
 \item [b.]
\end{itemize}
}
%>>>>>>>>>>>>>>   3.5  >>>>>>>>>>>>>>>>>>>>>>>>>>>>>>>>>>>>>>>>>>>>>>>>>>>>>>>>>>>>>>>>>>>>>>>>>>>>>>>>>>>>>>>>>>>>>>>>>>>>>>>>>>>>>>>>>>>>>>>>>>>>>>>>>>>>>>>>>>>>>>>>>>>
\subsection{}
Consider the Euclidean vector space $\R^5$ with the dot product. A subspace $U \subseteq \R^5$ and $\vecx \in \R^5$ are given by
$$U = \textrm{span} \left( \M{0\\-1\\2\\0\\2},
                                         \M{1\\-3\\1\\-1\\2},
                                         \M{-3\\4\\1\\2\\1},
                                         \M{-1\\-3\\5\\0\\7}
                                 \right), \vecx =
                                         \M{-1\\-9\\-1\\4\\1}$$
\begin{itemize}
 \item [a.] Determine the orthogonal projection $\pi_U(\vecx)$ of $\vecx$ onto $U$.
 \item [b.] Determine the distance $d(\vecx,U)$.
\end{itemize}
%------------------------------------------------------------------------------------------------------------------------------------------------------------------------
\textcolor{blue}{
\begin{itemize}
 \item [a.]
 \item [b.]
\end{itemize}
}
%>>>>>>>>>>>>>>   3.6  >>>>>>>>>>>>>>>>>>>>>>>>>>>>>>>>>>>>>>>>>>>>>>>>>>>>>>>>>>>>>>>>>>>>>>>>>>>>>>>>>>>>>>>>>>>>>>>>>>>>>>>>>>>>>>>>>>>>>>>>>>>>>>>>>>>>>>>>>>>>>>>>>>>
\subsection{}
Consider $\R^3$ with the inner product
$$\left< \vecx, \vecy \right> := \vecx^\top \M{2&1&0\\1&2&-1\\0&-1&2} \vecy.$$
Furthermore, we define $\vece_1, \vece_2, \vece_3$ as the standard/canonical basis in $\R^3$.
\begin{itemize}
 \item [a.] Determine the orthogonal projection $\pi_U(\vece_2)$ of $\vece_2$ onto
 $$U = \textrm{span}[\vece_1,\vece_3].$$
 Hint: Orthogonality is defined through the inner product.
 \item [b.] Compute the distance $d(\vece_2, U)$.
 \item [c.]Draw the scenario: standard basis vectors and $\pi_U(\vece_2)$.
\end{itemize}
%------------------------------------------------------------------------------------------------------------------------------------------------------------------------
\textcolor{blue}{
\begin{itemize}
 \item [a.]
 \item [b.]
 \item [c.]
\end{itemize}
}
%>>>>>>>>>>>>>>   3.7  >>>>>>>>>>>>>>>>>>>>>>>>>>>>>>>>>>>>>>>>>>>>>>>>>>>>>>>>>>>>>>>>>>>>>>>>>>>>>>>>>>>>>>>>>>>>>>>>>>>>>>>>>>>>>>>>>>>>>>>>>>>>>>>>>>>>>>>>>>>>>>>>>>>
\subsection{}
Let $V$ be a vector space and $\pi$ an endomorphism of $V$.
\begin{itemize}
 \item [a.] Prove that $\pi$ is a projection if and only if $\id_V-\pi$ is a projection, where $\id_V$ is the identity endomorphism on $V$.
 \item [b.] Assume now that $\pi$ is a projection. Calculate $\image(\id_V-\pi)$ and $\ker(\id_V-\pi)$ as a function of $\image(\pi)$ and $\ker(\pi)$.
\end{itemize}
%------------------------------------------------------------------------------------------------------------------------------------------------------------------------
\textcolor{blue}{
\begin{itemize}
 \item [a.]
 \item [b.]
\end{itemize}
}
%>>>>>>>>>>>>>>   3.8  >>>>>>>>>>>>>>>>>>>>>>>>>>>>>>>>>>>>>>>>>>>>>>>>>>>>>>>>>>>>>>>>>>>>>>>>>>>>>>>>>>>>>>>>>>>>>>>>>>>>>>>>>>>>>>>>>>>>>>>>>>>>>>>>>>>>>>>>>>>>>>>>>>>
\subsection{}
Using the Gram-Schmidt method, turn the basis $B = (\vecb_1, \vecb_2)$ of a two-dimensional subspace $U \subseteq \R^3$ into an ONB $C = (\vecc_1, \vecc_2)$ of $U$, where
$$\vecb_1 := \M{1\\1\\1\\}, \;\; \vecb_2 := \M{-1\\2\\0}.$$
%------------------------------------------------------------------------------------------------------------------------------------------------------------------------
\textcolor{blue}{
Answer to come...
}
%>>>>>>>>>>>>>>   3.9  >>>>>>>>>>>>>>>>>>>>>>>>>>>>>>>>>>>>>>>>>>>>>>>>>>>>>>>>>>>>>>>>>>>>>>>>>>>>>>>>>>>>>>>>>>>>>>>>>>>>>>>>>>>>>>>>>>>>>>>>>>>>>>>>>>>>>>>>>>>>>>>>>>>
\subsection{}
Let $n \in \N$ and let $x_1, ... , x_n > 0$ be $n$ positive real numbers so that $x_1+...+x_n=1$. Use the Cauchy-Schwarz inequality and show that
\begin{itemize}
 \item [a.] $\stsum x_i^2 \geq \frac{1}{n}$
 \item [b.] $\stsum \frac{1}{x_i} \geq n^2$
 Hint: Think about the dot product on $\R^n$. Then, choose specific vectors $\vecx, \vecy \in \R^n$ and apply the Cauchy-Schwarz inequality.
\end{itemize}
%------------------------------------------------------------------------------------------------------------------------------------------------------------------------
\textcolor{blue}{
\begin{itemize}
 \item [a.]
 \item [b.]
\end{itemize}
}
%>>>>>>>>>>>>>>  3.10  >>>>>>>>>>>>>>>>>>>>>>>>>>>>>>>>>>>>>>>>>>>>>>>>>>>>>>>>>>>>>>>>>>>>>>>>>>>>>>>>>>>>>>>>>>>>>>>>>>>>>>>>>>>>>>>>>>>>>>>>>>>>>>>>>>>>>>>>>>>>>>>>>>>
\subsection{}
Rotate the vectors
$$\vecx_1 = \M{2\\3}, \;\; \vecx_2 := \M{0\\-1}$$
%------------------------------------------------------------------------------------------------------------------------------------------------------------------------
\textcolor{blue}{
Answer to come...
}
\newpage%%%%%%%%%%%%%%%%%%%%%%%%%%%%%%%%%%%%%%%%%%%%%%%%%%%%%%%%%%%%%%%
\section{Matrix Decompositions}
%>>>>>>>>>>>>>>  4.1  >>>>>>>>>>>>>>>>>>>>>>>>>>>>>>>>>>>>>>>>>>>>>>>>>>>>>>>>>>>>>>>>>>>>>>>>>>>>>>>>>>>>>>>>>>>>>>>>>>>>>>>>>>>>>>>>>>>>>>>>>>>>>>>>>>>>>>>>>>>>>>>>>>>
\subsection{}
Computethe determinant using the Laplace expansion (using the first row) and the Sarrus rule for
$$\matA = \M{1&3&5\\2&4&6\\0&2&4}.$$
%------------------------------------------------------------------------------------------------------------------------------------------------------------------------
\textcolor{blue}{
Answer to come...
}
%>>>>>>>>>>>>>>   4.2  >>>>>>>>>>>>>>>>>>>>>>>>>>>>>>>>>>>>>>>>>>>>>>>>>>>>>>>>>>>>>>>>>>>>>>>>>>>>>>>>>>>>>>>>>>>>>>>>>>>>>>>>>>>>>>>>>>>>>>>>>>>>>>>>>>>>>>>>>>>>>>>>>>>
\subsection{}
Compute the following determinant efficiently:
$$\M{2 &0&1&2&0\\
         2 &-1&0&1&1\\
         0 &1&2&1&2\\
         -2&0&2&-1&2\\
         2 &0&0&1&1}$$
%------------------------------------------------------------------------------------------------------------------------------------------------------------------------
\textcolor{blue}{
Answer to come...
}
%>>>>>>>>>>>>>>   4.3  >>>>>>>>>>>>>>>>>>>>>>>>>>>>>>>>>>>>>>>>>>>>>>>>>>>>>>>>>>>>>>>>>>>>>>>>>>>>>>>>>>>>>>>>>>>>>>>>>>>>>>>>>>>>>>>>>>>>>>>>>>>>>>>>>>>>>>>>>>>>>>>>>>>
\subsection{}
Compute the eigenspaces of
\begin{itemize}
 \item [a.] $$\matA := \M{1&0\\1&1}$$
 \item [b.] $$\matB := \M{-2&2\\2&1}$$
\end{itemize}
%------------------------------------------------------------------------------------------------------------------------------------------------------------------------
\textcolor{blue}{
\begin{itemize}
 \item [a.]
 \item [b.]
\end{itemize}
}
%>>>>>>>>>>>>>>   4.4  >>>>>>>>>>>>>>>>>>>>>>>>>>>>>>>>>>>>>>>>>>>>>>>>>>>>>>>>>>>>>>>>>>>>>>>>>>>>>>>>>>>>>>>>>>>>>>>>>>>>>>>>>>>>>>>>>>>>>>>>>>>>>>>>>>>>>>>>>>>>>>>>>>>
\subsection{}
Compute all eigenspaces of
$$\matA = \M{0&-1&1&1\\
                       -1&1&-2&3\\
                       2&-1&0&0\\
                       1&-1&1&0}.$$
%------------------------------------------------------------------------------------------------------------------------------------------------------------------------
\textcolor{blue}{
Answer to come...
}
%>>>>>>>>>>>>>>   4.5  >>>>>>>>>>>>>>>>>>>>>>>>>>>>>>>>>>>>>>>>>>>>>>>>>>>>>>>>>>>>>>>>>>>>>>>>>>>>>>>>>>>>>>>>>>>>>>>>>>>>>>>>>>>>>>>>>>>>>>>>>>>>>>>>>>>>>>>>>>>>>>>>>>>
\subsection{}
Diagonalizability of a matrix is unrelated to it invertibility. Determine for the following four matrices whether they are diagonalizable and/or invertible
$$\M{1&0\\0&1}, \;\; \M{1&0\\0&0}, \;\; \M{1&1\\0&1}, \;\; \M{0&1\\0&0}.$$
%------------------------------------------------------------------------------------------------------------------------------------------------------------------------
\textcolor{blue}{
Answer to come...
}
%>>>>>>>>>>>>>>   4.6  >>>>>>>>>>>>>>>>>>>>>>>>>>>>>>>>>>>>>>>>>>>>>>>>>>>>>>>>>>>>>>>>>>>>>>>>>>>>>>>>>>>>>>>>>>>>>>>>>>>>>>>>>>>>>>>>>>>>>>>>>>>>>>>>>>>>>>>>>>>>>>>>>>>
\subsection{}
Compute the eigenspaces of the following transformation matrices. Are they diagonalizable?
\begin{itemize}
 \item [a.] For $$\matA = \M{2&3&0\\1&4&3\\0&0&1}$$
 \item [b.] For $$\matA = \M{1&1&0&0\\0&0&0&0\\0&0&0&0\\0&0&0&0}$$
\end{itemize}
%------------------------------------------------------------------------------------------------------------------------------------------------------------------------
\textcolor{blue}{
\begin{itemize}
 \item [a.]
 \item [b.]
\end{itemize}
}
%>>>>>>>>>>>>>>   4.7  >>>>>>>>>>>>>>>>>>>>>>>>>>>>>>>>>>>>>>>>>>>>>>>>>>>>>>>>>>>>>>>>>>>>>>>>>>>>>>>>>>>>>>>>>>>>>>>>>>>>>>>>>>>>>>>>>>>>>>>>>>>>>>>>>>>>>>>>>>>>>>>>>>>
\subsection{}
Are the following matrices diagonalizable? If yes, determine their diagonal form and give a basis with respect to which the transformation matrices are diagonal.
If no, give reasons why they are not diagonalizable.
\begin{itemize}
 \item [a.] $$\matA = \M{ 0&1\\
                                     -8&4}$$
 \item [b.] $$\matA = \M{1&1&1\\
                                     1&1&1\\
                                     1&1&1}$$
 \item [c.] $$\matA = \M{5&4&2&1\\
                                     0&1&-1&-1\\
                                    -1&-1&3&1\\
                                    1&1&-1&2}$$
 \item [d.] $$\matA = \M{5&-6&-6\\
                                    -1&4&2\\
                                    3&-6&-4}$$
\end{itemize}
%------------------------------------------------------------------------------------------------------------------------------------------------------------------------
\textcolor{blue}{
\begin{itemize}
 \item [a.]
 \item [b.]
 \item [c.]
 \item [d.]
\end{itemize}
}
%>>>>>>>>>>>>>>   4.8  >>>>>>>>>>>>>>>>>>>>>>>>>>>>>>>>>>>>>>>>>>>>>>>>>>>>>>>>>>>>>>>>>>>>>>>>>>>>>>>>>>>>>>>>>>>>>>>>>>>>>>>>>>>>>>>>>>>>>>>>>>>>>>>>>>>>>>>>>>>>>>>>>>>
\subsection{}
Find the SVD of the matrix
$$\matA = \M{3&2&2\\2&3&-2}$$
%------------------------------------------------------------------------------------------------------------------------------------------------------------------------
\textcolor{blue}{
Answer to come...
}
%>>>>>>>>>>>>>>   4.9  >>>>>>>>>>>>>>>>>>>>>>>>>>>>>>>>>>>>>>>>>>>>>>>>>>>>>>>>>>>>>>>>>>>>>>>>>>>>>>>>>>>>>>>>>>>>>>>>>>>>>>>>>>>>>>>>>>>>>>>>>>>>>>>>>>>>>>>>>>>>>>>>>>>
\subsection{}
Find the singular value decomposition of
$$\matA = \M{2&2\\-1&1}$$
%------------------------------------------------------------------------------------------------------------------------------------------------------------------------
\textcolor{blue}{
Answer to come...
}
%>>>>>>>>>>>>>>  4.10  >>>>>>>>>>>>>>>>>>>>>>>>>>>>>>>>>>>>>>>>>>>>>>>>>>>>>>>>>>>>>>>>>>>>>>>>>>>>>>>>>>>>>>>>>>>>>>>>>>>>>>>>>>>>>>>>>>>>>>>>>>>>>>>>>>>>>>>>>>>>>>>>>>>
\subsection{}
Find the rank-1 approximation of
$$\matA = \M{3&2&2\\2&3&-2}$$
%------------------------------------------------------------------------------------------------------------------------------------------------------------------------
\textcolor{blue}{
Answer to come...
}
%>>>>>>>>>>>>>>  4.11  >>>>>>>>>>>>>>>>>>>>>>>>>>>>>>>>>>>>>>>>>>>>>>>>>>>>>>>>>>>>>>>>>>>>>>>>>>>>>>>>>>>>>>>>>>>>>>>>>>>>>>>>>>>>>>>>>>>>>>>>>>>>>>>>>>>>>>>>>>>>>>>>>>>
\subsection{}
Show that for any $\matA \in \R^{m \times n}$ the matrices $\matA^top \matA$ and $\matA \matA^\top$ possess the same nonzero eigenvalues.
%------------------------------------------------------------------------------------------------------------------------------------------------------------------------
\textcolor{blue}{
Answer to come...
}
%>>>>>>>>>>>>>>  4.12  >>>>>>>>>>>>>>>>>>>>>>>>>>>>>>>>>>>>>>>>>>>>>>>>>>>>>>>>>>>>>>>>>>>>>>>>>>>>>>>>>>>>>>>>>>>>>>>>>>>>>>>>>>>>>>>>>>>>>>>>>>>>>>>>>>>>>>>>>>>>>>>>>>>
\subsection{}
Show that for $\vecx \neq \veco$ Theorem 4.24 holds, i.e., show that
$$\max\limits_x \dfrac{||\matA \vecx ||_2}{||\vecx||_2} = \sigma_1$$
where $\sigma_1$ is the largest singular value of $\matA \in \R^{m \times n}$.
%------------------------------------------------------------------------------------------------------------------------------------------------------------------------
\textcolor{blue}{
Answer to come...
}
\newpage%%%%%%%%%%%%%%%%%%%%%%%%%%%%%%%%%%%%%%%%%%%%%%%%%%%%%%%%%%%%%%%
\section{Vector Calculus}
%>>>>>>>>>>>>>>   5.1  >>>>>>>>>>>>>>>>>>>>>>>>>>>>>>>>>>>>>>>>>>>>>>>>>>>>>>>>>>>>>>>>>>>>>>>>>>>>>>>>>>>>>>>>>>>>>>>>>>>>>>>>>>>>>>>>>>>>>>>>>>>>>>>>>>>>>>>>>>>>>>>>>>>
\subsection{}
Compute the derivative $f\pr(x)$ for
$$f(x) = \log(x^4) \sin(x^3)$$
%------------------------------------------------------------------------------------------------------------------------------------------------------------------------
\textcolor{blue}{
Answer to come...
}
%>>>>>>>>>>>>>>   5.2  >>>>>>>>>>>>>>>>>>>>>>>>>>>>>>>>>>>>>>>>>>>>>>>>>>>>>>>>>>>>>>>>>>>>>>>>>>>>>>>>>>>>>>>>>>>>>>>>>>>>>>>>>>>>>>>>>>>>>>>>>>>>>>>>>>>>>>>>>>>>>>>>>>>
\subsection{}
Compute the derivative $f\pr(x)$ of the logistic sigmoid
$$f(x) = \dfrac{1}{1+\exp(-x)}.$$
%------------------------------------------------------------------------------------------------------------------------------------------------------------------------
\textcolor{blue}{
Answer to come...
}
%>>>>>>>>>>>>>>   5.3  >>>>>>>>>>>>>>>>>>>>>>>>>>>>>>>>>>>>>>>>>>>>>>>>>>>>>>>>>>>>>>>>>>>>>>>>>>>>>>>>>>>>>>>>>>>>>>>>>>>>>>>>>>>>>>>>>>>>>>>>>>>>>>>>>>>>>>>>>>>>>>>>>>>
\subsection{}
Compute the derivative $f\pr(x)$ of the function
$$f(x) = \exp(-\frac{1}{2\sigma^2}(x-mu)^2),$$
where $\mu, \sigma \in \R$ are constants.
%------------------------------------------------------------------------------------------------------------------------------------------------------------------------
\textcolor{blue}{
Answer to come...
}
%>>>>>>>>>>>>>>   5.4  >>>>>>>>>>>>>>>>>>>>>>>>>>>>>>>>>>>>>>>>>>>>>>>>>>>>>>>>>>>>>>>>>>>>>>>>>>>>>>>>>>>>>>>>>>>>>>>>>>>>>>>>>>>>>>>>>>>>>>>>>>>>>>>>>>>>>>>>>>>>>>>>>>>
\subsection{}
Compute the Taylor polynomials $T_n, n=0, ... , 5$ of $f(x) = \sin(x)+\cos(x)$ at $x_0=0$.
%------------------------------------------------------------------------------------------------------------------------------------------------------------------------
\textcolor{blue}{
Answer to come...
}
%>>>>>>>>>>>>>>   5.5  >>>>>>>>>>>>>>>>>>>>>>>>>>>>>>>>>>>>>>>>>>>>>>>>>>>>>>>>>>>>>>>>>>>>>>>>>>>>>>>>>>>>>>>>>>>>>>>>>>>>>>>>>>>>>>>>>>>>>>>>>>>>>>>>>>>>>>>>>>>>>>>>>>>
\subsection{}
Consider the following functions:
\begin{align*}
& f_(x) = \sin(x_1)\cos(x_2), \;\; \vecx \in \R^2 \\
& f_2(x,y) = x^\top y, \;\; \vecx, \vecy \in \R^n \\
& f_3(x) = \vecx \vecx^\top, \;\; \vecx \in \R^n
\end{align*}
\begin{itemize}
 \item [a.] What are the dimensions of $\dfrac{\partial f_i}{\partial \vecx}$
 \item [b.] Compute the Jacbians.
\end{itemize}
%------------------------------------------------------------------------------------------------------------------------------------------------------------------------
\textcolor{blue}{
\begin{itemize}
 \item [a.]
 \item [b.]
\end{itemize}
}
%>>>>>>>>>>>>>>   5.6  >>>>>>>>>>>>>>>>>>>>>>>>>>>>>>>>>>>>>>>>>>>>>>>>>>>>>>>>>>>>>>>>>>>>>>>>>>>>>>>>>>>>>>>>>>>>>>>>>>>>>>>>>>>>>>>>>>>>>>>>>>>>>>>>>>>>>>>>>>>>>>>>>>>
\subsection{}
Differentiate $f$ with respect to $\textbf{t}$ and $g$ with respect to $\textbf{X}$, where
\begin{align*}
& f(t)=\sin(\log(t^\top t)), & t \in \R^D \\
& g(\textbf{X}) = \textrm{tr}(\matA \textbf{X} \matB), & \matA \in \R^{D \times E}, \textbf{X} \in \R^{E \times F}, \matB \in \R^{F \times D},
\end{align*}
where tr$(\cdot)$ denotes the trace.
%------------------------------------------------------------------------------------------------------------------------------------------------------------------------
\textcolor{blue}{
Answer to come...
}
%>>>>>>>>>>>>>>   5.7  >>>>>>>>>>>>>>>>>>>>>>>>>>>>>>>>>>>>>>>>>>>>>>>>>>>>>>>>>>>>>>>>>>>>>>>>>>>>>>>>>>>>>>>>>>>>>>>>>>>>>>>>>>>>>>>>>>>>>>>>>>>>>>>>>>>>>>>>>>>>>>>>>>>
\subsection{}
Compute the derivatives $\textrm{d}f/\textrm{d}x$ of the following functions by using the chain rule. Provide the dimensions of every single partial derivative. Describe your steps in detail.
\begin{itemize}
 \item [a.] $$f(z) = \log(1+z), \;\; z = \vecx^\top \vecx, \;\; \vecx \in \R^D$$
 \item [b.] $$f(z) = \sin(x), \;\; \vecz = \matA \vecx + \vecb, \;\; \matA \in \R^{E \times D}, \;\; \vecx \in \R^D, \vecb \in \R^E$$
 where $\sin(\cdot)$ is applied to every element of $z$.
\end{itemize}

%------------------------------------------------------------------------------------------------------------------------------------------------------------------------
\textcolor{blue}{
\begin{itemize}
 \item [a.]
 \item [b.]
\end{itemize}
}
%>>>>>>>>>>>>>>   5.8  >>>>>>>>>>>>>>>>>>>>>>>>>>>>>>>>>>>>>>>>>>>>>>>>>>>>>>>>>>>>>>>>>>>>>>>>>>>>>>>>>>>>>>>>>>>>>>>>>>>>>>>>>>>>>>>>>>>>>>>>>>>>>>>>>>>>>>>>>>>>>>>>>>>
\subsection{}
Compute the derivatives $\textrm{d}f/\textrm{d}\vecx$ of the following functions. Describe your steps in detail.
\begin{itemize}
 \item [a.] Use the chain rule. Provide the dimensions of every single partial derivative.
 \begin{align*}
 f(x) &= \exp(-\frac{1}{2}z) \\
 z &= g(\vecy) = \vecy^\top \textbf{S}\inv \vecy \\
 \vecy &= h(\vecx) = \vecx - \mu
 \end{align*}
 where $\vecx, \mu \in \R^D, \textbf{S} \in \R^{D \times D}$.
 \item [b.] $$f(\vecx) = \textrm{tr}(\vecx \vecx^\top + \sigma^2 \textrm{I}), \;\; \vecx \in \R^D$$
 Here tr$(\matA)$ is the trace of $\matA$, i.e., the sum of the diagonal elements $A_{ii}$. \\
 \textit{Hint: Explicitly write out the outer product.}
 \item [c.] Use the chain rule. Provide the dimensions of every single partial derivative. You do not need to compute the product of the partial derivatives explicitly.
 \begin{align*}
 \textbf{\textit{f}} &= \tanh(\vecz) \in \R^M \\
 \vecz = \matA \vecx + \vecb, \;\; \vecx \in \R^N, \matA \in \R^{M \times N}, \vecb \in \R^M.
 \end{align*}
 Here, $\tanh$ is applied to every component of $\vecz$.
\end{itemize}
%------------------------------------------------------------------------------------------------------------------------------------------------------------------------
\textcolor{blue}{
\begin{itemize}
 \item [a.]
 \item [b.]
 \item [c.]
\end{itemize}
}
%>>>>>>>>>>>>>>   5.9  >>>>>>>>>>>>>>>>>>>>>>>>>>>>>>>>>>>>>>>>>>>>>>>>>>>>>>>>>>>>>>>>>>>>>>>>>>>>>>>>>>>>>>>>>>>>>>>>>>>>>>>>>>>>>>>>>>>>>>>>>>>>>>>>>>>>>>>>>>>>>>>>>>>
\subsection{}
We define
\begin{align*}
g(\vecz,
\textbf{$\nu$})
&:=
\log p(\vecx, \vecz) - \log q(\vecz, \mathbf{\nu}) \\
\vecz
&:=
t(\mathbf{\epsilon}, \mathbf{\nu})
\end{align*}
for differentiable functions $p,q,t$, and $\vecx \in \R^D, \vecz \in \R^E, \mathbf{\nu} \in \R ^F, \mathbf{\epsilon} \in \R^G$. By using the chain rule, compute the gradient $$\dfrac{\textrm{d}}{\textrm{d} \mathbf{\nu}} g(\vecz, \mathbf{\nu}).$$
%------------------------------------------------------------------------------------------------------------------------------------------------------------------------
\textcolor{blue}{
Answer to come...
}
\newpage%%%%%%%%%%%%%%%%%%%%%%%%%%%%%%%%%%%%%%%%%%%%%%%%%%%%%%%%%%%%%%%
\section{Probability and Distributions}
%>>>>>>>>>>>>>>   6.1  >>>>>>>>>>>>>>>>>>>>>>>>>>>>>>>>>>>>>>>>>>>>>>>>>>>>>>>>>>>>>>>>>>>>>>>>>>>>>>>>>>>>>>>>>>>>>>>>>>>>>>>>>>>>>>>>>>>>>>>>>>>>>>>>>>>>>>>>>>>>>>>>>>>
\subsection{}

%------------------------------------------------------------------------------------------------------------------------------------------------------------------------
\textcolor{blue}{
Answer to come...
}
%>>>>>>>>>>>>>>   6.2  >>>>>>>>>>>>>>>>>>>>>>>>>>>>>>>>>>>>>>>>>>>>>>>>>>>>>>>>>>>>>>>>>>>>>>>>>>>>>>>>>>>>>>>>>>>>>>>>>>>>>>>>>>>>>>>>>>>>>>>>>>>>>>>>>>>>>>>>>>>>>>>>>>>
\subsection{}

%------------------------------------------------------------------------------------------------------------------------------------------------------------------------
\textcolor{blue}{
Answer to come...
}
%>>>>>>>>>>>>>>   6.3  >>>>>>>>>>>>>>>>>>>>>>>>>>>>>>>>>>>>>>>>>>>>>>>>>>>>>>>>>>>>>>>>>>>>>>>>>>>>>>>>>>>>>>>>>>>>>>>>>>>>>>>>>>>>>>>>>>>>>>>>>>>>>>>>>>>>>>>>>>>>>>>>>>>
\subsection{}

%------------------------------------------------------------------------------------------------------------------------------------------------------------------------
\textcolor{blue}{
Answer to come...
}
%>>>>>>>>>>>>>>   6.4  >>>>>>>>>>>>>>>>>>>>>>>>>>>>>>>>>>>>>>>>>>>>>>>>>>>>>>>>>>>>>>>>>>>>>>>>>>>>>>>>>>>>>>>>>>>>>>>>>>>>>>>>>>>>>>>>>>>>>>>>>>>>>>>>>>>>>>>>>>>>>>>>>>>
\subsection{}

%------------------------------------------------------------------------------------------------------------------------------------------------------------------------
\textcolor{blue}{
Answer to come...
}
%>>>>>>>>>>>>>>   6.5  >>>>>>>>>>>>>>>>>>>>>>>>>>>>>>>>>>>>>>>>>>>>>>>>>>>>>>>>>>>>>>>>>>>>>>>>>>>>>>>>>>>>>>>>>>>>>>>>>>>>>>>>>>>>>>>>>>>>>>>>>>>>>>>>>>>>>>>>>>>>>>>>>>>
\subsection{}

%------------------------------------------------------------------------------------------------------------------------------------------------------------------------
\textcolor{blue}{
Answer to come...
}
%>>>>>>>>>>>>>>   6.6  >>>>>>>>>>>>>>>>>>>>>>>>>>>>>>>>>>>>>>>>>>>>>>>>>>>>>>>>>>>>>>>>>>>>>>>>>>>>>>>>>>>>>>>>>>>>>>>>>>>>>>>>>>>>>>>>>>>>>>>>>>>>>>>>>>>>>>>>>>>>>>>>>>>
\subsection{}

%------------------------------------------------------------------------------------------------------------------------------------------------------------------------
\textcolor{blue}{
Answer to come...
}
%>>>>>>>>>>>>>>   6.7  >>>>>>>>>>>>>>>>>>>>>>>>>>>>>>>>>>>>>>>>>>>>>>>>>>>>>>>>>>>>>>>>>>>>>>>>>>>>>>>>>>>>>>>>>>>>>>>>>>>>>>>>>>>>>>>>>>>>>>>>>>>>>>>>>>>>>>>>>>>>>>>>>>>
\subsection{}

%------------------------------------------------------------------------------------------------------------------------------------------------------------------------
\textcolor{blue}{
Answer to come...
}
%>>>>>>>>>>>>>>   6.8  >>>>>>>>>>>>>>>>>>>>>>>>>>>>>>>>>>>>>>>>>>>>>>>>>>>>>>>>>>>>>>>>>>>>>>>>>>>>>>>>>>>>>>>>>>>>>>>>>>>>>>>>>>>>>>>>>>>>>>>>>>>>>>>>>>>>>>>>>>>>>>>>>>>
\subsection{}

%------------------------------------------------------------------------------------------------------------------------------------------------------------------------
\textcolor{blue}{
Answer to come...
}
%>>>>>>>>>>>>>>   6.9  >>>>>>>>>>>>>>>>>>>>>>>>>>>>>>>>>>>>>>>>>>>>>>>>>>>>>>>>>>>>>>>>>>>>>>>>>>>>>>>>>>>>>>>>>>>>>>>>>>>>>>>>>>>>>>>>>>>>>>>>>>>>>>>>>>>>>>>>>>>>>>>>>>>
\subsection{}

%------------------------------------------------------------------------------------------------------------------------------------------------------------------------
\textcolor{blue}{
Answer to come...
}
%>>>>>>>>>>>>>>  6.10  >>>>>>>>>>>>>>>>>>>>>>>>>>>>>>>>>>>>>>>>>>>>>>>>>>>>>>>>>>>>>>>>>>>>>>>>>>>>>>>>>>>>>>>>>>>>>>>>>>>>>>>>>>>>>>>>>>>>>>>>>>>>>>>>>>>>>>>>>>>>>>>>>>>
\subsection{}

%------------------------------------------------------------------------------------------------------------------------------------------------------------------------
\textcolor{blue}{
Answer to come...
}
%>>>>>>>>>>>>>>  6.11  >>>>>>>>>>>>>>>>>>>>>>>>>>>>>>>>>>>>>>>>>>>>>>>>>>>>>>>>>>>>>>>>>>>>>>>>>>>>>>>>>>>>>>>>>>>>>>>>>>>>>>>>>>>>>>>>>>>>>>>>>>>>>>>>>>>>>>>>>>>>>>>>>>>
\subsection{}

%------------------------------------------------------------------------------------------------------------------------------------------------------------------------
\textcolor{blue}{
Answer to come...
}
%>>>>>>>>>>>>>>  6.12  >>>>>>>>>>>>>>>>>>>>>>>>>>>>>>>>>>>>>>>>>>>>>>>>>>>>>>>>>>>>>>>>>>>>>>>>>>>>>>>>>>>>>>>>>>>>>>>>>>>>>>>>>>>>>>>>>>>>>>>>>>>>>>>>>>>>>>>>>>>>>>>>>>>
\subsection{}

%------------------------------------------------------------------------------------------------------------------------------------------------------------------------
\textcolor{blue}{
Answer to come...
}
%>>>>>>>>>>>>>>  6.13  >>>>>>>>>>>>>>>>>>>>>>>>>>>>>>>>>>>>>>>>>>>>>>>>>>>>>>>>>>>>>>>>>>>>>>>>>>>>>>>>>>>>>>>>>>>>>>>>>>>>>>>>>>>>>>>>>>>>>>>>>>>>>>>>>>>>>>>>>>>>>>>>>>>
\subsection{}

%------------------------------------------------------------------------------------------------------------------------------------------------------------------------
\textcolor{blue}{
Answer to come...
}
\newpage%%%%%%%%%%%%%%%%%%%%%%%%%%%%%%%%%%%%%%%%%%%%%%%%%%%%%%%%%%%%%%%
\section{Continuous Optimization}
%>>>>>>>>>>>>>>   7.1  >>>>>>>>>>>>>>>>>>>>>>>>>>>>>>>>>>>>>>>>>>>>>>>>>>>>>>>>>>>>>>>>>>>>>>>>>>>>>>>>>>>>>>>>>>>>>>>>>>>>>>>>>>>>>>>>>>>>>>>>>>>>>>>>>>>>>>>>>>>>>>>>>>>
\subsection{}

%------------------------------------------------------------------------------------------------------------------------------------------------------------------------
\textcolor{blue}{
Answer to come...
}
%>>>>>>>>>>>>>>   7.2  >>>>>>>>>>>>>>>>>>>>>>>>>>>>>>>>>>>>>>>>>>>>>>>>>>>>>>>>>>>>>>>>>>>>>>>>>>>>>>>>>>>>>>>>>>>>>>>>>>>>>>>>>>>>>>>>>>>>>>>>>>>>>>>>>>>>>>>>>>>>>>>>>>>
\subsection{}

%------------------------------------------------------------------------------------------------------------------------------------------------------------------------
\textcolor{blue}{
Answer to come...
}
%>>>>>>>>>>>>>>   7.3  >>>>>>>>>>>>>>>>>>>>>>>>>>>>>>>>>>>>>>>>>>>>>>>>>>>>>>>>>>>>>>>>>>>>>>>>>>>>>>>>>>>>>>>>>>>>>>>>>>>>>>>>>>>>>>>>>>>>>>>>>>>>>>>>>>>>>>>>>>>>>>>>>>>
\subsection{}

%------------------------------------------------------------------------------------------------------------------------------------------------------------------------
\textcolor{blue}{
Answer to come...
}
%>>>>>>>>>>>>>>   7.4  >>>>>>>>>>>>>>>>>>>>>>>>>>>>>>>>>>>>>>>>>>>>>>>>>>>>>>>>>>>>>>>>>>>>>>>>>>>>>>>>>>>>>>>>>>>>>>>>>>>>>>>>>>>>>>>>>>>>>>>>>>>>>>>>>>>>>>>>>>>>>>>>>>>
\subsection{}

%------------------------------------------------------------------------------------------------------------------------------------------------------------------------
\textcolor{blue}{
Answer to come...
}
%>>>>>>>>>>>>>>   7.5  >>>>>>>>>>>>>>>>>>>>>>>>>>>>>>>>>>>>>>>>>>>>>>>>>>>>>>>>>>>>>>>>>>>>>>>>>>>>>>>>>>>>>>>>>>>>>>>>>>>>>>>>>>>>>>>>>>>>>>>>>>>>>>>>>>>>>>>>>>>>>>>>>>>
\subsection{}

%------------------------------------------------------------------------------------------------------------------------------------------------------------------------
\textcolor{blue}{
Answer to come...
}
%>>>>>>>>>>>>>>   7.6  >>>>>>>>>>>>>>>>>>>>>>>>>>>>>>>>>>>>>>>>>>>>>>>>>>>>>>>>>>>>>>>>>>>>>>>>>>>>>>>>>>>>>>>>>>>>>>>>>>>>>>>>>>>>>>>>>>>>>>>>>>>>>>>>>>>>>>>>>>>>>>>>>>>
\subsection{}

%------------------------------------------------------------------------------------------------------------------------------------------------------------------------
\textcolor{blue}{
Answer to come...
}
%>>>>>>>>>>>>>>   7.7  >>>>>>>>>>>>>>>>>>>>>>>>>>>>>>>>>>>>>>>>>>>>>>>>>>>>>>>>>>>>>>>>>>>>>>>>>>>>>>>>>>>>>>>>>>>>>>>>>>>>>>>>>>>>>>>>>>>>>>>>>>>>>>>>>>>>>>>>>>>>>>>>>>>
\subsection{}

%------------------------------------------------------------------------------------------------------------------------------------------------------------------------
\textcolor{blue}{
Answer to come...
}
%>>>>>>>>>>>>>>   7.8  >>>>>>>>>>>>>>>>>>>>>>>>>>>>>>>>>>>>>>>>>>>>>>>>>>>>>>>>>>>>>>>>>>>>>>>>>>>>>>>>>>>>>>>>>>>>>>>>>>>>>>>>>>>>>>>>>>>>>>>>>>>>>>>>>>>>>>>>>>>>>>>>>>>
\subsection{}

%------------------------------------------------------------------------------------------------------------------------------------------------------------------------
\textcolor{blue}{
Answer to come...
}
%>>>>>>>>>>>>>>   7.9  >>>>>>>>>>>>>>>>>>>>>>>>>>>>>>>>>>>>>>>>>>>>>>>>>>>>>>>>>>>>>>>>>>>>>>>>>>>>>>>>>>>>>>>>>>>>>>>>>>>>>>>>>>>>>>>>>>>>>>>>>>>>>>>>>>>>>>>>>>>>>>>>>>>
\subsection{}

%------------------------------------------------------------------------------------------------------------------------------------------------------------------------
\textcolor{blue}{
Answer to come...
}
%>>>>>>>>>>>>>>  7.10  >>>>>>>>>>>>>>>>>>>>>>>>>>>>>>>>>>>>>>>>>>>>>>>>>>>>>>>>>>>>>>>>>>>>>>>>>>>>>>>>>>>>>>>>>>>>>>>>>>>>>>>>>>>>>>>>>>>>>>>>>>>>>>>>>>>>>>>>>>>>>>>>>>>
\subsection{}

%------------------------------------------------------------------------------------------------------------------------------------------------------------------------
\textcolor{blue}{
Answer to come...
}
%>>>>>>>>>>>>>>  7.11  >>>>>>>>>>>>>>>>>>>>>>>>>>>>>>>>>>>>>>>>>>>>>>>>>>>>>>>>>>>>>>>>>>>>>>>>>>>>>>>>>>>>>>>>>>>>>>>>>>>>>>>>>>>>>>>>>>>>>>>>>>>>>>>>>>>>>>>>>>>>>>>>>>>
\subsection{}

%------------------------------------------------------------------------------------------------------------------------------------------------------------------------
\textcolor{blue}{
Answer to come...
}

\end{document}

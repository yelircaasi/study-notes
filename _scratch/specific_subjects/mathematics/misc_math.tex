\documentclass[a4, 12pt]{article}
\usepackage[english]{babel}
\usepackage[utf8]{inputenc}
\usepackage{amsmath}
\usepackage{graphicx}
\usepackage{fancyhdr}
\usepackage[margin=3cm]{geometry}
\usepackage{todonotes}

\newcommand{\followup}[1]{\textcolor{red}{ #1 }}
\newcommand{\vecv}{\textbf{v}}
\newcommand{\vecu}{\textbf{u}}
\newcommand{\vecw}{\textbf{w}}
\newcommand{\no}[1]{\lVert #1 \rVert}

\pagestyle{fancy}
\fancyhf{}
\lhead{}
\lfoot{Math Notes}
\cfoot{\thepage}
\rfoot{Isaac Riley}
\renewcommand{\headrulewidth}{0.5pt}
\renewcommand{\footrulewidth}{0.5pt}

\begin{document}

\begin{titlepage}


\thispagestyle{fancy}

\vphantom{x}

\vspace{0.5in}

\center


\textsc{\large }

\vspace{0.5in}

\noindent\makebox[\linewidth]{\rule{\linewidth}{1.2pt}}\\
\vspace{2mm}
\textsc{ \textbf{\large Miscellaneous Notes - Mathematics }}
\noindent\makebox[\linewidth]{\rule{\linewidth}{1.2pt}}

\vspace{2.5in}
Isaac Riley\\~\\

Last edited: \today

\end{titlepage}

\newpage

\setcounter{page}{2}
\tableofcontents
\newpage
%%%%%%%%%%%%%%%%%%%%%%%%%%%%%%%%%%%%%%%%%%%%%%%%%%%%%%%%%%%%%%%%%%%%%%%%%%%%%%%%%%%%%%%%%%%%
\section{Derivation of Angle Formula for Two Vectors}
\todo{Add graphics}
Let $\vecu$ and $\vecv$ be vectors in an arbitrary vector space. \\~\\
By the Law of Cosines, $$\lVert\vecu - \vecu\rVert^2 = \lVert\vecu\rVert^2+\lVert\vecv\rVert^2-2\lVert\vecu\rVert\cdot\lVert\vecv\rVert\cos \theta.$$
By the definition of norms, $$\lVert\vecu - \vecu\rVert^2 = \vecu \cdot \vecu - 2 \vecu \cdot \vecv + \vecv \cdot \vecv$$
Combining these, we have
\begin{align*}
\vecu \cdot \vecu - 2 \vecu \cdot \vecv + \vecv \cdot \vecv &= \lVert\vecu\rVert^2+\lVert\vecv\rVert^2-2\lVert\vecu\rVert\cdot\lVert\vecv\rVert\cos \theta.\\
-2\vecu \cdot \vecv &= \2 \lVert\vecu\rVert \cdot \lVert\vecv\rVert \cos \theta.\\
\cos \theta &= \dfrac{\vecu \cdot \vecv}{\lVert\vecu\rVert \cdot \lVert\vecv\rVert}.\\
\theta &=
\cos^{-1}
\left(
\dfrac{\vecu \cdot \vecv}{\lVert \vecu \rVert \cdot \lVert \vecv \rVert}
\right).
\end{align*}
Alternatively: $\cos \theta$ is the dot product of corresponding unit vectors.
\section{Dot Product Intuition}
\begin{itemize}
    \item overall similarity of vectors
    \item ranges from 0 (perpendicular) to maximum when collinear
    \item invariant to rotation: $a \cdot b = \no{a}\no{b} \cos \theta$
\end{itemize}

\section{Proof of Law of Cosines}


\section{Proof that $\sin^2 \theta + \cos^2 \theta = 1$}


\section{Eigendecomposition to SVD}


\section{$\sigma$-Algebra}


\section{Proof of Triangle Inequality}

\newpage
\section{Proof of Cauchy-Schwarz Inequality}
Note that $|x \cdot y| \leq \no{x}\no{y}$ is equivalent to $\left( \sum\limits_{i=1}^n x_i^2 \right) \left( \sum\limits_{i=1}^n y_i^2 \right) \geq \left( \sum\limits_{i=1}^n x_i y_i \right)^2$.\\~\\
We derive the result from the self-evident fact that square numbers are nonnegative, the key step involving the fact that $\sum\limits_{i=1}^n x_i = \sum\limits_{j=1}^n x_j$.
\begin{align*}
    \sum\limits_{i=1}^n \sum\limits_{j=1}^n (x_i y_j - x_j y_i)^2 &\geq 0 \\
    \sum\limits_{i=1}^n x_i^2 \sum\limits_{j=1}^n y_j^2 +\sum\limits_{i=1}^n y_i^2 \sum\limits_{j=1}^n x_j^2 - 2\sum\limits_{i=1}^n x_i y_i \sum\limits_{j=1}^n x_j y_j &\geq 0 \\
    2\left( \sum\limits_{i=1}^n x_i^2 \right)\left( \sum\limits_{i=1}^n y_i^2 \right) - 2\left( \sum\limits_{i=1}^n x_i y_i \right)^2 &\geq 0  \\
    \Rightarrow \left( \sum\limits_{i=1}^n x_i^2 \right)\left( \sum\limits_{i=1}^n y_i^2 \right) - \left( \sum\limits_{i=1}^n x_i y_i \right)^2 &\geq 0 \\
    \no{x}^2\no{y}^2 &\geq (x \cdot y)^2 \\
    \no{x}\no{y} &\geq |x \cdot y|
\end{align*}
\todo{Dig into intuition!}
\section{Alternative Proof of Cauchy-Schwarz Inequality}


\section{Proof of Vector Projection Formula}


\section{Constructive Derivation of General Determinant}


%%%%%%%%%%%%%%%%%%%%%%%%%%%%%%%%%%%%%%%%%%%%%%%%%%%%%%%%%%%%%%%%%%%%%%%%%%%%%%%%%%%%%%%%%%%%
\bibliography{sources}
\bibliographystyle{IEEEtran}
\end{document}

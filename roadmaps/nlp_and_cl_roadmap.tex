\documentclass[a4, landscape, 12pt]{article}
\usepackage[english]{babel}
\usepackage[utf8]{inputenc}
\usepackage[margin=2cm]{geometry}
\usepackage{amssymb}
\usepackage{pifont}
\usepackage{hyperref}
\newcommand{\cmark}{\ding{51}}%
%\newcommand{\xmark}{\ding{55}}%
\newcommand{\checkbox}{$\square$}%
%\newcommand{\checked}{ \rlap{$\square$}{\hspace{1pt}\cmark} }
\newcommand{\checked}{\rlap{\raisebox{0.3ex}{\hspace{0.4ex}\tiny \ding{52}}}$\square$}

\title{Natural Language Processing and Computational Linguistics Roadmap}
\author{Isaac Riley}
\date{Last edited on \today}

\begin{document}
\maketitle
\begin{itemize}
\item [\checkbox]  \href{https://nlpoverview.com/}{NLP Overview} | \textbf{ Chapter 1
}
\item [\checkbox]  \href{https://nlpoverview.com/}{NLP Overview} | \textbf{ Chapter 2
}
\item [\checkbox]  \href{https://nlpoverview.com/}{NLP Overview} | \textbf{ Chapter 3
}
\item [\checkbox]  \href{https://nlpoverview.com/}{NLP Overview} | \textbf{ Chapter 4
}
\item [\checkbox]  \href{https://nlpoverview.com/}{NLP Overview} | \textbf{ Chapter 5
}
\item [\checkbox]  \href{https://nlpoverview.com/}{NLP Overview} | \textbf{ Chapter 6
}
\item [\checkbox]  \href{https://nlpoverview.com/}{NLP Overview} | \textbf{ Chapter 7
}
\item [\checkbox]  \href{https://nlpoverview.com/}{NLP Overview} | \textbf{ Chapter 8
}
\item [\checkbox]  NLP Overview | \textbf{ Chapter 9
}
\item [\checkbox]  SLP, 3E | \textbf{ 1. Introduction
}
\item [\checkbox]  SLP, 3E | \textbf{ 2. Regular Expressions, Text Normalization, Edit Distance
}
\item [\checkbox]  SLP, 3E | \textbf{ 3. N-gram Language Models
}
\item [\checkbox]  SLP, 3E | \textbf{ 4. Naive Bayes and Sentiment Classification
}
\item [\checkbox]  SLP, 3E | \textbf{ 5. Logistic Regression
}
\item [\checkbox]  SLP, 3E | \textbf{ 6. Vector Semantics and Embeddings
}
\item [\checkbox]  SLP, 3E | \textbf{ 7. Neural Networks and Neural Language Models
}
\item [\checkbox]  SLP, 3E | \textbf{ 8. Sequence Labeling for Parts of Speech and Named Entities
}
\item [\checkbox]  SLP, 3E | \textbf{ 9. Deep Learning Architectures for Sequence Processing
}
\item [\checkbox]  (SLP, 3E | \textbf{ 10. Contextual Embeddings)
}
\item [\checkbox]  SLP, 3E | \textbf{ 11. Machine Translation and Encoder-Decoder Models
}
\item [\checkbox]  SLP, 3E | \textbf{ 12. Constituency Grammars
}
\item [\checkbox]  SLP, 3E | \textbf{ 13. Constituency Parsing
}
\item [\checkbox]  SLP, 3E | \textbf{ 14. Dependency Parsing
}
\item [\checkbox]  SLP, 3E | \textbf{ 15. Logical Representations of Sentence Meaning
}
\item [\checkbox]  (SLP, 3E | \textbf{ 16. Computational Semantics and Semantic Parsing)
}
\item [\checkbox]  SLP, 3E | \textbf{ 17. Information Extraction
}
\item [\checkbox]  SLP, 3E | \textbf{ 18. Word Senses and WordNet
}
\item [\checkbox]  SLP, 3E | \textbf{ 19. Semantic Role Labeling
}
\item [\checkbox]  SLP, 3E | \textbf{ 20. Lexicons for Sentiment, Affect, and Connotation
}
\item [\checkbox]  SLP, 3E | \textbf{ 21. Coreference Resolution
}
\item [\checkbox]  SLP, 3E | \textbf{ 22. Discourse Coherence
}
\item [\checkbox]  SLP, 3E | \textbf{ 23. Question Answering
}
\item [\checkbox]  SLP, 3E | \textbf{ 24. Chatbots & Dialogue Systems
}
\item [\checkbox]  SLP, 3E | \textbf{ 25. Phonetics
}
\item [\checkbox]  SLP, 3E | \textbf{ 26. Automatic Speech Recognition and Text-to-Speech
}
\item [\checkbox]  NLP and CL I (Kurdi) | \textbf{ Chapter 1. Linguistic Resources for NLP
}
\item [\checkbox]  NLP and CL I (Kurdi) | \textbf{ 1.1. The concept of a corpus
}
\item [\checkbox]  NLP and CL I (Kurdi) | \textbf{ 1.2. Corpus taxonomy
}
\item [\checkbox]  NLP and CL I (Kurdi) | \textbf{ 1.3. Who collects and distributes corpora?
}
\item [\checkbox]  NLP and CL I (Kurdi) | \textbf{ 1.4. The lifecycle of a corpus
}
\item [\checkbox]  NLP and CL I (Kurdi) | \textbf{ 1.5. Examples of existing corpora
}
\item [\checkbox]  NLP and CL I (Kurdi) | \textbf{ Chapter 2. The Sphere of Speech
}
\item [\checkbox]  NLP and CL I (Kurdi) | \textbf{ 2.1. Linguistic studies of speech
}
\item [\checkbox]  NLP and CL I (Kurdi) | \textbf{ 2.2. Speech processing
}
\item [\checkbox]  NLP and CL I (Kurdi) | \textbf{ Chapter 3. Morphology Sphere
}
\item [\checkbox]  NLP and CL I (Kurdi) | \textbf{ 3.1. Elements of morphology
}
\item [\checkbox]  NLP and CL I (Kurdi) | \textbf{ 3.2. Automatic morphological analysis
}
\item [\checkbox]  NLP and CL I (Kurdi) | \textbf{ Chapter 4. Syntax Sphere
}
\item [\checkbox]  NLP and CL I (Kurdi) | \textbf{ 4.1. Basic syntactic concepts
}
\item [\checkbox]  NLP and CL I (Kurdi) | \textbf{ 4.2. Elements of formal syntax
}
\item [\checkbox]  NLP and CL I (Kurdi) | \textbf{ 4.3. Syntactic formalisms
}
\item [\checkbox]  NLP and CL I (Kurdi) | \textbf{ 4.4. Automatic parsing
}
\item [\checkbox]  NLP and CL II (Kurdi) | \textbf{ Chapter 1. The Sphere of Lexicons and Knowledge
}
\item [\checkbox]  NLP and CL II (Kurdi) | \textbf{ 1.1. Lexical semantics
}
\item [\checkbox]  NLP and CL II (Kurdi) | \textbf{ 1.2. Lexical databases
}
\item [\checkbox]  NLP and CL II (Kurdi) | \textbf{ 1.3. Knowledge representation and ontologies
}
\item [\checkbox]  NLP and CL II (Kurdi) | \textbf{ Chapter 2. The Sphere of Semantics
}
\item [\checkbox]  NLP and CL II (Kurdi) | \textbf{ 2.1. Combinatorial semantics
}
\item [\checkbox]  NLP and CL II (Kurdi) | \textbf{ 2.2. Formal semantics
}
\item [\checkbox]  NLP and CL II (Kurdi) | \textbf{ Chapter 3. The Sphere of Discourse and Text
}
\item [\checkbox]  NLP and CL II (Kurdi) | \textbf{ 3.1. Discourse analysis and pragmatics
}
\item [\checkbox]  NLP and CL II (Kurdi) | \textbf{ 3.2. Computational approaches to discourse
}
\item [\checkbox]  NLP and CL II (Kurdi) | \textbf{ Chapter 4. The Sphere of Applications
}
\item [\checkbox]  NLP and CL II (Kurdi) | \textbf{ 4.1. Software engineering for NLP software
}
\item [\checkbox]  NLP and CL II (Kurdi) | \textbf{ 4.2. Machine translation (MT)
}
\item [\checkbox]  NLP and CL II (Kurdi) | \textbf{ 4.3. Information retrieval (IR)
}
\item [\checkbox]  NLP and CL II (Kurdi) | \textbf{ 4.4. Big Data (BD) and information extraction
}
\item [\checkbox]  NLP (Eisenstein) | \textbf{ 1 Introduction
}
\item [\checkbox]  NLP (Eisenstein) | \textbf{ 1.1 Natural language processing and its neighbors
}
\item [\checkbox]  NLP (Eisenstein) | \textbf{ 1.2 Three themes in natural language processing
}
\item [\checkbox]  NLP (Eisenstein) | \textbf{ I Learning
}
\item [\checkbox]  NLP (Eisenstein) | \textbf{ 2 Linear text classification
}
\item [\checkbox]  NLP (Eisenstein) | \textbf{ 2.1 The bag of words
}
\item [\checkbox]  NLP (Eisenstein) | \textbf{ 2.2 Naïve Bayes
}
\item [\checkbox]  NLP (Eisenstein) | \textbf{ 2.3 Discriminative learning
}
\item [\checkbox]  NLP (Eisenstein) | \textbf{ 2.4 Loss functions and large-margin classification
}
\item [\checkbox]  NLP (Eisenstein) | \textbf{ 2.5 Logistic regression
}
\item [\checkbox]  NLP (Eisenstein) | \textbf{ 2.6 Optimization
}
\item [\checkbox]  NLP (Eisenstein) | \textbf{ 2.7 *Additional topics in classification
}
\item [\checkbox]  NLP (Eisenstein) | \textbf{ 2.8 Summary of learning algorithms
}
\item [\checkbox]  NLP (Eisenstein) | \textbf{ 3 Nonlinear classification
}
\item [\checkbox]  NLP (Eisenstein) | \textbf{ 3.1 Feedforward neural networks
}
\item [\checkbox]  NLP (Eisenstein) | \textbf{ 3.2 Designing neural networks
}
\item [\checkbox]  NLP (Eisenstein) | \textbf{ 3.3 Learning neural networks
}
\item [\checkbox]  NLP (Eisenstein) | \textbf{ 3.4 Convolutional neural networks
}
\item [\checkbox]  NLP (Eisenstein) | \textbf{ 4 Linguistic applications of classification
}
\item [\checkbox]  NLP (Eisenstein) | \textbf{ 4.1 Sentiment and opinion analysis
}
\item [\checkbox]  NLP (Eisenstein) | \textbf{ 4.2 Word sense disambiguation
}
\item [\checkbox]  NLP (Eisenstein) | \textbf{ 4.3 Design decisions for text classification
}
\item [\checkbox]  NLP (Eisenstein) | \textbf{ 4.4 Evaluating classifiers
}
\item [\checkbox]  NLP (Eisenstein) | \textbf{ 4.5 Building datasets
}
\item [\checkbox]  NLP (Eisenstein) | \textbf{ 5 Learning without supervision
}
\item [\checkbox]  NLP (Eisenstein) | \textbf{ 5.1 Unsupervised learning
}
\item [\checkbox]  NLP (Eisenstein) | \textbf{ 5.2 Applications of expectation-maximization
}
\item [\checkbox]  NLP (Eisenstein) | \textbf{ 5.3 Semi-supervised learning
}
\item [\checkbox]  NLP (Eisenstein) | \textbf{ 5.4 Domain adaptation
}
\item [\checkbox]  NLP (Eisenstein) | \textbf{ 5.5 *Other approaches to learning with latent variables
}
\item [\checkbox]  NLP (Eisenstein) | \textbf{ II Sequences and trees
}
\item [\checkbox]  NLP (Eisenstein) | \textbf{ 6 Language models
}
\item [\checkbox]  NLP (Eisenstein) | \textbf{ 6.1 N-gram language models
}
\item [\checkbox]  NLP (Eisenstein) | \textbf{ 6.2 Smoothing and discounting
}
\item [\checkbox]  NLP (Eisenstein) | \textbf{ 6.3 Recurrent neural network language models
}
\item [\checkbox]  NLP (Eisenstein) | \textbf{ 6.4 Evaluating language models
}
\item [\checkbox]  NLP (Eisenstein) | \textbf{ 6.5 Out-of-vocabulary words
}
\item [\checkbox]  NLP (Eisenstein) | \textbf{ 7 Sequence labeling
}
\item [\checkbox]  NLP (Eisenstein) | \textbf{ 7.1 Sequence labeling as classification
}
\item [\checkbox]  NLP (Eisenstein) | \textbf{ 7.2 Sequence labeling as structure prediction
}
\item [\checkbox]  NLP (Eisenstein) | \textbf{ 7.3 The Viterbi algorithm
}
\item [\checkbox]  NLP (Eisenstein) | \textbf{ 7.3.1 Example
}
\item [\checkbox]  NLP (Eisenstein) | \textbf{ 7.3.2 Higher-order features
}
\item [\checkbox]  NLP (Eisenstein) | \textbf{ 7.4 Hidden Markov Models
}
\item [\checkbox]  NLP (Eisenstein) | \textbf{ 7.5 Discriminative sequence labeling with features
}
\item [\checkbox]  NLP (Eisenstein) | \textbf{ 7.6 Neural sequence labeling
}
\item [\checkbox]  NLP (Eisenstein) | \textbf{ 7.7 *Unsupervised sequence labeling
}
\item [\checkbox]  NLP (Eisenstein) | \textbf{ 8 Applications of sequence labeling
}
\item [\checkbox]  NLP (Eisenstein) | \textbf{ 8.1 Part-of-speech tagging
}
\item [\checkbox]  NLP (Eisenstein) | \textbf{ 8.2 Morphosyntactic Attributes
}
\item [\checkbox]  NLP (Eisenstein) | \textbf{ 8.3 Named Entity Recognition
}
\item [\checkbox]  NLP (Eisenstein) | \textbf{ 8.4 Tokenization
}
\item [\checkbox]  NLP (Eisenstein) | \textbf{ 8.5 Code switching
}
\item [\checkbox]  NLP (Eisenstein) | \textbf{ 8.6 Dialogue acts
}
\item [\checkbox]  NLP (Eisenstein) | \textbf{ 9 Formal language theory
}
\item [\checkbox]  NLP (Eisenstein) | \textbf{ 9.1 Regular languages
}
\item [\checkbox]  NLP (Eisenstein) | \textbf{ 9.2 Context-free languages
}
\item [\checkbox]  NLP (Eisenstein) | \textbf{ 9.3 *Mildly context-sensitive languages
}
\item [\checkbox]  NLP (Eisenstein) | \textbf{ 10 Context-free parsing
}
\item [\checkbox]  NLP (Eisenstein) | \textbf{ 10.1 Deterministic bottom-up parsing
}
\item [\checkbox]  NLP (Eisenstein) | \textbf{ 10.2 Ambiguity
}
\item [\checkbox]  NLP (Eisenstein) | \textbf{ 10.3 Weighted Context-Free Grammars
}
\item [\checkbox]  NLP (Eisenstein) | \textbf{ 10.4 Learning weighted context-free grammars
}
\item [\checkbox]  NLP (Eisenstein) | \textbf{ 10.5 Grammar refinement
}
\item [\checkbox]  NLP (Eisenstein) | \textbf{ 10.6 Beyond context-free parsing
}
\item [\checkbox]  NLP (Eisenstein) | \textbf{ 11 Dependency parsing
}
\item [\checkbox]  NLP (Eisenstein) | \textbf{ 11.1 Dependency grammar
}
\item [\checkbox]  NLP (Eisenstein) | \textbf{ 11.2 Graph-based dependency parsing
}
\item [\checkbox]  NLP (Eisenstein) | \textbf{ 11.3 Transition-based dependency parsing
}
\item [\checkbox]  NLP (Eisenstein) | \textbf{ 11.4 Applications
}
\item [\checkbox]  NLP (Eisenstein) | \textbf{ III Meaning
}
\item [\checkbox]  NLP (Eisenstein) | \textbf{ 12 Logical semantics
}
\item [\checkbox]  NLP (Eisenstein) | \textbf{ 12.1 Meaning and denotation
}
\item [\checkbox]  NLP (Eisenstein) | \textbf{ 12.2 Logical representations of meaning
}
\item [\checkbox]  NLP (Eisenstein) | \textbf{ 12.3 Semantic parsing and the lambda calculus
}
\item [\checkbox]  NLP (Eisenstein) | \textbf{ 12.4 Learning semantic parsers
}
\item [\checkbox]  NLP (Eisenstein) | \textbf{ 13 Predicate-argument semantics
}
\item [\checkbox]  NLP (Eisenstein) | \textbf{ 13.1 Semantic roles
}
\item [\checkbox]  NLP (Eisenstein) | \textbf{ 13.2 Semantic role labeling
}
\item [\checkbox]  NLP (Eisenstein) | \textbf{ 13.3 Abstract Meaning Representation
}
\item [\checkbox]  NLP (Eisenstein) | \textbf{ 14 Distributional and distributed semantics
}
\item [\checkbox]  NLP (Eisenstein) | \textbf{ 14.1 The distributional hypothesis
}
\item [\checkbox]  NLP (Eisenstein) | \textbf{ 14.2 Design decisions for word representations
}
\item [\checkbox]  NLP (Eisenstein) | \textbf{ 14.3 Latent semantic analysis
}
\item [\checkbox]  NLP (Eisenstein) | \textbf{ 14.4 Brown clusters
}
\item [\checkbox]  NLP (Eisenstein) | \textbf{ 14.5 Neural word embeddings
}
\item [\checkbox]  NLP (Eisenstein) | \textbf{ 14.6 Evaluating word embeddings
}
\item [\checkbox]  NLP (Eisenstein) | \textbf{ 14.7 Distributed representations beyond distributional statistics
}
\item [\checkbox]  NLP (Eisenstein) | \textbf{ 14.8 Distributed representations of multiword units
}
\item [\checkbox]  NLP (Eisenstein) | \textbf{ 15 Reference Resolution
}
\item [\checkbox]  NLP (Eisenstein) | \textbf{ 15.1 Forms of referring expressions
}
\item [\checkbox]  NLP (Eisenstein) | \textbf{ 15.2 Algorithms for coreference resolution
}
\item [\checkbox]  NLP (Eisenstein) | \textbf{ 15.3 Representations for coreference resolution
}
\item [\checkbox]  NLP (Eisenstein) | \textbf{ 15.4 Evaluating coreference resolution
}
\item [\checkbox]  NLP (Eisenstein) | \textbf{ 16 Discourse
}
\item [\checkbox]  NLP (Eisenstein) | \textbf{ 16.1 Segments
}
\item [\checkbox]  NLP (Eisenstein) | \textbf{ 16.2 Entities and reference
}
\item [\checkbox]  NLP (Eisenstein) | \textbf{ 16.3 Relations
}
\item [\checkbox]  NLP (Eisenstein) | \textbf{ IV Applications
}
\item [\checkbox]  NLP (Eisenstein) | \textbf{ 17 Information extraction
}
\item [\checkbox]  NLP (Eisenstein) | \textbf{ 17.1 Entities
}
\item [\checkbox]  NLP (Eisenstein) | \textbf{ 17.2 Relations
}
\item [\checkbox]  NLP (Eisenstein) | \textbf{ 17.3 Events
}
\item [\checkbox]  NLP (Eisenstein) | \textbf{ 17.4 Hedges, denials, and hypotheticals
}
\item [\checkbox]  NLP (Eisenstein) | \textbf{ 17.5 Question answering and machine reading
}
\item [\checkbox]  NLP (Eisenstein) | \textbf{ 18 Machine translation
}
\item [\checkbox]  NLP (Eisenstein) | \textbf{ 18.1 Machine translation as a task
}
\item [\checkbox]  NLP (Eisenstein) | \textbf{ 18.2 Statistical machine translation
}
\item [\checkbox]  NLP (Eisenstein) | \textbf{ 18.3 Neural machine translation
}
\item [\checkbox]  NLP (Eisenstein) | \textbf{ 18.4 Decoding
}
\item [\checkbox]  NLP (Eisenstein) | \textbf{ 18.5 Training towards the evaluation metric
}
\item [\checkbox]  NLP (Eisenstein) | \textbf{ 19 Text generation
}
\item [\checkbox]  NLP (Eisenstein) | \textbf{ 19.1 Data-to-text generation
}
\item [\checkbox]  NLP (Eisenstein) | \textbf{ 19.2 Text-to-text generation
}
\item [\checkbox]  NLP (Eisenstein) | \textbf{ 19.3 Dialogue
}
\item [\checkbox]  NLP (Eisenstein) | \textbf{ A Probability
}
\item [\checkbox]  NLP (Eisenstein) | \textbf{ A.1 Probabilities of event combinations
}
\item [\checkbox]  NLP (Eisenstein) | \textbf{ A.2 Conditional probability and Bayes’ rule
}
\item [\checkbox]  NLP (Eisenstein) | \textbf{ A.3 Independence
}
\item [\checkbox]  NLP (Eisenstein) | \textbf{ A.4 Random variables
}
\item [\checkbox]  NLP (Eisenstein) | \textbf{ A.5 Expectations
}
\item [\checkbox]  NLP (Eisenstein) | \textbf{ A.6 Modeling and estimation
}
\item [\checkbox]  NLP (Eisenstein) | \textbf{ B Numerical optimization
}
\item [\checkbox]  NLP (Eisenstein) | \textbf{ B.1 Gradient descent
}
\item [\checkbox]  NLP (Eisenstein) | \textbf{ B.2 Constrained optimization
}
\item [\checkbox]  NLP (Eisenstein) | \textbf{ B.3 Example: Passive-aggressive online learning
}
\item [\checkbox]  Deep Learning for NLP and Speech Recognition | \textbf{ 1. Introduction
}
\item [\checkbox]  Deep Learning for NLP and Speech Recognition | \textbf{ 2. Basics of Machine Learning
}
\item [\checkbox]  Deep Learning for NLP and Speech Recognition | \textbf{ 3. Text and Speech Basics
}
\item [\checkbox]  Deep Learning for NLP and Speech Recognition | \textbf{ 4. Basics of Deep Learning
}
\item [\checkbox]  Deep Learning for NLP and Speech Recognition | \textbf{ 5. Distributed Representations
}
\item [\checkbox]  Deep Learning for NLP and Speech Recognition | \textbf{ 6. Convolutional Neural Networks
}
\item [\checkbox]  Deep Learning for NLP and Speech Recognition | \textbf{ 7. Recurrent Neural Networks
}
\item [\checkbox]  Deep Learning for NLP and Speech Recognition | \textbf{ 8. Automatic Speech Recognition
}
\item [\checkbox]  Deep Learning for NLP and Speech Recognition | \textbf{ 9. Attention and Memory Augmented Networks
}
\item [\checkbox]  Deep Learning for NLP and Speech Recognition | \textbf{ 10. Transfer Learning: Scenarios, Self-Taught Learning, and Multitask Learning
}
\item [\checkbox]  Deep Learning for NLP and Speech Recognition | \textbf{ 11. Transfer Learning: Domain Adaptation
}
\item [\checkbox]  Deep Learning for NLP and Speech Recognition | \textbf{ 12. End-to-End Speech Recognition
}
\item [\checkbox]  Deep Learning for NLP and Speech Recognition | \textbf{ 13. Deep Reinforcement Learning for Text and Speech
}
\item [\checkbox]  Deep Learning for NLP and Speech Recognition | \textbf{ A. References
}
\item [\checkbox]  Deep Learning for NLP and Speech Recognition | \textbf{ B. Future Outlook
}
\item [\checkbox]  Deep Learning for NLP and Speech Recognition | \textbf{ C. End-to-End Architecture Prevalence
}
\item [\checkbox]  Deep Learning for NLP and Speech Recognition | \textbf{ D. Transition to AI-Centric
}
\item [\checkbox]  Deep Learning for NLP and Speech Recognition | \textbf{ E. Specialized Hardware
}
\item [\checkbox]  Deep Learning for NLP and Speech Recognition | \textbf{ F. Transition Away from Supervised Learning
}
\item [\checkbox]  Deep Learning for NLP and Speech Recognition | \textbf{ G. Explainable AI
}
\item [\checkbox]  Deep Learning for NLP and Speech Recognition | \textbf{ H. Model Development and Deployment Process
}
\item [\checkbox]  Deep Learning for NLP and Speech Recognition | \textbf{ I. Democratization of AI
}
\item [\checkbox]  Deep Learning for NLP and Speech Recognition | \textbf{ J. NLP Trends
}
\item [\checkbox]  Deep Learning for NLP and Speech Recognition | \textbf{ K. Speech Trends
}
\item [\checkbox]  Deep Learning for NLP and Speech Recognition | \textbf{ L. Closing Remarks
}
\item [\checkbox]  Deep Learning in Natural Language Processing | \textbf{ 1. A Joint Introduction to Natural Processing and to Deep Learning
}
\item [\checkbox]  Deep Learning in Natural Language Processing | \textbf{ 2. Deep Learning in Conversational Language Understanding
}
\item [\checkbox]  Deep Learning in Natural Language Processing | \textbf{ 3. Deep Learning in Spoken and Text-Based Dialog Systems
}
\item [\checkbox]  Deep Learning in Natural Language Processing | \textbf{ 4. Deep Learning in Lexical Analysis and Parsing
}
\item [\checkbox]  Deep Learning in Natural Language Processing | \textbf{ 5. Deep Learning in Knowledge Graphs
}
\item [\checkbox]  Deep Learning in Natural Language Processing | \textbf{ 6. Deep Learning in Machine Translation
}
\item [\checkbox]  Deep Learning in Natural Language Processing | \textbf{ 7. Deep Learning in Question Answering
}
\item [\checkbox]  Deep Learning in Natural Language Processing | \textbf{ 8. Deep Learning in Sentiment Analysis
}
\item [\checkbox]  Deep Learning in Natural Language Processing | \textbf{ 9. Deep Learning in Sociial Computing
}
\item [\checkbox]  Deep Learning in Natural Language Processing | \textbf{ 10. Deep Learning in Natural Language Generation from Images
}
\item [\checkbox]  Deep Learning in Natural Language Processing | \textbf{ 11. Epilogue: Frontiers of NLP in the Deep Learning Era
}
\item [\checkbox]  Oxford Handbook of CL and NLP (2010) | \textbf{ 1. Formal Language Theory
}
\item [\checkbox]  Oxford Handbook of CL and NLP (2010) | \textbf{ 2. Computational Complexity in Natural Language
}
\item [\checkbox]  Oxford Handbook of CL and NLP (2010) | \textbf{ 3. Statistical Language Modeling
}
\item [\checkbox]  Oxford Handbook of CL and NLP (2010) | \textbf{ 4. Theory of Parsing
}
\item [\checkbox]  Oxford Handbook of CL and NLP (2010) | \textbf{ 5. Maximum Entropy Models
}
\item [\checkbox]  Oxford Handbook of CL and NLP (2010) | \textbf{ 6. Memory-Based Learning
}
\item [\checkbox]  Oxford Handbook of CL and NLP (2010) | \textbf{ 7. Decision Trees
}
\item [\checkbox]  Oxford Handbook of CL and NLP (2010) | \textbf{ 8. Unsupervised Learning and Grammar Induction
}
\item [\checkbox]  Oxford Handbook of CL and NLP (2010) | \textbf{ 9. Artificial Neural Networks
}
\item [\checkbox]  Oxford Handbook of CL and NLP (2010) | \textbf{ 10. Linguistic Annotation
}
\item [\checkbox]  Oxford Handbook of CL and NLP (2010) | \textbf{ 11. Evaluation of NLP Systems
}
\item [\checkbox]  Oxford Handbook of CL and NLP (2010) | \textbf{ 12. Speech Recognition
}
\item [\checkbox]  Oxford Handbook of CL and NLP (2010) | \textbf{ 13. Statistical Parsing
}
\item [\checkbox]  Oxford Handbook of CL and NLP (2010) | \textbf{ 14. Segmentation and Morphology
}
\item [\checkbox]  Oxford Handbook of CL and NLP (2010) | \textbf{ 15. Computational Semantics
}
\item [\checkbox]  Oxford Handbook of CL and NLP (2010) | \textbf{ 16. Computational Models of Dialogue
}
\item [\checkbox]  Oxford Handbook of CL and NLP (2010) | \textbf{ 17. Computational Psycholinguistics
}
\item [\checkbox]  Oxford Handbook of CL and NLP (2010) | \textbf{ 18. Information Extraction
}
\item [\checkbox]  Oxford Handbook of CL and NLP (2010) | \textbf{ 19. Machine Translation
}
\item [\checkbox]  Oxford Handbook of CL and NLP (2010) | \textbf{ 20. Natural Language Generation
}
\item [\checkbox]  Oxford Handbook of CL and NLP (2010) | \textbf{ 21. Discourse Processing
}
\item [\checkbox]  Oxford Handbook of CL and NLP (2010) | \textbf{ 22. Question Answering
}
\item [\checkbox]  CRC Handbook of NLP (2010) | \textbf{ 1. Classical Approaches to Natural Language Processing
}
\item [\checkbox]  CRC Handbook of NLP (2010) | \textbf{ 2. Text Preprocessing
}
\item [\checkbox]  CRC Handbook of NLP (2010) | \textbf{ 3. Lexical Analysis
}
\item [\checkbox]  CRC Handbook of NLP (2010) | \textbf{ 4. Syntactic Parsing
}
\item [\checkbox]  CRC Handbook of NLP (2010) | \textbf{ 5. Semantic Analysis
}
\item [\checkbox]  CRC Handbook of NLP (2010) | \textbf{ 6. Natural Language Generation
}
\item [\checkbox]  CRC Handbook of NLP (2010) | \textbf{ 7. Corpus Creation
}
\item [\checkbox]  CRC Handbook of NLP (2010) | \textbf{ 8. Treebank Annotation
}
\item [\checkbox]  CRC Handbook of NLP (2010) | \textbf{ 9. Fundamental Statistical Techniques
}
\item [\checkbox]  CRC Handbook of NLP (2010) | \textbf{ 10. Part-of-Speech Tagging
}
\item [\checkbox]  CRC Handbook of NLP (2010) | \textbf{ 11. Statistical Parsing
}
\item [\checkbox]  CRC Handbook of NLP (2010) | \textbf{ 12. Multiword Expressions
}
\item [\checkbox]  CRC Handbook of NLP (2010) | \textbf{ 13. Normalized Web Distance and Word Similarity
}
\item [\checkbox]  CRC Handbook of NLP (2010) | \textbf{ 14. Word Sense Disambiguation
}
\item [\checkbox]  CRC Handbook of NLP (2010) | \textbf{ 15. An Overview of Modern Speech Recognition
}
\item [\checkbox]  CRC Handbook of NLP (2010) | \textbf{ 16. Alignment
}
\item [\checkbox]  CRC Handbook of NLP (2010) | \textbf{ 17. Statistical Machine Translation
}
\item [\checkbox]  CRC Handbook of NLP (2010) | \textbf{ 18. Chinese Machine Translation
}
\item [\checkbox]  CRC Handbook of NLP (2010) | \textbf{ 19. Information Retrieval
}
\item [\checkbox]  CRC Handbook of NLP (2010) | \textbf{ 20. Question Answering
}
\item [\checkbox]  CRC Handbook of NLP (2010) | \textbf{ 21. Information Extraction
}
\item [\checkbox]  CRC Handbook of NLP (2010) | \textbf{ 22. Report Generation
}
\item [\checkbox]  CRC Handbook of NLP (2010) | \textbf{ 23. Emerging Applications of Natural Language Generation in Information Visualization, Education, and Health Care
}
\item [\checkbox]  CRC Handbook of NLP (2010) | \textbf{ 24. Ontology Construction
}
\item [\checkbox]  CRC Handbook of NLP (2010) | \textbf{ 25. BioNLP: Biomedical Text Mining
}
\item [\checkbox]  CRC Handbook of NLP (2010) | \textbf{ 26. Sentiment Analysis and Subjectivity
}
\item [\checkbox]  Mathematical Linguistics | \textbf{ 1. Introduction
}
\item [\checkbox]  Mathematical Linguistics | \textbf{ 2. The Elements
}
\item [\checkbox]  Mathematical Linguistics | \textbf{ 3. Phonology
}
\item [\checkbox]  Mathematical Linguistics | \textbf{ 4. Morphology
}
\item [\checkbox]  Mathematical Linguistics | \textbf{ 5. Syntax
}
\item [\checkbox]  Mathematical Lingusitics | \textbf{ 6. Semantics
}
\item [\checkbox]  Mathematical Lingusitics | \textbf{ 7. Complexity
}
\item [\checkbox]  Mathematical Lingusitics | \textbf{ 8. Linguistic Pattern Recognition
}
\item [\checkbox]  Mathematical Lingusitics | \textbf{ 9. Speech and Handwriting
}
\item [\checkbox]  Mathematical Lingusitics | \textbf{ 10. Simplicity
}
\item [\checkbox]  An Introduction to Information Retrieval | \textbf{ 1. Boolean retrieval
}
\item [\checkbox]  An Introduction to Information Retrieval | \textbf{ 2. The term vocabulary and postings lists
}
\item [\checkbox]  An Introduction to Information Retrieval | \textbf{ 3. Dictionaries and tolerant retrieval
}
\item [\checkbox]  An Introduction to Information Retrieval | \textbf{ 4. Index construction
}
\item [\checkbox]  An Introduction to Information Retrieval | \textbf{ 5. Index compression
}
\item [\checkbox]  An Introduction to Information Retrieval | \textbf{ 6. Scoring, term weighting and the vector space model
}
\item [\checkbox]  An Introduction to Information Retrieval | \textbf{ 7. Computing scores in a complete search system
}
\item [\checkbox]  An Introduction to Information Retrieval | \textbf{ 8. Evaluation in information retrieval
}
\item [\checkbox]  An Introduction to Information Retrieval | \textbf{ 9. Relevance feedback and query expansion
}
\item [\checkbox]  An Introduction to Information Retrieval | \textbf{ 10. XML retrieval
}
\item [\checkbox]  An Introduction to Information Retrieval | \textbf{ 11. Probabilistic information retrieval
}
\item [\checkbox]  An Introduction to Information Retrieval | \textbf{ 12. Language models for information retrieval
}
\item [\checkbox]  An Introduction to Information Retrieval | \textbf{ 13. Text classification and Naive Bayes
}
\item [\checkbox]  An Introduction to Information Retrieval | \textbf{ 14. Vector space classification
}
\item [\checkbox]  An Introduction to Information Retrieval | \textbf{ 15. Support vector machines and machine learning on documents
}
\item [\checkbox]  An Introduction to Information Retrieval | \textbf{ 16. Flat clustering
}
\item [\checkbox]  An Introduction to Information Retrieval | \textbf{ 17. Hierarchical clustering
}
\item [\checkbox]  An Introduction to Information Retrieval | \textbf{ 18. Matrix decompositions and latent semantic indexing
}
\item [\checkbox]  An Introduction to Information Retrieval | \textbf{ 19. Web search basics
}
\item [\checkbox]  An Introduction to Information Retrieval | \textbf{ 20. Web crawling and indexes
}
\item [\checkbox]  An Introduction to Information Retrieval | \textbf{ 21. Link analysis
}
\item [\checkbox]  Foundations of Statistical Natural Language Processing | \textbf{ 1. Introduction
}
\item [\checkbox]  Foundations of Statistical Natural Language Processing | \textbf{ 2.1. Elementary Probability Theory
}
\item [\checkbox]  Foundations of Statistical Natural Language Processing | \textbf{ 2.2. Essential Information Theory
}
\item [\checkbox]  Foundations of Statistical Natural Language Processing | \textbf{ 3. Linguistic Essentials
}
\item [\checkbox]  Foundations of Statistical Natural Language Processing | \textbf{ 4. Corpus-Based Work
}
\item [\checkbox]  Foundations of Statistical Natural Language Processing | \textbf{ 5. Collocations
}
\item [\checkbox]  Foundations of Statistical Natural Language Processing | \textbf{ 6. Statistical Inference: n-gram Models over Sparse Data
}
\item [\checkbox]  Foundations of Statistical Natural Language Processing | \textbf{ 7. Word Sense Disambiguation
}
\item [\checkbox]  Foundations of Statistical Natural Language Processing | \textbf{ 8. Lexical Acquisition
}
\item [\checkbox]  Foundations of Statistical Natural Language Processing | \textbf{ 9. Markov Models
}
\item [\checkbox]  Foundations of Statistical Natural Language Processing | \textbf{ 10. Part-of-Speech Tagging
}
\item [\checkbox]  Foundations of Statistical Natural Language Processing | \textbf{ 11. Probabilistic Context Free Grammars
}
\item [\checkbox]  Foundations of Statistical Natural Language Processing | \textbf{ 12. Probabilistic Parsing
}
\item [\checkbox]  Foundations of Statistical Natural Language Processing | \textbf{ 13. Statistical Alignment and Machine Translation
}
\item [\checkbox]  Foundations of Statistical Natural Language Processing | \textbf{ 14. Clustering
}
\item [\checkbox]  Foundations of Statistical Natural Language Processing | \textbf{ 15. Topics in Information Retrieval
}
\item [\checkbox]  Foundations of Statistical Natural Language Processing | \textbf{ 16. Text Categorization
}
\item [\checkbox]  Using Praat for Linguistic Research (Styler) | \textbf{ 2. Introduction
}
\item [\checkbox]  Using Praat for Linguistic Research (Styler) | \textbf{ 3. About Praat
}
\item [\checkbox]  Using Praat for Linguistic Research (Styler) | \textbf{ 4. Recording Sounds
}
\item [\checkbox]  Using Praat for Linguistic Research (Styler) | \textbf{ 5. Opening and Saving Files
}
\item [\checkbox]  Using Praat for Linguistic Research (Styler) | \textbf{ 6. Phonetic Measurement and Analysis in Praat
}
\item [\checkbox]  Using Praat for Linguistic Research (Styler) | \textbf{ 7. Creating and Manipulating Sound Files in Praat
}
\item [\checkbox]  Using Praat for Linguistic Research (Styler) | \textbf{ 8. Exporting Images for Use and Publication
}
\item [\checkbox]  Using Praat for Linguistic Research (Styler) | \textbf{ 9. Annotating Soundfiles (Praat TextGrids)
}
\item [\checkbox]  Using Praat for Linguistic Research (Styler) | \textbf{ 10. Using Log Files
}
\item [\checkbox]  Using Praat for Linguistic Research (Styler) | \textbf{ 11. Scripting in Praat
}
\item [\checkbox]  Using Praat for Linguistic Research (Styler) | \textbf{ 12. Advanced Techniques
}
\item [\checkbox]  Using Praat for Linguistic Research (Styler) | \textbf{ 13. Conclusion
}
\item [\checkbox]  Routledge Handbook of Semantics | \textbf{ 1. Descriptive Externalism in Semantics
}
\item [\checkbox]  Routledge Handbook of Semantics | \textbf{ 2. Internalist Semantics: meaning, conceptualization and expression
}
\item [\checkbox]  Routledge Handbook of Semantics | \textbf{ 3. A History of Semantics
}
\item [\checkbox]  Routledge Handbook of Semantics | \textbf{ 4. Foundations of Formal Semantics
}
\item [\checkbox]  Routledge Handbook of Semantics | \textbf{ 5. Cognitive Semantics
}
\item [\checkbox]  Routledge Handbook of Semantics | \textbf{ 6. Corpus Semantics
}
\item [\checkbox]  Routledge Handbook of Semantics | \textbf{ 7. Categories, prototypes and exemplars
}
\item [\checkbox]  Routledge Handbook of Semantics | \textbf{ 8. Embodiment, simulation and meaning
}
\item [\checkbox]  Routledge Handbook of Semantics | \textbf{ 9. Linguistic Relativity
}
\item [\checkbox]  Routledge Handbook of Semantics | \textbf{ Semantics and Pragmatics
}
\item [\checkbox]  Routledge Handbook of Semantics | \textbf{ 11. Contextual Adjustment of Meaning
}
\item [\checkbox]  Routledge Handbook of Semantics | \textbf{ 12. Lexical Decomposition
}
\item [\checkbox]  Routledge Handbook of Semantics | \textbf{ 13. Sense Individuation
}
\item [\checkbox]  Routledge Handbook of Semantics | \textbf{ 14. Sense Relations
}
\item [\checkbox]  Routledge Handbook of Semantics | \textbf{ 15. Semantic Shift
}
\item [\checkbox]  Routledge Handbook of Semantics | \textbf{ 16. The Semantics of Nominals
}
\item [\checkbox]  Routledge Handbook of Semantics | \textbf{ 17. Negation and Polarity
}
\item [\checkbox]  Routledge Handbook of Semantics | \textbf{ 18. Varieties of Quantifiation
}
\item [\checkbox]  Routledge Handbook of Semantics | \textbf{ 19. Lexical and Grammatical Aspect
}
\item [\checkbox]  Routledge Handbook of Semantics | \textbf{ 20. Tense
}
\item [\checkbox]  Routledge Handbook of Semantics | \textbf{ 21. Modality
}
\item [\checkbox]  Routledge Handbook of Semantics | \textbf{ 22. Event Semantics
}
\item [\checkbox]  Routledge Handbook of Semantics | \textbf{ 23. Participant Roles
}
\item [\checkbox]  Routledge Handbook of Semantics | \textbf{ 24. Compositionality
}
\item [\checkbox]  Routledge Handbook of Semantics | \textbf{ 25. The Semantics of Lexical Typology
}
\item [\checkbox]  Routledge Handbook of Semantics | \textbf{ 26. Acquisition of Meaning
}
\item [\checkbox]  Routledge Handbook of Semantics | \textbf{ 27. Expressives
}
\item [\checkbox]  Routledge Handbook of Semantics | \textbf{ 28. Interpretative Semantics
}
\item [\checkbox]  Routledge Handbook of Semantics | \textbf{ 29. Semantic Processing
}
\item [\checkbox]  Linguistic Fundamentals for NLP | \textbf{ 0. Knowing about linguistic structure is important for feature design and error analysis in NLP
}
\item [\checkbox]  Linguistic Fundamentals for NLP | \textbf{ 1. Morphosyntax is the difference between a sentence and a bag of words
}
\item [\checkbox]  Linguistic Fundamentals for NLP | \textbf{ 2. The morphosyntax of a language is the constraints that it places on how words can be combined both in form and in the resulting meaning
}
\item [\checkbox]  Linguistic Fundamentals for NLP | \textbf{ 3. Languages use morphology and syntax to indicate who did what to whom, and make use of a range of strategies to do so .
}
\item [\checkbox]  Linguistic Fundamentals for NLP | \textbf{ 4. Languages can be classified ‘genetically’, areally, or typologically.
}
\item [\checkbox]  Linguistic Fundamentals for NLP | \textbf{ 5. There are approximately 7,000 known living languages distributed across 128 language families
}
\item [\checkbox]  Linguistic Fundamentals for NLP | \textbf{ 6. Incorporating information about linguistic structure and variation can make for more cross-linguistically portable NLP systems
}
\item [\checkbox]  Linguistic Fundamentals for NLP | \textbf{ 7. Morphemes are the smallest meaningful units of language, usually consisting of a sequence of phones paired with concrete meaning
}
\item [\checkbox]  Linguistic Fundamentals for NLP | \textbf{ 8. The phones making up a morpheme don’t have to be contiguous
}
\item [\checkbox]  Linguistic Fundamentals for NLP | \textbf{ 9. The form of a morpheme doesn’t have to consist of phones
}
\item [\checkbox]  Linguistic Fundamentals for NLP | \textbf{ 10. The form of a morpheme can be null
}
\item [\checkbox]  Linguistic Fundamentals for NLP | \textbf{ 11. Root morphemes convey core lexical meaning
}
\item [\checkbox]  Linguistic Fundamentals for NLP | \textbf{ 12. Derivational affixes can change lexical meaning
}
\item [\checkbox]  Linguistic Fundamentals for NLP | \textbf{ 13. Root+derivational affix combinations can have idiosyncratic meanings
}
\item [\checkbox]  Linguistic Fundamentals for NLP | \textbf{ 14. Inflectional affixes add syntactically or semantically relevant features
}
\item [\checkbox]  Linguistic Fundamentals for NLP | \textbf{ 15. Morphemes can be ambiguous and/or underspecified in their meaning
}
\item [\checkbox]  Linguistic Fundamentals for NLP | \textbf{ 16. The notion ‘word’ can be contentious in many languages
}
\item [\checkbox]  Linguistic Fundamentals for NLP | \textbf{ 17. Constraints on order operate differently between words than they do between morphemes
}
\item [\checkbox]  Linguistic Fundamentals for NLP | \textbf{ 18. The distinction between words and morphemes is blurred by processes of language change
}
\item [\checkbox]  Linguistic Fundamentals for NLP | \textbf{ 19. A clitic is a linguistic element which is syntactically independent but phonologically dependent
}
\item [\checkbox]  Linguistic Fundamentals for NLP | \textbf{ 20. Languages vary in how many morphemes they have per word (on average and maximally)
}
\item [\checkbox]  Linguistic Fundamentals for NLP | \textbf{ 21. Languages vary in whether they are primarily prefixing or suffixing in their morphology
}
\item [\checkbox]  Linguistic Fundamentals for NLP | \textbf{ 22. Languages vary in how easy it is to find the boundaries between morphemes within a word
}
\item [\checkbox]  Linguistic Fundamentals for NLP | \textbf{ 23. The morphophonology of a language describes the way in which surface forms are related to underlying, abstract sequences of morphemes
}
\item [\checkbox]  Linguistic Fundamentals for NLP | \textbf{ 24. The form of a morpheme (root or affix) can be sensitive to its phonological context
}
\item [\checkbox]  Linguistic Fundamentals for NLP | \textbf{ 25. The form of a morpheme (root or affix) can be sensitive to its morphological context
}
\item [\checkbox]  Linguistic Fundamentals for NLP | \textbf{ 26. Suppletive forms replace a stem+affix combination with a wholly different word
}
\item [\checkbox]  Linguistic Fundamentals for NLP | \textbf{ 27. Alphabetic and syllabic writing systems tend to reflect some but not all phonological processes
}
\item [\checkbox]  Linguistic Fundamentals for NLP | \textbf{ 28. The morphosyntax of a language describes how the morphemes in a word affect its combinatoric potential
}
\item [\checkbox]  Linguistic Fundamentals for NLP | \textbf{ 29. Morphological features associated with verbs and adjectives (and sometimes nouns) can include information about tense, aspect and mood
}
\item [\checkbox]  Linguistic Fundamentals for NLP | \textbf{ 30. Morphological features associated with nouns can contribute information about person, number and gender
}
\item [\checkbox]  Linguistic Fundamentals for NLP | \textbf{ 31. Morphological features associated with nouns can contribute information about case
}
\item [\checkbox]  Linguistic Fundamentals for NLP | \textbf{ 32. Negation can be marked morphologically
}
\item [\checkbox]  Linguistic Fundamentals for NLP | \textbf{ 33. Evidentiality can be marked morphologically
}
\item [\checkbox]  Linguistic Fundamentals for NLP | \textbf{ 34. Definiteness can be marked morphologically
}
\item [\checkbox]  Linguistic Fundamentals for NLP | \textbf{ 35. Honorifics can be marked morphologically
}
\item [\checkbox]  Linguistic Fundamentals for NLP | \textbf{ 36. Possessives can be marked morphologically
}
\item [\checkbox]  Linguistic Fundamentals for NLP | \textbf{ 37. Yet more grammatical notions can be marked morphologically
}
\item [\checkbox]  Linguistic Fundamentals for NLP | \textbf{ 38. When an inflectional category is marked on multiple elements of sentence or phrase, it is usually considered to belong to one element and to express agreement on the others
}
\item [\checkbox]  Linguistic Fundamentals for NLP | \textbf{ 39. Verbs commonly agree in person/number/gender with one or more arguments
}
\item [\checkbox]  Linguistic Fundamentals for NLP | \textbf{ 40. Determiners and adjectives commonly agree with nouns in number, gender and case
}
\item [\checkbox]  Linguistic Fundamentals for NLP | \textbf{ 41. Agreement can be with a feature that is not overtly marked on the controller
}
\item [\checkbox]  Linguistic Fundamentals for NLP | \textbf{ 42. Languages vary in which kinds of information they mark morphologically
}
\item [\checkbox]  Linguistic Fundamentals for NLP | \textbf{ 43. Languages vary in how many distinctions they draw within each morphologically marked category
}
\item [\checkbox]  Linguistic Fundamentals for NLP | \textbf{ 44. Syntax places constraints on possible sentences
}
\item [\checkbox]  Linguistic Fundamentals for NLP | \textbf{ 45. Syntax provides scaffolding for semantic composition
}
\item [\checkbox]  Linguistic Fundamentals for NLP | \textbf{ 46. Constraints ruling out some strings as ungrammatical usually also constrain the range of possible semantic interpretations of other strings
}
\item [\checkbox]  Linguistic Fundamentals for NLP | \textbf{ 47. Parts of speech can be defined distributionally (in terms of morphology and syntax)
}
\item [\checkbox]  Linguistic Fundamentals for NLP | \textbf{ 48. Parts of speech can also be defined functionally (but not metaphysically)
}
\item [\checkbox]  Linguistic Fundamentals for NLP | \textbf{ 49. There is no one universal set of parts of speech, even among the major categories
}
\item [\checkbox]  Linguistic Fundamentals for NLP | \textbf{ 50. Part of speech extends to phrasal constituents
}
\item [\checkbox]  Linguistic Fundamentals for NLP | \textbf{ 51. Words within sentences form intermediate groupings called constituents
}
\item [\checkbox]  Linguistic Fundamentals for NLP | \textbf{ 52. A syntactic head determines the internal structure and external distribution of the constituent it projects
}
\item [\checkbox]  Linguistic Fundamentals for NLP | \textbf{ 53. Syntactic dependents can be classified as arguments and adjuncts
}
\item [\checkbox]  Linguistic Fundamentals for NLP | \textbf{ 54. The number of semantic arguments provided for by a head is a fundamental lexical property
}
\item [\checkbox]  Linguistic Fundamentals for NLP | \textbf{ 55. In many (perhaps all) languages, (some) arguments can be left unexpressed
}
\item [\checkbox]  Linguistic Fundamentals for NLP | \textbf{ 56. Words from different parts of speech can serve as heads selecting arguments
}
\item [\checkbox]  Linguistic Fundamentals for NLP | \textbf{ 57. Adjuncts are not required by heads and generally can iterate
}
\item [\checkbox]  Linguistic Fundamentals for NLP | \textbf{ 58. Adjuncts are syntactically dependents but semantically introduce predicates with take the syntactic head as an argument
}
\item [\checkbox]  Linguistic Fundamentals for NLP | \textbf{ 59. Obligatoriness can be used as a test to distinguish arguments from adjuncts
}
\item [\checkbox]  Linguistic Fundamentals for NLP | \textbf{ 60. Entailment can be used as a test to distinguish arguments from adjuncts
}
\item [\checkbox]  Linguistic Fundamentals for NLP | \textbf{ 61. Adjuncts can be single words, phrases, or clauses
}
\item [\checkbox]  Linguistic Fundamentals for NLP | \textbf{ 62. Adjuncts can modify nominal constituents
}
\item [\checkbox]  Linguistic Fundamentals for NLP | \textbf{ 63. Adjuncts can modify verbal constituents
}
\item [\checkbox]  Linguistic Fundamentals for NLP | \textbf{ 64. Adjuncts can modify other types of constituents
}
\item [\checkbox]  Linguistic Fundamentals for NLP | \textbf{ 65. Adjuncts express a wide range of meanings
}
\item [\checkbox]  Linguistic Fundamentals for NLP | \textbf{ 66. The potential to be a modifier is inherent to the syntax of a constituent
}
\item [\checkbox]  Linguistic Fundamentals for NLP | \textbf{ 67. Just about anything can be an argument, for some head
}
\item [\checkbox]  Linguistic Fundamentals for NLP | \textbf{ 68. There is no agreed upon universal set of semantic roles, even for one language; nonetheless, arguments can be roughly categorized semantically
}
\item [\checkbox]  Linguistic Fundamentals for NLP | \textbf{ 69. Arguments can also be categorized syntactically, though again there may not be universal syntactic argument types
}
\item [\checkbox]  Linguistic Fundamentals for NLP | \textbf{ 70. A subject is the distinguished argument of a predicate and may be the only one to display certain grammatical properties
}
\item [\checkbox]  Linguistic Fundamentals for NLP | \textbf{ 71. Arguments can generally be arranged in order of obliqueness
}
\item [\checkbox]  Linguistic Fundamentals for NLP | \textbf{ 72. Clauses, finite or non-finite, open or closed, can also be arguments
}
\item [\checkbox]  Linguistic Fundamentals for NLP | \textbf{ 73. Syntactic and semantic arguments aren’t the same, though they often stand in regular relations to each other
}
\item [\checkbox]  Linguistic Fundamentals for NLP | \textbf{ 74. For many applications, it is not the surface (syntactic) relations, but the deep (semantic) dependencies that matter
}
\item [\checkbox]  Linguistic Fundamentals for NLP | \textbf{ 75. Lexical items map semantic roles to grammatical functions
}
\item [\checkbox]  Linguistic Fundamentals for NLP | \textbf{ 76. Syntactic phenomena are sensitive to grammatical functions
}
\item [\checkbox]  Linguistic Fundamentals for NLP | \textbf{ 77. Identifying the grammatical function of a constituent can help us understand its semantic role with respect to the head 91
}
\item [\checkbox]  Linguistic Fundamentals for NLP | \textbf{ 78. Some languages identify grammatical functions primarily through word order
}
\item [\checkbox]  Linguistic Fundamentals for NLP | \textbf{ 79. Some languages identify grammatical functions through agreement
}
\item [\checkbox]  Linguistic Fundamentals for NLP | \textbf{ 80. Some languages identify grammatical functions through case marking
}
\item [\checkbox]  Linguistic Fundamentals for NLP | \textbf{ 81. Marking of dependencies on heads is more common cross-linguistically than marking on dependents
}
\item [\checkbox]  Linguistic Fundamentals for NLP | \textbf{ 82. Some morphosyntactic phenomena rearrange the lexical mapping
}
\item [\checkbox]  Linguistic Fundamentals for NLP | \textbf{ 83. There are a variety of syntactic phenomena which obscure the relationship between syntactic and semantic arguments
}
\item [\checkbox]  Linguistic Fundamentals for NLP | \textbf{ 84. Passive is a grammatical process which demotes the subject to oblique status, making room for the next most prominent argument to appear as the subject
}
\item [\checkbox]  Linguistic Fundamentals for NLP | \textbf{ 85. Related constructions include anti-passives, impersonal passives, and middles
}
\item [\checkbox]  Linguistic Fundamentals for NLP | \textbf{ 86. English dative shift also affects the mapping between syntactic and semantic arguments
}
\item [\checkbox]  Linguistic Fundamentals for NLP | \textbf{ 87. Morphological causatives add an argument and change the expression of at least one other
}
\item [\checkbox]  Linguistic Fundamentals for NLP | \textbf{ 88. Many (all?) languages have semantically empty words which serve as syntactic glue
}
\item [\checkbox]  Linguistic Fundamentals for NLP | \textbf{ 89. Expletives are constituents that can fill syntactic argument positions that don’t have any associated semantic role
}
\item [\checkbox]  Linguistic Fundamentals for NLP | \textbf{ 90. Raising verbs provide a syntactic argument position with no (local) semantic role, and relate it to a syntactic argument position of another predicate
}
\item [\checkbox]  Linguistic Fundamentals for NLP | \textbf{ 91. Control verbs provide a syntactic and semantic argument which is related to a syntactic argument position of another predicate
}
\item [\checkbox]  Linguistic Fundamentals for NLP | \textbf{ 92. In complex predicate constructions the arguments of a clause are licensed by multiple predicates working together
}
\item [\checkbox]  Linguistic Fundamentals for NLP | \textbf{ 93. Coordinated structures can lead to one-to-many and many-to-one dependency relations
}
\item [\checkbox]  Linguistic Fundamentals for NLP | \textbf{ 94. Long-distance dependencies separate arguments/adjuncts from their associated heads
}
\item [\checkbox]  Linguistic Fundamentals for NLP | \textbf{ 95. Some languages allow adnominal adjuncts to be separated from their head nouns
}
\item [\checkbox]  Linguistic Fundamentals for NLP | \textbf{ 96. Many (all?) languages can drop arguments, but permissible argument drop varies by word class and by language
}
\item [\checkbox]  Linguistic Fundamentals for NLP | \textbf{ 97. The referent of a dropped argument can be definite or indefinite, depending on the lexical item or construction licensing the argument drop
}
\item [\checkbox]  Linguistic Fundamentals for NLP | \textbf{ 98. Morphological analyzers map surface strings (words in standard orthography) to regularized strings of morphemes or morphological features
}
\item [\checkbox]  Linguistic Fundamentals for NLP | \textbf{ 99. ‘Deep’ syntactic parsers map surface strings (sentences) to semantic structures, including semantic dependencies
}
\item [\checkbox]  Linguistic Fundamentals for NLP | \textbf{ 100. Typological databases summarize properties of languages at a high level
}
\item [\checkbox]  Linguistic Fundamentals for NLP | \textbf{ A. Summary
}
\item [\checkbox]  Linguistic Fundamentals for NLP | \textbf{ B. Grams used in IGT
}
\item [\checkbox]  Cambridge Handbook of Formal Semantics | \textbf{ 1. Formal semantics
}
\item [\checkbox]  Cambridge Handbook of Formal Semantics | \textbf{ 2. Lexical semantics
}
\item [\checkbox]  Cambridge Handbook of Formal Semantics | \textbf{ 3. Sentential semantics
}
\item [\checkbox]  Cambridge Handbook of Formal Semantics | \textbf{ 4. Discourse semantics
}
\item [\checkbox]  Cambridge Handbook of Formal Semantics | \textbf{ 5. Semantics of dialogue
}
\item [\checkbox]  Cambridge Handbook of Formal Semantics | \textbf{ 6. Reference
}
\item [\checkbox]  Cambridge Handbook of Formal Semantics | \textbf{ 7. Generalized quantifiers
}
\item [\checkbox]  Cambridge Handbook of Formal Semantics | \textbf{ 8. Indefinites
}
\item [\checkbox]  Cambridge Handbook of Formal Semantics | \textbf{ 9. Plurality
}
\item [\checkbox]  Cambridge Handbook of Formal Semantics | \textbf{ 10. Genericity
}
\item [\checkbox]  Cambridge Handbook of Formal Semantics | \textbf{ 11. Tense
}
\item [\checkbox]  Cambridge Handbook of Formal Semantics | \textbf{ 12. Aspect
}
\item [\checkbox]  Cambridge Handbook of Formal Semantics | \textbf{ 13. Mereology
}
\item [\checkbox]  Cambridge Handbook of Formal Semantics | \textbf{ 14. Vagueness
}
\item [\checkbox]  Cambridge Handbook of Formal Semantics | \textbf{ 15. Modification
}
\item [\checkbox]  Cambridge Handbook of Formal Semantics | \textbf{ 16. Negation
}
\item [\checkbox]  Cambridge Handbook of Formal Semantics | \textbf{ 17. Conditionals
}
\item [\checkbox]  Cambridge Handbook of Formal Semantics | \textbf{ 18. Modality
}
\item [\checkbox]  Cambridge Handbook of Formal Semantics | \textbf{ 19. Questions
}
\item [\checkbox]  Cambridge Handbook of Formal Semantics | \textbf{ 20. Imperatives
}
\item [\checkbox]  Cambridge Handbook of Formal Semantics | \textbf{ 21. The syntax–semantics interface
}
\item [\checkbox]  Cambridge Handbook of Formal Semantics | \textbf{ 22. The semantics–pragmatics interface
}
\item [\checkbox]  Cambridge Handbook of Formal Semantics | \textbf{ 23. Information structure
}
\item [\checkbox]  Cambridge Handbook of Formal Semantics | \textbf{ 24. Semantics and cognition
}
\item [\checkbox]  Cambridge Handbook of Formal Semantics | \textbf{ 25. Semantics and computation
}
\item [\checkbox]  Elements of Formal Semantics | \textbf{ 1 Introduction
}
\item [\checkbox]  Elements of Formal Semantics | \textbf{ 2 Meaning and Form
}
\item [\checkbox]  Elements of Formal Semantics | \textbf{ 3 Types and Meaning Composition
}
\item [\checkbox]  Elements of Formal Semantics | \textbf{ 4 Quantified Noun Phrases
}
\item [\checkbox]  Elements of Formal Semantics | \textbf{ 5 Long-Distance Meaning Relationships
}
\item [\checkbox]  Elements of Formal Semantics | \textbf{ 6 Intensionality and Possible Worlds
}
\item [\checkbox]  Elements of Formal Semantics | \textbf{ 7 Conclusion and Further Topics
}
\item [\checkbox]  Essentials of Linguistics (Anderson) | \textbf{ 1. 1.1 Linguistics is Science
}
\item [\checkbox]  Essentials of Linguistics (Anderson) | \textbf{ 2. 1.2 Mental Grammar
}
\item [\checkbox]  Essentials of Linguistics (Anderson) | \textbf{ 3. 1.3 Creativity and Generativity
}
\item [\checkbox]  Essentials of Linguistics (Anderson) | \textbf{ 4. 1.4 Fundamental Properties of Language
}
\item [\checkbox]  Essentials of Linguistics (Anderson) | \textbf{ 5. Practice Time
}
\item [\checkbox]  Essentials of Linguistics (Anderson) | \textbf{ 6. Summary
}
\item [\checkbox]  Essentials of Linguistics (Anderson) | \textbf{ 7. 2.1 How Humans Produce Speech
}
\item [\checkbox]  Essentials of Linguistics (Anderson) | \textbf{ 8. 2.2 Articulators
}
\item [\checkbox]  Essentials of Linguistics (Anderson) | \textbf{ 9. 2.3 Describing Speech Sounds: the IPA
}
\item [\checkbox]  Essentials of Linguistics (Anderson) | \textbf{ 10. 2.4 IPA symbols and speech sounds
}
\item [\checkbox]  Essentials of Linguistics (Anderson) | \textbf{ 11. 2.5 Sonority, Consonants, and Vowels
}
\item [\checkbox]  Essentials of Linguistics (Anderson) | \textbf{ 12. 2.6 Classifying Consonants
}
\item [\checkbox]  Essentials of Linguistics (Anderson) | \textbf{ 13. 2.7 Classifying Vowels
}
\item [\checkbox]  Essentials of Linguistics (Anderson) | \textbf{ 14. 2.8 Diphthongs
}
\item [\checkbox]  Essentials of Linguistics (Anderson) | \textbf{ 15. 2.9 Various Accents of English
}
\item [\checkbox]  Essentials of Linguistics (Anderson) | \textbf{ 16. Practice Time
}
\item [\checkbox]  Essentials of Linguistics (Anderson) | \textbf{ 17. Summary
}
\item [\checkbox]  Essentials of Linguistics (Anderson) | \textbf{ 18. 3.1 Broad and Narrow Transcription
}
\item [\checkbox]  Essentials of Linguistics (Anderson) | \textbf{ 19. 3.2 IPA for Canadian English
}
\item [\checkbox]  Essentials of Linguistics (Anderson) | \textbf{ 20. 3.3 Syllabic Consonants
}
\item [\checkbox]  Essentials of Linguistics (Anderson) | \textbf{ 21. 3.4 Aspirated Stops in English
}
\item [\checkbox]  Essentials of Linguistics (Anderson) | \textbf{ 22. 3.5 Articulatory Processes: Assimilation
}
\item [\checkbox]  Essentials of Linguistics (Anderson) | \textbf{ 23. 3.6 Other Articulatory Processes
}
\item [\checkbox]  Essentials of Linguistics (Anderson) | \textbf{ 24. 3.7 Suprasegmentals
}
\item [\checkbox]  Essentials of Linguistics (Anderson) | \textbf{ 25. 3.8 Transcribing Casual Speech
}
\item [\checkbox]  Essentials of Linguistics (Anderson) | \textbf{ 26. Practice Time
}
\item [\checkbox]  Essentials of Linguistics (Anderson) | \textbf{ 27. Summary
}
\item [\checkbox]  Essentials of Linguistics (Anderson) | \textbf{ 28. 4.1 Phonemes and Contrast
}
\item [\checkbox]  Essentials of Linguistics (Anderson) | \textbf{ 29. 4.2 Allophones and Predictable Variation
}
\item [\checkbox]  Essentials of Linguistics (Anderson) | \textbf{ 30. 4.3 Phonetic Segments and Features
}
\item [\checkbox]  Essentials of Linguistics (Anderson) | \textbf{ 31. 4.4 Natural Classes
}
\item [\checkbox]  Essentials of Linguistics (Anderson) | \textbf{ 32. 4.5 Phonological Derivations
}
\item [\checkbox]  Essentials of Linguistics (Anderson) | \textbf{ 33. Practice Time
}
\item [\checkbox]  Essentials of Linguistics (Anderson) | \textbf{ 34. Summary
}
\item [\checkbox]  Essentials of Linguistics (Anderson) | \textbf{ 35. 5.1 How Babies Learn the Phoneme Categories of Their Language
}
\item [\checkbox]  Essentials of Linguistics (Anderson) | \textbf{ 36. 5.2 How Adults Learn the Phoneme Categories in a New Language
}
\item [\checkbox]  Essentials of Linguistics (Anderson) | \textbf{ 37. Practice Time
}
\item [\checkbox]  Essentials of Linguistics (Anderson) | \textbf{ 38. Summary
}
\item [\checkbox]  Essentials of Linguistics (Anderson) | \textbf{ 39. 6.1 Words and Morphemes
}
\item [\checkbox]  Essentials of Linguistics (Anderson) | \textbf{ 40. 6.2 Allomorphs
}
\item [\checkbox]  Essentials of Linguistics (Anderson) | \textbf{ 41. 6.3 Inflectional Morphology
}
\item [\checkbox]  Essentials of Linguistics (Anderson) | \textbf{ 42. 6.4 Derivational Morphology
}
\item [\checkbox]  Essentials of Linguistics (Anderson) | \textbf{ 43. 6.5 Inflectional Morphology in Some Indigenous Languages
}
\item [\checkbox]  Essentials of Linguistics (Anderson) | \textbf{ 44. Practice Time
}
\item [\checkbox]  Essentials of Linguistics (Anderson) | \textbf{ 45. Summary
}
\item [\checkbox]  Essentials of Linguistics (Anderson) | \textbf{ 46. 7.1 Nouns, Verbs and Adjectives: Open Class Categories
}
\item [\checkbox]  Essentials of Linguistics (Anderson) | \textbf{ 47. 7.2 Compound Words
}
\item [\checkbox]  Essentials of Linguistics (Anderson) | \textbf{ 48. 7.3 Closed Class Categories (Function Words)
}
\item [\checkbox]  Essentials of Linguistics (Anderson) | \textbf{ 49. 7.4 Auxiliaries
}
\item [\checkbox]  Essentials of Linguistics (Anderson) | \textbf{ 50. 7.5 Neurolinguistics: Syntactic Category Differences
}
\item [\checkbox]  Essentials of Linguistics (Anderson) | \textbf{ 51. Practice Time
}
\item [\checkbox]  Essentials of Linguistics (Anderson) | \textbf{ 52. Summary
}
\item [\checkbox]  Essentials of Linguistics (Anderson) | \textbf{ 53. 8.1 Tree Diagrams
}
\item [\checkbox]  Essentials of Linguistics (Anderson) | \textbf{ 54. 8.2 X-bar Phrase Structure
}
\item [\checkbox]  Essentials of Linguistics (Anderson) | \textbf{ 55. 8.3 Constituents
}
\item [\checkbox]  Essentials of Linguistics (Anderson) | \textbf{ 56. 8.4 Sentences are Phrases
}
\item [\checkbox]  Essentials of Linguistics (Anderson) | \textbf{ 57. 8.5 English Verb Forms
}
\item [\checkbox]  Essentials of Linguistics (Anderson) | \textbf{ 58. 8.6 Subcategories
}
\item [\checkbox]  Essentials of Linguistics (Anderson) | \textbf{ 59. 8.7 Grammatical Roles
}
\item [\checkbox]  Essentials of Linguistics (Anderson) | \textbf{ 60. 8.8 Adjuncts
}
\item [\checkbox]  Essentials of Linguistics (Anderson) | \textbf{ 61. 8.9 Move
}
\item [\checkbox]  Essentials of Linguistics (Anderson) | \textbf{ 62. 8.10 Wh-Movement
}
\item [\checkbox]  Essentials of Linguistics (Anderson) | \textbf{ 63. 8.11 Do-Support
}
\item [\checkbox]  Essentials of Linguistics (Anderson) | \textbf{ 64. 8.12 Psycholinguistics: Traces in the Mind
}
\item [\checkbox]  Essentials of Linguistics (Anderson) | \textbf{ 65. Practice Time
}
\item [\checkbox]  Essentials of Linguistics (Anderson) | \textbf{ 66. Summary
}
\item [\checkbox]  Essentials of Linguistics (Anderson) | \textbf{ 67. 9.1 Ambiguity
}
\item [\checkbox]  Essentials of Linguistics (Anderson) | \textbf{ 68. 9.2 Events, Participants, and Thematic Roles
}
\item [\checkbox]  Essentials of Linguistics (Anderson) | \textbf{ 69. 9.3 Thematic Roles and Passive Sentences
}
\item [\checkbox]  Essentials of Linguistics (Anderson) | \textbf{ 70. 9.4 Neurolinguistics: Using EEG to Investigate Syntax and Semantics
}
\item [\checkbox]  Essentials of Linguistics (Anderson) | \textbf{ 71. 9.5 Neurolinguistics and Second Language Learning
}
\item [\checkbox]  Essentials of Linguistics (Anderson) | \textbf{ 72. Practice Time
}
\item [\checkbox]  Essentials of Linguistics (Anderson) | \textbf{ 73. Summary
}
\item [\checkbox]  Essentials of Linguistics (Anderson) | \textbf{ 74. 10.1 Elements of Word Meaning: Intensions and Extensions
}
\item [\checkbox]  Essentials of Linguistics (Anderson) | \textbf{ 75. 10.2 Intensions in the Mind
}
\item [\checkbox]  Essentials of Linguistics (Anderson) | \textbf{ 76. 10.3 Psycholinguistics of Word Meanings
}
\item [\checkbox]  Essentials of Linguistics (Anderson) | \textbf{ 77. Practice Time
}
\item [\checkbox]  Essentials of Linguistics (Anderson) | \textbf{ 78. Summary
}
\item [\checkbox]  Essentials of Linguistics (Anderson) | \textbf{ 79. 11.1 Indigenous Languages and the Legacy of Residential Schools
}
\item [\checkbox]  Essentials of Linguistics (Anderson) | \textbf{ 80. 11.2 Preserving Mohawk
}
\item [\checkbox]  Essentials of Linguistics (Anderson) | \textbf{ 81. 11.3 Learning Mohawk
}
\item [\checkbox]  Essentials of Linguistics (Anderson) | \textbf{ 82. 11.4 Mohawk Culture and Language
}
\item [\checkbox]  Essentials of Linguistics (Anderson) | \textbf{ 83. 11.5 Creating Materials for Teaching Mohawk
}
\item [\checkbox]  Essentials of Linguistics (Anderson) | \textbf{ 84. 11.6 Speaking Mohawk and Reconciliation
}
\item [\checkbox]  Essentials of Linguistics (Anderson) | \textbf{ 85. 11.7 The Future of Indigenous Languages in Canada
}
\item [\checkbox]  Essentials of Linguistics (Anderson) | \textbf{ 86. Practice Time
}
\item [\checkbox]  Essentials of Linguistics (Anderson) | \textbf{ 87. Summary
}
\item [\checkbox]  An Introduction to Language (Fromkin) | \textbf{ 1. WHAT IS LANGUAGE?
}
\item [\checkbox]  An Introduction to Language (Fromkin) | \textbf{ 2. MORPHOLOGY: THE WORDS OF LANGUAGE.
}
\item [\checkbox]  An Introduction to Language (Fromkin) | \textbf{ 3. SYNTAX: THE SENTENCE PATTERNS OF LANGUAGE.
}
\item [\checkbox]  An Introduction to Language (Fromkin) | \textbf{ 4. THE MEANING OF LANGUAGE.
}
\item [\checkbox]  An Introduction to Language (Fromkin) | \textbf{ 5. PHONETICS: THE SOUNDS OF LANGUAGE.
}
\item [\checkbox]  An Introduction to Language (Fromkin) | \textbf{ 6. PHONOLOGY: THE SOUND OF LANGUAGE.
}
\item [\checkbox]  An Introduction to Language (Fromkin) | \textbf{ 7. LANGUAGE IN SOCIETY.
}
\item [\checkbox]  An Introduction to Language (Fromkin) | \textbf{ 8. LANGUAGE CHANGE: THE SYLLABLES OF TIME.
}
\item [\checkbox]  An Introduction to Language (Fromkin) | \textbf{ 9. LANGUAGE ACQUISITION.
}
\item [\checkbox]  An Introduction to Language (Fromkin) | \textbf{ 10. LANGUAGE PROCESSING AND THE HUMAN BRAIN.
}
\item [\checkbox]  Introduction to Language (Yule) | \textbf{ 1. The Origins of Language
}
\item [\checkbox]  Introduction to Language (Yule) | \textbf{ 2. Animals and Human Language
}
\item [\checkbox]  Introduction to Language (Yule) | \textbf{ 3. The Sounds of Language
}
\item [\checkbox]  Introduction to Language (Yule) | \textbf{ 4. The Sound Patterns of Language
}
\item [\checkbox]  Introduction to Language (Yule) | \textbf{ 5. Word Formation
}
\item [\checkbox]  Introduction to Language (Yule) | \textbf{ 6. Morphology
}
\item [\checkbox]  Introduction to Language (Yule) | \textbf{ 7. Grammar
}
\item [\checkbox]  Introduction to Language (Yule) | \textbf{ 8. Syntax
}
\item [\checkbox]  Introduction to Language (Yule) | \textbf{ 9. Semantics
}
\item [\checkbox]  Introduction to Language (Yule) | \textbf{ 10. Pragmatics
}
\item [\checkbox]  Introduction to Language (Yule) | \textbf{ 11. Discourse Analysis
}
\item [\checkbox]  Introduction to Language (Yule) | \textbf{ 12. Language and the Brain
}
\item [\checkbox]  Introduction to Language (Yule) | \textbf{ 13. First Language Acquisition
}
\item [\checkbox]  Introduction to Language (Yule) | \textbf{ 14. Second Language Acquisition/Learning
}
\item [\checkbox]  Introduction to Language (Yule) | \textbf{ 15. Gestures and Sign Languages
}
\item [\checkbox]  Introduction to Language (Yule) | \textbf{ 16. Written Language
}
\item [\checkbox]  Introduction to Language (Yule) | \textbf{ 17. Language History and Change
}
\item [\checkbox]  Introduction to Language (Yule) | \textbf{ 18. Regional Variation in Language
}
\item [\checkbox]  Introduction to Language (Yule) | \textbf{ 19. Social Variation in Language
}
\item [\checkbox]  Introduction to Language (Yule) | \textbf{ 20. Language and Culture
}
\item [\checkbox]  Introduction to Language (Yule) | \textbf{ A. Glossary
}
\item [\checkbox]  Psycholinguistics - The Key Concepts | \textbf{ Access code - Attrition
}
\item [\checkbox]  Psycholinguistics - The Key Concepts | \textbf{ Auditory perception - Chunking
}
\item [\checkbox]  Psycholinguistics - The Key Concepts | \textbf{ Cluttering - Design features
}
\item [\checkbox]  Psycholinguistics - The Key Concepts | \textbf{ Deviance - Fluency
}
\item [\checkbox]  Psycholinguistics - The Key Concepts | \textbf{ Focus - Incidental learning
}
\item [\checkbox]  Psycholinguistics - The Key Concepts | \textbf{ Indeterminacy - Language universals
}
\item [\checkbox]  Psycholinguistics - The Key Concepts | \textbf{ Latency - Memory
}
\item [\checkbox]  Psycholinguistics - The Key Concepts | \textbf{ Mental model - Parallel distributed processing
}
\item [\checkbox]  Psycholinguistics - The Key Concepts | \textbf{ Parallel processing - Processing
}
\item [\checkbox]  Psycholinguistics - The Key Concepts | \textbf{ Proposition - Search model
}
\item [\checkbox]  Psycholinguistics - The Key Concepts | \textbf{ Second language acquisition (SLA) - Speech perception: autonomous vs interactive
}
\item [\checkbox]  Psycholinguistics - The Key Concepts | \textbf{ Speech perception: phoneme variation - Twins
}
\item [\checkbox]  Psycholinguistics - The Key Concepts | \textbf{ Typing - Zipf’s law
}
\item [\checkbox]  The Sounds of the World's Languages | \textbf{ 1. The Sounds of the World's Languages
}
\item [\checkbox]  The Sounds of the World's Languages | \textbf{ 2. Places of Articulation
}
\item [\checkbox]  The Sounds of the World's Languages | \textbf{ 3. Stops
}
\item [\checkbox]  The Sounds of the World's Languages | \textbf{ 4. Nasals and Nasalized Consonants
}
\item [\checkbox]  The Sounds of the World's Languages | \textbf{ 5. Fricatives
}
\item [\checkbox]  The Sounds of the World's Languages | \textbf{ 6. Laterals
}
\item [\checkbox]  The Sounds of the World's Languages | \textbf{ 7. Rhotics
}
\item [\checkbox]  The Sounds of the World's Languages | \textbf{ 8. Clicks
}
\item [\checkbox]  The Sounds of the World's Languages | \textbf{ 9. Vowels
}
\item [\checkbox]  The Sounds of the World's Languages | \textbf{ 10. Multiple Articulatory Gestures
}
\item [\checkbox]  The Sounds of the World's Languages | \textbf{ 11. Coda
}
\item [\checkbox]  Acoustic Phonetics | \textbf{ 1. Anatomy and Physiology of Speech Production
}
\item [\checkbox]  Acoustic Phonetics | \textbf{ 2. Source Mechanisms
}
\item [\checkbox]  Acoustic Phonetics | \textbf{ 3. Basic Acoustics of Vocal Tract Resonators
}
\item [\checkbox]  Acoustic Phonetics | \textbf{ 4. Auditory Processing of Speechlike Sounds
}
\item [\checkbox]  Acoustic Phonetics | \textbf{ 5. Phonological Representation of Utterances
}
\item [\checkbox]  Acoustic Phonetics | \textbf{ 6. Vowels: Acoustic Events with a Relatively Open Vocal Tract
}
\item [\checkbox]  Acoustic Phonetics | \textbf{ 7. The Basic Stop Consonants: Bursts and Formant Transitions
}
\item [\checkbox]  Acoustic Phonetics | \textbf{ 8. Obstruent Consonants
}
\item [\checkbox]  Acoustic Phonetics | \textbf{ 9. Sonorant Consonants
}
\item [\checkbox]  Acoustic Phonetics | \textbf{ 10. Some Influences of Context on Speech Sound Production
}
\item [\checkbox]  Acoustic Phonetics | \textbf{ A. Notes
}
\item [\checkbox]  Dictionary of Linguistics and Phonetics | \textbf{ 1. A-C
}
\item [\checkbox]  Dictionary of Linguistics and Phonetics | \textbf{ 2. D-F
}
\item [\checkbox]  Dictionary of Linguistics and Phonetics | \textbf{ 3. G-I
}
\item [\checkbox]  Dictionary of Linguistics and Phonetics | \textbf{ 4. J-L
}
\item [\checkbox]  Dictionary of Linguistics and Phonetics | \textbf{ 5. M-O
}
\item [\checkbox]  Dictionary of Linguistics and Phonetics | \textbf{ 6. P-Q
}
\item [\checkbox]  Dictionary of Linguistics and Phonetics | \textbf{ 7. R-S
}
\item [\checkbox]  Dictionary of Linguistics and Phonetics | \textbf{ 8. T-V
}
\item [\checkbox]  Dictionary of Linguistics and Phonetics | \textbf{ 9. W-Z
}
\item [\checkbox]  Articulatory Phonetics | \textbf{ 1. Sound Identification
}
\item [\checkbox]  Articulatory Phonetics | \textbf{ 2. Face Diagrams
}
\item [\checkbox]  Articulatory Phonetics | \textbf{ 3. Fricatives
}
\item [\checkbox]  Articulatory Phonetics | \textbf{ 4. Stops
}
\item [\checkbox]  Articulatory Phonetics | \textbf{ 5. Vowels
}
\item [\checkbox]  Articulatory Phonetics | \textbf{ 6. Nasals
}
\item [\checkbox]  Articulatory Phonetics | \textbf{ 7. More Vowels
}
\item [\checkbox]  Articulatory Phonetics | \textbf{ 8. Tracking
}
\item [\checkbox]  Articulatory Phonetics | \textbf{ 9. Sibilants
}
\item [\checkbox]  Articulatory Phonetics | \textbf{ 10. Uses of Pitch Variation
}
\item [\checkbox]  Articulatory Phonetics | \textbf{ 11. Stress
}
\item [\checkbox]  Articulatory Phonetics | \textbf{ 12. Nasalized Vowels
}
\item [\checkbox]  Articulatory Phonetics | \textbf{ 13. Laterals
}
\item [\checkbox]  Articulatory Phonetics | \textbf{ 14. Length
}
\item [\checkbox]  Articulatory Phonetics | \textbf{ 15. Voiceless Vowels
}
\item [\checkbox]  Articulatory Phonetics | \textbf{ 16. Affricates
}
\item [\checkbox]  Articulatory Phonetics | \textbf{ 17. Glottal Consonants
}
\item [\checkbox]  Articulatory Phonetics | \textbf{ 18. Central Approximants
}
\item [\checkbox]  Articulatory Phonetics | \textbf{ 19. Review Exercises and Tables (I)
}
\item [\checkbox]  Articulatory Phonetics | \textbf{ 20. Palatal and Uvular Consonants
}
\item [\checkbox]  Articulatory Phonetics | \textbf{ 21. Syllabic Consonants and Prenasalization
}
\item [\checkbox]  Articulatory Phonetics | \textbf{ 22. Transition and Release of Consonants
}
\item [\checkbox]  Articulatory Phonetics | \textbf{ 23. Speech Styles
}
\item [\checkbox]  Articulatory Phonetics | \textbf{ 24. Fronting and Retroflexion
}
\item [\checkbox]  Articulatory Phonetics | \textbf{ 25. Ejectives
}
\item [\checkbox]  Articulatory Phonetics | \textbf{ 26. Flaps and Trills
}
\item [\checkbox]  Articulatory Phonetics | \textbf{ 27. States of the Glottis
}
\item [\checkbox]  Articulatory Phonetics | \textbf{ 28. Implosives
}
\item [\checkbox]  Articulatory Phonetics | \textbf{ 29. Breathy Stops and Affricates
}
\item [\checkbox]  Articulatory Phonetics | \textbf{ 30. Pharyngeal and Epiglottal Consonants
}
\item [\checkbox]  Articulatory Phonetics | \textbf{ 31. Secondary Articulations
}
\item [\checkbox]  Articulatory Phonetics | \textbf{ 32. Consonant Clusters, Vowel Clusters, and Vowel Glides
}
\item [\checkbox]  Articulatory Phonetics | \textbf{ 33. Double Articulations
}
\item [\checkbox]  Articulatory Phonetics | \textbf{ 34. Tongue Root Placement and Vowels
}
\item [\checkbox]  Articulatory Phonetics | \textbf{ 35. Fortis and Lenis Consonants; Controlled and Ballistic Syllables
}
\item [\checkbox]  Articulatory Phonetics | \textbf{ 36. Clicks
}
\item [\checkbox]  Articulatory Phonetics | \textbf{ 37. Palatography
}
\item [\checkbox]  Articulatory Phonetics | \textbf{ 38. Miscellaneous Final Details
}
\item [\checkbox]  Articulatory Phonetics | \textbf{ 39. Review Exercises and Tables (II)
}
\item [\checkbox]  Comprehensive Articulatory Phonetics | \textbf{ 1. Introduction to Sounds
}
\item [\checkbox]  Comprehensive Articulatory Phonetics | \textbf{ 2. Fricatives and Voicing
}
\item [\checkbox]  Comprehensive Articulatory Phonetics | \textbf{ 3. Pitch Variations
}
\item [\checkbox]  Comprehensive Articulatory Phonetics | \textbf{ 4. Stops and Voice Onset Time
}
\item [\checkbox]  Comprehensive Articulatory Phonetics | \textbf{ 5. Facial Diagrams
}
\item [\checkbox]  Comprehensive Articulatory Phonetics | \textbf{ 6. Progressive Pitch Control
}
\item [\checkbox]  Comprehensive Articulatory Phonetics | \textbf{ 7. Aspiration and Glottal Stops
}
\item [\checkbox]  Comprehensive Articulatory Phonetics | \textbf{ 8. Advanced Intonation
}
\item [\checkbox]  Comprehensive Articulatory Phonetics | \textbf{ 9. Affricates
}
\item [\checkbox]  Comprehensive Articulatory Phonetics | \textbf{ 10. Introduction to Vowels
}
\item [\checkbox]  Comprehensive Articulatory Phonetics | \textbf{ 11. Characteristics of Syllables
}
\item [\checkbox]  Comprehensive Articulatory Phonetics | \textbf{ 12. Vowel Glides
}
\item [\checkbox]  Comprehensive Articulatory Phonetics | \textbf{ 13. Fronting, Retroflexion, and Sibilants
}
\item [\checkbox]  Comprehensive Articulatory Phonetics | \textbf{ 14. Back Vowels
}
\item [\checkbox]  Comprehensive Articulatory Phonetics | \textbf{ 15. Nasals175
}
\item [\checkbox]  Comprehensive Articulatory Phonetics | \textbf{ 16. Front Vowels
}
\item [\checkbox]  Comprehensive Articulatory Phonetics | \textbf{ 17. Laterals
}
\item [\checkbox]  Comprehensive Articulatory Phonetics | \textbf{ 18. Open Vowels and Length
}
\item [\checkbox]  Comprehensive Articulatory Phonetics | \textbf{ 19. Flaps and Trills
}
\item [\checkbox]  Comprehensive Articulatory Phonetics | \textbf{ 20. Central Vowels and Approximants
}
\item [\checkbox]  Comprehensive Articulatory Phonetics | \textbf{ 21. Alveopalatal Stops
}
\item [\checkbox]  Comprehensive Articulatory Phonetics | \textbf{ 22. Vowel and Glide Clusters
}
\item [\checkbox]  Comprehensive Articulatory Phonetics | \textbf{ 23. Palatal and Uvular Consonants
}
\item [\checkbox]  Comprehensive Articulatory Phonetics | \textbf{ 24. Nasalized Vowels
}
\item [\checkbox]  Comprehensive Articulatory Phonetics | \textbf{ 25. Double Articulations and Prenasalization
}
\item [\checkbox]  Comprehensive Articulatory Phonetics | \textbf{ 26. Front Rounded and Back Unrounded Vowels
}
\item [\checkbox]  Comprehensive Articulatory Phonetics | \textbf{ 27. Transition and Release
}
\item [\checkbox]  Comprehensive Articulatory Phonetics | \textbf{ 28. States of the Glottis
}
\item [\checkbox]  Comprehensive Articulatory Phonetics | \textbf{ 29. Implosives
}
\item [\checkbox]  Comprehensive Articulatory Phonetics | \textbf{ 30. Breathy Consonants and Consonant Clusters
}
\item [\checkbox]  Comprehensive Articulatory Phonetics | \textbf{ 31. Ejectives
}
\item [\checkbox]  Comprehensive Articulatory Phonetics | \textbf{ 32. Tongue Root Placement
}
\item [\checkbox]  Comprehensive Articulatory Phonetics | \textbf{ 33. Secondary Articulations
}
\item [\checkbox]  Comprehensive Articulatory Phonetics | \textbf{ 34. Fortis and Lenis Articulation
}
\item [\checkbox]  Comprehensive Articulatory Phonetics | \textbf{ 35. Clicks
}
\item [\checkbox]  Comprehensive Articulatory Phonetics | \textbf{ 36. Speech Styles
}
\item [\checkbox]  Comprehensive Articulatory Phonetics | \textbf{ A. Appendix
}
\item [\checkbox]  Elements of Acoustical Phonetics | \textbf{ 1. Sound Waves
}
\item [\checkbox]  Elements of Acoustical Phonetics | \textbf{ 2. Loudness and Pitch
}
\item [\checkbox]  Elements of Acoustical Phonetics | \textbf{ 3. Quality
}
\item [\checkbox]  Elements of Acoustical Phonetics | \textbf{ 4. Wave Analysis
}
\item [\checkbox]  Elements of Acoustical Phonetics | \textbf{ 5. Resonance
}
\item [\checkbox]  Elements of Acoustical Phonetics | \textbf{ 6. Hearing
}
\item [\checkbox]  Elements of Acoustical Phonetics | \textbf{ 7. The Production of Speech
}
\item [\checkbox]  Elements of Acoustical Phonetics | \textbf{ 8. Resonances of the Vocal Tract
}
\item [\checkbox]  Elements of Acoustical Phonetics | \textbf{ 9. Digital Speech Processing
}
\item [\checkbox]  Elements of Acoustical Phonetics | \textbf{ 10. Fourier Analysis
}
\item [\checkbox]  Elements of Acoustical Phonetics | \textbf{ 11. Digital Filters and LPC Analysis
}
\item [\checkbox]  A Practical Introduction to Phonetics | \textbf{ 1. Introduction
}
\item [\checkbox]  A Practical Introduction to Phonetics | \textbf{ 2. Basic Components of Speech
}
\item [\checkbox]  A Practical Introduction to Phonetics | \textbf{ 3. Phonation: A Third Basic Component
}
\item [\checkbox]  A Practical Introduction to Phonetics | \textbf{ 4. Articulation: Stricture Types
}
\item [\checkbox]  A Practical Introduction to Phonetics | \textbf{ 5. Articulations: Locations
}
\item [\checkbox]  A Practical Introduction to Phonetics | \textbf{ 6. Co-articulation and Sequences
}
\item [\checkbox]  A Practical Introduction to Phonetics | \textbf{ 7. Vowels: Introduction
}
\item [\checkbox]  A Practical Introduction to Phonetics | \textbf{ 8. The Cardinal Vowels (CVs)
}
\item [\checkbox]  A Practical Introduction to Phonetics | \textbf{ 9. Prosodic Features
}
\item [\checkbox]  A Practical Introduction to Phonetics | \textbf{ 10. Sound-systems of Languages
}
\item [\checkbox]  A Practical Introduction to Phonetics | \textbf{ 11. Review
}
\item [\checkbox]  Mathematics of Language | \textbf{ 1. Basic Concepts of Set Theory
}
\item [\checkbox]  Mathematics of Language | \textbf{ 2. Relations and Functions
}
\item [\checkbox]  Mathematics of Language | \textbf{ 3. Properties and Relations
}
\item [\checkbox]  Mathematics of Language | \textbf{ 4. Infinities
}
\item [\checkbox]  Mathematics of Language | \textbf{ 5. Basic Concepts of Logic and Formal Systems
}
\item [\checkbox]  Mathematics of Language | \textbf{ 6. Statement Logic
}
\item [\checkbox]  Mathematics of Language | \textbf{ 7. Predicate Logic
}
\item [\checkbox]  Mathematics of Language | \textbf{ 8. Formal Systems, Axiomatization, and Model Theory
}
\item [\checkbox]  Mathematics of Language | \textbf{ B1. Alternative Notations and Connectives
}
\item [\checkbox]  Mathematics of Language | \textbf{ B2. Kleene's Three-Valued Logic
}
\item [\checkbox]  Mathematics of Language | \textbf{ 9. Basic Concepts of Algebra
}
\item [\checkbox]  Mathematics of Language | \textbf{ 10. Operational Structures
}
\item [\checkbox]  Mathematics of Language | \textbf{ 11. Lattices
}
\item [\checkbox]  Mathematics of Language | \textbf{ 12. Boolean and Heyting Algebras
}
\item [\checkbox]  Mathematics of Language | \textbf{ 13. Basic Concepts of English as a Formal Language
}
\item [\checkbox]  Mathematics of Language | \textbf{ 14. Generalized Quantifiers
}
\item [\checkbox]  Mathematics of Language | \textbf{ 15. Intensionality
}
\item [\checkbox]  Mathematics of Language | \textbf{ 16. Basic Concepts of Languages, Grammars, and Automata
}
\item [\checkbox]  Mathematics of Language | \textbf{ 17. Finite Automata. Regular Languages, and Type 3 Grammars
}
\item [\checkbox]  Mathematics of Language | \textbf{ 18. Pushdown Automata, Context Free Grmmars, and Languages
}
\item [\checkbox]  Mathematics of Language | \textbf{ 19. Turing Machines, Recursively Enumerable Languages and Grammars
}
\item [\checkbox]  Mathematics of Language | \textbf{ 20. Linear Bounded Automata, Context Sensitive Languages, and Type 1 Grammars
}
\item [\checkbox]  Mathematics of Language | \textbf{ 21. Languages between Context Free and Context Sensitive
}
\item [\checkbox]  Mathematics of Language | \textbf{ Transformational Grammars
}
\item [\checkbox]  Mathematics of Language | \textbf{ E1. The Grammar Hierarchy
}
\item [\checkbox]  Mathematics of Language | \textbf{ E2. Semantic Automata
}
\end{itemize}
\end{document}